% Abstract for the TUM report document
% Included by MAIN.TEX

%\clearemptydoublepage
\phantomsection
\addcontentsline{toc}{chapter}{Abstract}	

%\vspace*{2cm}
%{\Large \bf Abstract}
%\vspace{1cm}
\chapter*{Abstract}

Reaching out large-scale of people via Internet is a fast and cost efficient way comparing with postal mail or telephone. Therefore, email has been used not just for research, but also for marketing, customer support, and other data collection purposes. However, getting an acceptable response rates on the sent out emails requires additional efforts from the researchers' side. This thesis investigates a communication system, which contributes increasing the response rates while minimizing the burden on the researchers' side. 
\vspace{1cm}

To achieve this, the system constructs a workflow supporting researchers to extract information, providing rule based automated decision making mechanism on respondents' emails, and personalize the content of the emails with the respondents' information which is extracted from the current state or earlier conversations. System also provides an option to enable contribution of other researchers to interact with the workflow under the permission of the initial researcher. Therefore, distribution of the work can ease individual's efforts on the mass email communication. This feature can be further extended on enabling crowd workers on distribution of the work.
\vspace{1cm}

This thesis demonstrates that providing a proper workflow and the possibility of an assistant contribution, a mass email communication can be achieved as if each email is individually tailored to each recipient, which contributes to high response rates. Therefore, while it keeps the efforts at minimum on the creation of emails, it maximizes the scale on the number of people communicated.
