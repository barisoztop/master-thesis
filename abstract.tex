\clearemptydoublepage
\phantomsection
\addcontentsline{toc}{chapter}{Abstract}	

\chapter*{Abstract}

Reaching out to a large-scale audience via the Internet is a fast and cost-effective way compared with postal mail or telephone. Therefore, email has been used not just for research but also for marketing, customer support, and other data collection purposes. However, getting an acceptable response rate on the sent emails requires additional efforts on the researchers' side. This study investigates a communication system which contributes to increasing the response rate while minimizing the burden on the researchers' side. 
\vspace{1cm}

To achieve this, the system constructs a workflow which supports the researchers in extracting information, providing a rule-based and automated decision-making mechanism for respondents' emails, and personalizes the content of the emails with the respondents' information which is extracted from either current or earlier conversations. It also provides an option to enable contributions from other researchers such as assistants to interact with the workflow with the permission of the initial researcher. Therefore, the distribution of the work can ease an individual's efforts of the mass email communication. This feature can be further extended by enabling crowd assistants to contribute to nearly all phases of the communication flow and getting guidance or assistance from the lead researcher when required.
\vspace{1cm}

This study demonstrates that by providing a proper workflow and enabling the possibility of an assistant's contribution, effective and efficient mass email communication can be achieved in a way each email was individually tailored for each recipient, which can contribute to higher response rate. As it minimizes the effort required to create emails, it maximizes the scale of people communicated with.
