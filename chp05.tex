\chapter{Final Solution}
\label{chp:5:FinaSolu}
In this chapter, a final design of a mass communication system, named Myriad, will be introduced. Firstly, the improvements on the requirements will be compared to the ones defined for the prototype from previous chapter. Later, the actual product's features, and its architecture will be discussed.

\section{Improved Requirements}
\label{sec:5.1:ImprRequ}

Having seen the prototype in action made us to review the requirements, and bring the ones. Some of those requirements are shaped according to the feedbacks that we got at the Stanford \ac{HCI} group, as well as the organizations who do mass email communication on weekly basis to reach their community. Following sections will discussed those new features on the final product. 

\subsection{Assistant Support}
\label{subsec:5.1.1:AssiSupp}
As it it discussed in section \ref{subsec:4.1.1:Cust}, the standard that we want to achieve within a mass email communication is the most adequately personalized emails according to recipients with minimum effort on a researcher's side. The initial idea and the prototype brought several features as discussed in chapter \ref{chp:4:InitIdeaProt} for this purpose. However, if you consider the gold standard in the figure \ref{fig:ChartEffortCustom} we would like to achieve, the prototype still leaves effort on a researcher to accomplished a successful mass email campaign.
\vspace{1cm}

In order to minimize the efforts on the researcher's side, the additional assistants' involvement was considered. Therefore, a primary researcher will be able to share the tasks with permitted assistants. These tasks may be extracting information from the incoming answers or even writing replies to those answers. Hence, the primary researcher will only need to interact with the flow of a mass email campaign when there is situation where it requires primary researcher attention. However, the system still needs to provide the necessary features that we will see in the next sections to support the work flow in a mass email campaign, hence assistants will only need to interact with this work flow to let the continuation of the email campaign by providing answers with the email templates, and extract information as \ac{KVP}s.

\subsection{Dynamic Variables and \ac{KVP}s}
\label{subsec:5.1.2:DynmVariKVPs}
We introduced dynamic variables at the initial prototype, however it was only limited at the salutation of the email. Since it allows the personalization of emails easier, and email marketing applications also supports this feature in the same purpose in section \ref{subsec:3.3.3:EmaiMarktAppl}, the final product will also include this feature. As a result, application users can create \ac{KVP}s and use those keys in the content of the email message to be replaced dynamically by its value according to the recipients. Therefore, the extracted information from the emails will not just help us to gather information in an organized, but also use them to personalize the emails
\vspace{1cm}

As a result, instead of keeping the \ac{KVP}s in the system according to the responds, system should keep them according to the recipients itself. This is where the \ac{KVP}-idea differs from the prototype as well. So, we have now profiles of contacts for a campaign having all \ac{KVP}s of a recipients visible during the whole state of the conversation, not just at one message of the recipient. However, system should offer an option to hide individual \ac{KVP}s to avoid cluttering on the view, and keep the \ac{KVP} list with the ones that are actively used.

\vspace{1cm}
Importing the \ac{KVP}s should be done in several ways. One option is that system should be able to synchronize with an online spreadsheet, e.g. Google Spreadsheet, to get the \ac{KVP}s at the beginning of a campaign. This is a convenient way for the researchers since they are already familiar with spreadsheet environment. Other options should be a campaign-wide view and a contact specific view in the system. Also, system should allow to edit both keys and values in a campaign.

\subsection{Importing and Exporting Contacts and Their Information}
\label{subsec:5.1.3:ImpoExpoContInfo}
In the prototype, application user had to enter the basic information of the recipients such as first name, last name, and email address into the system manually. However, we realized while they were doing this, they use a spreadsheet and copy the contact information from there. This was also the case for their regular email client that they used for email campaigns. The applications that was investigated in section \ref{sec:3.3:Resul} had an option to import from spreadsheet to ease the process. Therefore, the system should offer an option to import contacts from the spreadsheet environment.
\vspace{1cm}

However, importing should not be limited only the basic information of the recipients. Since we already mentioned about importing \ac{KVP}s from spreadsheet in the previous section, system should detect and import the contact information and \ac{KVP}s related to that contact if it is available from a provided spreadsheet. 
\vspace{1cm}

System should provide a bi-directional synchronization, not just importing data from the spreadsheet. Therefore, system should provide an option to export contact information and their created \ac{KVP}s from the system to the spreadsheet. This will also gives a reporting functionality to the application users, where they can see all the contacts, and their extracted information from a campaign in one view.

\subsection{Interoperability with Other Email Clients}
\label{subsec:5.1.4:InteEmaiClie}
Even though we provide a new system for the users to conduct their mass email communication, there might be some cases where a mass email communication initiated in the user's regular email client, and the system is not aware of that campaign since it was not created with it. Application should provide an option to import those emails messages created with another email client in to the system by recognizing those annotated messages by the user.
\vspace{1cm}

Enabling the system to import email conversations from other email clients reduce the dependency on the application, and while a researcher continue to use their own email client, assigned assistants can take care of those imported emails by the researcher. We saw such a import feature in \ac{CRM} applications in section \ref{subsec:3.3.1:CRMAppl} as well, where a user is able to forward an email into the \ac{CRM} application, and system takes care of assigning the imported email to the corresponding recipient entry in the system.

\subsection{Automated Decision-Making and Notifications}
\label{subsec:5.1.5:AutoDeciMakiNoti}
Even though the involvement of assistants make the initial researcher's life easier, the system still needs to provide an automated approach to answer the emails whose status is clear in the flow of a mass email communication. Therefore, a rule based decision-making mechanism should be used, where users sets the values of the keys of \ac{KVP}s, and it triggers the action of sending an email to a respondent.
\vspace{1cm}

Since the application's only purpose is the managing mass email communication and each campaign results great amount of messages in the inbox, the system should provide notifications regarding what should be done next for each recipients. Labels added to the email conversation should state whose turn is next in the communication. For instance, by proving a label saying "You need to reply" to the application user, the state of the communication waits an action from a researcher or an assistant. In the same way the unread conversations and the conversation waiting an answer from the respondents should be also annotated in the similar fashion. System should also provide email notification to the assistants' email addresses to notify them there is an action waiting to be taken care of.

\section{Final System}
\label{sec:5.2:FinaSyst}
In this section, we will see that how the revised and improved requirements from section \ref{sec:5.1:ImprRequ} reflected to the final product, named "Myriad".

\subsection{Creating an Email Campaign}
\label{subsec:5.2.1:CreaEmaiCamp}

As it was the same case for the prototype in section \ref{subsec:4.2.2:ProtSyst}, Myriad also requires a Gmail account to work with. the reason behind is not just the popularity of the Gmail, but also Stanford University uses Google Apps by default. Therefore, each member of the university has a Google account to use with Myriad. This also provides flexibility to use Myriad, because there  is no registration form or sign in screen.
\vspace{1cm}

After user sign in to the system, the first screen that they will see is the campaigns overview screen, in where all the created campaigns including the ones that are shared with the user (as a assistant) are shown as in figure \ref{fig:CampaignsOverviewScreen}. It has a simple and a clean \ac{UI}, deferring from regular email clients to emphasize its focus on mass email communication.
\vspace{1cm}

\begin{figure}[htbp]
	\centering
	\includegraphics[width=1.00\textwidth]{imgs/CampaignsOverviewScreen.png}
	\caption[Myriad's Campaigns Overview Screen]{Myriad's Campaigns Overview Screen}
	\label{fig:CampaignsOverviewScreen}
\end{figure}

\subsection{Synchronization with Other Sytems}
\label{subsec:5.2.2:SyncOtheSyst}

Myriad is able to get the contacts' information and their \ac{KVP}s from Google Spreadsheet. This is a convenient way to import contacts into the system since many people already keeps their information in a spreadsheet environment as we discussed in section \ref{subsec:5.1.3:ImpoExpoContInfo.}. Therefore, Myriad offers a bi-directional syncing from and to the spreadsheet defined at the creation of a campaign.
\vspace{1cm}

The corresponding columns in the spreadsheet start with "first name", "last name", and "email address" as in the figure \ref{fig:GoogleSpreadsheet}. The rest of the columns will be acted as a \ac{KVP}, and imported into the system as well.

\begin{figure}[htbp]
	\centering
	\includegraphics[width=1.00\textwidth]{imgs/GoogleSpreadsheet.png}
	\caption[A Google Spreadsheet to Import into Myriad]{A Google Spreadsheet to Import into Myriad}
	\label{fig:GoogleSpreadsheet}
\end{figure}

As it was discussed in section \ref{subsec:5.1.4:InteEmaiClie}, importing existing email conversation as a campaign into the system is an important feature, since researchers may initiate conversations in their regular email client, and later on to make it more manageable, they can import them into Myriad. Myriad leverages Gmail's labeling feature which is equivalent to the \ac{IMAP} protocol's folders \citep{GoogleInc.2013}.  

  from other email client

The campaign creation screen 

\begin{figure}[htbp]
	\centering
	\includegraphics[width=1.00\textwidth]{imgs/CreateCampaign.png}
	\caption[Creating a Campaing in Myriad]{Creating a Campaing in Myriad}
	\label{fig:CreateCampaign}
\end{figure}

\section{Architecture}
\label{sec:5.3:FinaArch}

\section{Conclusion}
\label{sec:5.4:Conc}

\begin{comment}

--> Gmail labellari ni anlatirken, diger urunlerde import nasil email forwardingle yapiliyordu onu anlat kesin.


However, we still needs to supply a system, where it offers a work flow to make it happen. satisfy these 

distribution of the work

initial idea is reducing effort as we disscussed


Requirements'dan once urunu tanit screenshot'larla cunku cok zaman kaybedecen


write first about the revised requirements,
then describe the product with pictures,
then some technical details

\section{Product}
\label{sec:5.1:FinaProd}

\subsection{Improved Requirements}
\label{subsec:5.1.1:Cust}

\subsection{Final System}
\label{subsec:5.1.2:FinaSyst}

\subsection{Architecture}
\label{subsec:5.1.3:FinaArch}

-------------
chapter 6
-------------
Evaluation
- Experiences
- Statistics
- Conclusion

\end{comment}
