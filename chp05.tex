\chapter{Final Solution}
\label{chp:5:FinaSolu}
In this chapter, a final design of a mass communication system, named Myriad, will be introduced. Firstly, the improvements on the requirements will be compared to the ones defined for the prototype from previous chapter. Later, the actual product's features, and its architecture will be discussed.

\section{Improved Requirements}
\label{sec:5.1:ImprRequ}

Having seen the prototype in action made us to review the requirements, and bring the ones. Some of those requirements are shaped according to the feedbacks that we got at the Stanford \ac{HCI} group, as well as the organizations who do mass email communication on weekly basis to reach their community. Following sections will discussed those new features on the final product. 

\subsection{Assistant Support}
\label{subsec:5.1.1:AssiSupp}
As it it discussed in section \ref{subsec:4.1.1:Cust}, the standard that we want to achieve within a mass email communication is the most adequately personalized emails according to recipients with minimum effort on a researcher's side. The initial idea and the prototype brought several features as discussed in chapter \ref{chp:4:InitIdeaProt} for this purpose. However, if you consider the gold standard in the figure \ref{fig:ChartEffortCustom} we would like to achieve, the prototype still leaves effort on a researcher to accomplished a successful mass email campaign.
\vspace{1cm}

In order to minimize the efforts on the researcher's side, the additional assistants' involvement was considered. Therefore, a primary researcher will be able to share the tasks with permitted assistants. These tasks may be extracting information from the incoming answers or even writing replies to those answers. Hence, the primary researcher will only need to interact with the flow of a mass email campaign when there is situation where it requires primary researcher attention.
\vspace{1cm}

By making involvement of assistants possible on a mass email campaign, we distribute the tasks, and reduce the primary researcher's efforts to minimum.


\subsection{Dynamic Variables and \ac{KVP}s}
\label{subsec:5.1.2:DynmVariKVPs}
We introduced dynamic variables at the initial prototype, however it was only limited at the salutation of the email. Since it allows the personalization of emails easier, and email marketing applications also supports this feature in the same purpose in section \ref{subsec:3.3.3:EmaiMarktAppl}, the final product will also include this feature. As a result, application users can create \ac{KVP}s and use those keys in the content of the email message to be replaced dynamically by its value according to the recipients. Therefore, the extracted information from the emails will not just help us to gather information in an organized, but also use them to personalize the emails
\vspace{1cm}

As a result, instead of keeping the \ac{KVP}s in the system according to the responds, system should keep them according to the recipients itself. This is where the \ac{KVP}-idea differs from the prototype as well. So, we have now profiles of contacts for a campaign having all \ac{KVP}s of a recipients visible during the whole state of the conversation, not just at one message of the recipient. However, system should offer an option to hide individual \ac{KVP}s to avoid cluttering on the view, and keep the \ac{KVP} list with the ones that are actively used.

\vspace{1cm}
Importing the \ac{KVP}s should be done in several ways. One option is that system should be able to synchronize with an online spreadsheet, e.g. Google Spreadsheet, to get the \ac{KVP}s at the beginning of a campaign. This is a convenient way for the researchers since they are already familiar with spreadsheet environment. Other options should be a campaign-wide view and a contact specific view in the system. Also, system should allow to edit both keys and values in a campaign.

\subsection{Importing Contacts and Information}
\label{subsec:5.1.1:AssiSupp}

\subsection{Dependency on a Email Client}
\label{subsec:5.1.1:AssiSupp}
- gmail labels

\subsection{Notifications}
\label{subsec:5.1.1:AssiSupp}
- you are assigned as a assistant
- there is an action that you need to take

\subsection{Automated Decision-Making}
\label{subsec:5.1.1:AssiSupp}


rule-based approchdan bahset burda


\section{Final System}
\label{sec:5.2:FinaSyst}

\section{Architecture}
\label{sec:5.3:FinaArch}

\section{Conclusion}
\label{sec:5.4:Conc}

\begin{comment}
However, we still needs to supply a system, where it offers a work flow to make it happen. satisfy these 

distribution of the work

initial idea is reducing effort as we disscussed


Requirements'dan once urunu tanit screenshot'larla cunku cok zaman kaybedecen


write first about the revised requirements,
then describe the product with pictures,
then some technical details

\section{Product}
\label{sec:5.1:FinaProd}

\subsection{Improved Requirements}
\label{subsec:5.1.1:Cust}

\subsection{Final System}
\label{subsec:5.1.2:FinaSyst}

\subsection{Architecture}
\label{subsec:5.1.3:FinaArch}

-------------
chapter 6
-------------
Evaluation
- Experiences
- Statistics
- Conclusion

\end{comment}
