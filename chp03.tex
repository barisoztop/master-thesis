\chapter{Evaluation of Existing Tools}
\label{chp:3:EvalExisToo}
After building the foundation on related work on personalized mass email communication, this section will evaluate existing systems available in the market. 

\section{Tool Categories and Their Relation with Thesis}
\label{sec:3.1:SystCate}

There are three different tool catagories that are related with this thesis, and focusing on email communication directly or indirectly. Followings section will give a brief description of those are these product types, and their relation with this thesis:

\subsection{Customer Relationship Management (CRM)}
\label{subsec:3.1.1:Cust}
A \ac{CRM} application helps to manage customer relationships effectively, which is a topic studied both by academia and industry in recent years. Such applications play an important role in the marketing where organizations use more customer oriented instead of product or brand oriented marketing strategies. Therefore, each customer's economic value is different to the company, and organizations' customer relation strategies require to adapt their customer offerings and communication strategy personalized according to individual customer \citep{Reinartz2004}. 
\vspace{1cm}

One of the reason why this thesis considers \ac{CRM} applications to evaluate is its communication aspects of a company with their clients. Another reason is, as it is mentioned at section 2.3, the adequate amount of personalization in emails is crucial on response rates, and people's increased daily interactions with digital world make the true authentic personalization more rare. To achieve such a level of personalization requires getting know each recipient very well by considering not only the recent conversation, but also earlier conversations, and all the information that might be extracted from those conversations helping to build a relation with respondents. Since a \ac{CRM} system aims to keep track of each customer history regarding a product or a brand, such a data store could be leveraged to add adequate amount of personalized information to a email conversation. 

\subsection{Help Desk}
\label{subsec:3.1.2:HelpDeskSoft}
Another application that focuses on a company and its relation with their clients is help desk applications. It's main purpose to provide information and support related to a company's products and services to their customers. As a part of knowledge acquisition, help desks supports both sides of the communication in a way that while customers or end users find the knowledge they need, and the people who provide help by making the knowledge available and reusable \citep{Halverson2004}.
\vspace{1cm}

Reusing the existing knowledge requires to structure the captured knowledge. This is where it makes the relation with this thesis. Because, a help desk application provides a workflow where both parties develop a communication where person who needs assistance describes his/her problem while people who provide help identify the problem by looking earlier cases or asking questions to clarify the initial question. This also requires the cooperation of assistants while providing help to a problem at which one person might have previous experience to guide other assistants. As a result, a help desk application is similar to a mass email communication where a researcher initiate an open ended questionnaire, extracting information from the coming replies, and organize them according to the answers that researcher seeks for. In addition, respondents might also come up with some questions to clarify things, where existing answers can easily be reused. Having such a email conversation with large groups require great amount of effort from a researcher's side, where he might assign tasks to distribute the efforts to other researchers to deal with the large size of the group.


\subsection{Email Marketing}
\label{subsec:3.1.3:EmaiMarkt}
Organizations and marketers use email marketing for several reasons. Some of those purposes are brand and customer loyalty building, acquiring or converting customers, advertising the brand or the product, solicit sales or donation, communicating for promotional offers and even educational purposes. At the end, these approaches can be group under following categories \cite{Eley2009}:

\begin{compactitem}
	\item \textbf{Educational Communication:} An educational message is given in the form of newsletter, avoiding sale push, but it might still consists some content indirectly by encouraging recipients. For example, free monthly newsletter which contains tips about digital photography, and photography accessories used in the tips might be linked to an online shopping website. 
	\item \textbf{News and Updates:} To notify the customers about important updates or changes to a business. For instance, release of a new product, changes on contact details or major changes on a company's website
	\item \textbf{Direct Sales Messages:} Emails sent by others consists marketing ads, and clear message on offers.
	\item \textbf{Housekeeping:} Emails such as subscription confirmation messages or welcome emails. These messages are often system generated automated messages. However, they can be used to promote a message as well like offering a discount code a long with the registration confirmation email.
\end{compactitem}

Since these categories consists communication with large group of people, this thesis also evaluates existing tools in the market for email marketing including its technical aspects.

\section{Methodology}
\label{sec:3.2:Meth}

The analysis examined two products from the categories of \ac{CRM}, help desk, and email marketing. Selection of the products depends on the several product comparison websites including Toptenreviews.com\footnote{http://\{email-marketing-software-review, crm-software-review\}.toptenreviews.com/ }, Softwareshortlist.com\footnote{http://www.softwareshortlist.com/crm/solutions/}, as well as from the suggestions of Stanford HCI group members. In addition to those websites and suggestions, their demo or trial version availability were also considered, since some of the products required fee before using them. After the products were shortlisted, the last filtering was done by getting their web traffic rankings from Compete.com\footnote{https://www.compete.com/}, Alexa\footnote{http://www.alexa.com/}, and Google Trends\footnote{http://www.google.com/trends/}.

\section{Results}
\label{sec:3.3:Resul}
Evaluation of the products will be done according to their category. A brief description of the products will be presented. This description will mainly focus on the features which are related to support email communication as explained in section~\ref{sec:3.1:SystCate}. After that each category will be concluded with a comparison matrix of the selected products.

\subsection{Customer Relationship Management (CRM)}
\label{subsec:3.3.1:Cust}

\begin{table}[!ht]
\begin{center}
	%\renewcommand{\arraystretch}{2}
	%\tiny
	%\setlength{\tabcolsep}{5pt}
	\caption[Comparison Matrix for CRM Applications]{Comparison Matrix for CRM Applications} \label{tab:comp_matr_crm}
    \begin{tabular}{ p{3cm} p{3cm}  p{3cm} }
	\hline
	& \textbf{SugarCRM} & \textbf{Highrise} \\ \hline
	\textbf{Neutral power} & 143 (40.1) & 158 (44.4) \\
	\textbf{High power} & 150 (42.1) & 154 (43.3) \\ \hline
    \end{tabular}
\end{center}
\end{table}

\paragraph{SugarCRM}

\paragraph{Highrise}


\subsection{Help Desk}
\label{subsec:3.3.2:HelpDeskSoft}


\paragraph{Zendesk}
Cloud-based customer service software Zendesk \footnote{http://www.zendesk.com} provides a nice and clean user interface. Provided reports and analytics let users access to the key metrics that will help to make workflow improvements and increase efficiencies on customer support. Zendesk has more than 30,000 businesses from a wide variety of industries. Along with documentation of application usage, Zendesk also has a forum to give support on how to use \citep{Zendesk2013}.
\\
--> Clean UI \\
--> Provides reporting and analytics,\\

\paragraph{Kayako}

--> Complicated UI


\subsection{Email Marketing}
\label{subsec:3.1.3:EmaiMarkt}


\paragraph{MailChimp}

\paragraph{Constant Contact}



\begin{comment}
--> At the end, put the all the comparison tables together to the Appendix
--> These features are also helped us to add them into our app, decided by saying they all support so we can also support
\end{comment}
