\chapter{Conclusion and Future Work}
\label{chp:6:ConcFutu}

In this final chapter, the study will be finalized by summarizing it under the conclusion section, and its possible extensions will be discussed in the future work section.

\section{Conclusion}
\label{sec:6.1:Conc}

Increased Internet usage made email a popular medium for communication because of its low cost, quick turnover and the flexibility of connectivity through the use of mobile devices. The benefits of email communication also attracted researchers to use it as a data collection medium to explore, describe, and explain things by communicating with a large group of people.
\vspace{1cm}

However, the nature of communication with large groups is different compared to small ones due to the required effort in personalizing messages according to its respective recipients. As a result, researchers tend to write more generic emails, ignoring recipient-specific details. Researchers investigated many response rate influences, and addressed that personalization of the message content is an important factor in increasing the response rate \citep{Dillman1991,Schaefer1998}. Messages that are not personalized results in a low response rate for the answers expected from the recipients. As a result, this may end up with a nonresponse error.
\vspace{1cm}

Nonresponse error has been considered as a major problem by many researchers, because of the tendency of some of the respondents as a part of a large group who fails to respond might result in a biased estimate for the proper representation of the population. Hence, it affects the outcome of a research \citep{Bogen1996}.
\vspace{1cm}

For this reason, the researchers investigated possible theories regarding the reason behind proper personalization's effect on the response rate. While \cite{DillmanDonA.SmythJoleneD.Christian2009} emphasized the social exchange theory since the personalization of emails helps build a connection between the respondents and the researcher, \cite{Barron2002} stressed on the diffusion of responsibility theory, which talks about the awareness of the availability of the other volunteers will result in a higher utility of not volunteering.
\vspace{1cm}

The researchers conducted studies on investigating the diffusion of responsibility. \cite{Barron2002} showed that the number of replies where they used a single email address in the "To" field got a 20\% higher response rate and the number of "very helpful" replies was 187\% higher compared to the replies they got when they used groups of email addresses. In the study of \cite{Selm2006}, the recipients showed privacy concerns when the header of the email contained all the email addresses of the other respondents explicitly.
\vspace{1cm}

Other researchers investigated the social exchange theory. \cite{Heerwegh2005} applied personalized salutations in the the emails, and got a 6.9\% higher login rate for the provided survey link in the email compared to the non-personalized group in their study. In the study of \cite{Joinson2007}, when the level of authority or status of the sender was used with the personalized salutation, they got a 53.4\% response rate while non-personalized salutations with a neutral power of the sender status got a 40.1\% response rate.
\vspace{1cm}

Even though personalization of emails helps increase the response rates, \cite{DillmanDonA.SmythJoleneD.Christian2009} emphasized that an overwhelmed personalization can also result to a low response rate. Hence, only the adequate amount of personalization and its appropriateness should be considered for personalizing emails. In addition to this, experienced email users can easily notice if a message is manually written by a person, or if it is computer generated. Therefore, it is hard for the digital world to achieve authentic personalization, and achieving such level of personalization requires getting to know each recipient very well.
\vspace{1cm}

To understand what existing applications offer in support for personalized mass email communication, \ac{CRM}, Help Desk, and Email Marketing applications were evaluated in chapter \ref{chp:3:EvalExisAppl}. Several features of these applications are considered to be helpful to researchers for their mass email communication. Some of these features are the following:

\begin{compactitem}
	\item Storing client related information extracted from conversations in \ac{KVP}s.
	\item Integration with popular email clients such as Gmail.
	\item Importing recipient lists from third-party applications.
	\item Assigning emails to a recipient profile.
	\item Using dynamic variables in the emails to be replaced by their values.
	\item Assigning a help desk ticket to an assistant.
	\item Reusability of an existing email as a template.
\end{compactitem}
\vspace{1cm}

Moreover, none of the available products indicated above offers these mentioned features in helping a personalized mass email communication all under a single product and some of these features needs additional effort to use since their application focus is not a mass email communication. Therefore, this study came up with an initial solution as a prototype featured in chapter \ref{chp:4:InitIdeaProt}. The prototype supported following features to help a researcher reach large groups of people in a personalized way with less effort:

\begin{compactitem}
	\item Kept extracted information in \ac{KVP}s along with the recipient messages.
	\item Used dynamic variables to personalize the salutations of emails.
	\item Allowed to reuse earlier sent messages as a template.
	\item Provided a tree structure to get an overview of the state and flow of the conversation with the used templates.
\end{compactitem}
\vspace{1cm}

Even though the prototype provided the above mentioned features in support to a personalized mass email communication, we moved the prototype from its origin product, EmailValet, to a new project after the informal user test with an organization who performs mass email communication regularly and the \ac{HCI} group at Stanford University. The user test ended up with new features, and it helped us realize the limitations of the existing EmailValet as an email client we used for the prototype.
\vspace{1cm}

The improved requirements after user test brought up the final solution in chapter \ref{chp:5:FinaSolu}, and its features in support for mass email communication as follows:

\begin{compactitem}
	\item Recipients' information can be synchronized to the Myriad system through a Google Spreadsheet to remove the effort on importing and exporting recipients' information from the researchers.
	\item A researcher can assign assistants to an email campaign for them to interact with the flow of a mass email communication with tasks such as extracting information from incoming answers, proofreading the primary researcher's replies before sending them or even composing replies to those answers.
	\item Extracted respondents' information can be recorded to the recipients' respective profiles in the \ac{KVP}s during the whole state of a campaign. 
	\item Those recorded \ac{KVP}s can be used as dynamic variables in personalizing the content of the emails.
	\item Reusability of exiting emails and getting an overview of the state of the conversation.
	\item Provided a decision-making mechanism to automate earlier decisions on sending out emails in a user verified manner.
	\item Provided visual cues to notify users on the current state of communication with individual recipients, and suggesting potential action based on the previous decisions of the user.
	\item All conversations and extracted information of each recipient are provided under a single view respectively to conveniently reach all the necessary information to write recipient specific personalized emails.
 \end{compactitem}

Considering the features gathered in the final solution, we provided a workflow to the researchers aimed to help reduce the amount of time needed in personalizing the content of the emails. Considering the Effort vs. Personalization chart from figure \ref{fig:ChartEffortCustom} after the final solution, it could be replaced in a way as shown in figure \ref{fig:drawingEfforPersonalizationAnnotations}, compared with other products in the market focusing on email communication and data collection. Considering the user feedbacks and the amount of people actively using Myriad during its beta stage, we believe that Myriad was able to bridge the gap in achieving the gold standard of effort vs. personalization, as annotated in the figure.
\vspace{1cm}

Finally, possible future work on enhancing the existing solution and remove the known existing restrictions are to be discussed in the next section.

\begin{figure}[htbp]
	\centering
	\begin{pdfpic}
	    %LaTeX with PSTricks extensions
%%Creator: inkscape 0.48.2
%%Please note this file requires PSTricks extensions
\psset{xunit=.5pt,yunit=.5pt,runit=.5pt}
\begin{pspicture}(744.09448242,1052.36218262)
{
\newrgbcolor{curcolor}{1 1 1}
\pscustom[linestyle=none,fillstyle=solid,fillcolor=curcolor]
{
\newpath
\moveto(273.59164,714.8561659)
\curveto(273.59164,703.35393162)(263.46710135,694.0295284)(250.97785,694.0295284)
\curveto(238.48859865,694.0295284)(228.36406,703.35393162)(228.36406,714.8561659)
\curveto(228.36406,726.35840018)(238.48859865,735.6828034)(250.97785,735.6828034)
\curveto(263.46710135,735.6828034)(273.59164,726.35840018)(273.59164,714.8561659)
\closepath
}
}
{
\newrgbcolor{curcolor}{0 0 0}
\pscustom[linewidth=1.63443683,linecolor=curcolor]
{
\newpath
\moveto(273.59164,714.8561659)
\curveto(273.59164,703.35393162)(263.46710135,694.0295284)(250.97785,694.0295284)
\curveto(238.48859865,694.0295284)(228.36406,703.35393162)(228.36406,714.8561659)
\curveto(228.36406,726.35840018)(238.48859865,735.6828034)(250.97785,735.6828034)
\curveto(263.46710135,735.6828034)(273.59164,726.35840018)(273.59164,714.8561659)
\closepath
}
}
{
\newrgbcolor{curcolor}{1 1 1}
\pscustom[linestyle=none,fillstyle=solid,fillcolor=curcolor]
{
\newpath
\moveto(360.36423,790.5351959)
\curveto(360.36423,779.03296162)(350.23969135,769.7085584)(337.75044,769.7085584)
\curveto(325.26118865,769.7085584)(315.13665,779.03296162)(315.13665,790.5351959)
\curveto(315.13665,802.03743018)(325.26118865,811.3618334)(337.75044,811.3618334)
\curveto(350.23969135,811.3618334)(360.36423,802.03743018)(360.36423,790.5351959)
\closepath
}
}
{
\newrgbcolor{curcolor}{0 0 0}
\pscustom[linewidth=1.63443683,linecolor=curcolor]
{
\newpath
\moveto(360.36423,790.5351959)
\curveto(360.36423,779.03296162)(350.23969135,769.7085584)(337.75044,769.7085584)
\curveto(325.26118865,769.7085584)(315.13665,779.03296162)(315.13665,790.5351959)
\curveto(315.13665,802.03743018)(325.26118865,811.3618334)(337.75044,811.3618334)
\curveto(350.23969135,811.3618334)(360.36423,802.03743018)(360.36423,790.5351959)
\closepath
}
}
{
\newrgbcolor{curcolor}{1 1 1}
\pscustom[linestyle=none,fillstyle=solid,fillcolor=curcolor]
{
\newpath
\moveto(516.16048,886.6921059)
\curveto(516.16048,875.18987162)(506.03594135,865.8654684)(493.54669,865.8654684)
\curveto(481.05743865,865.8654684)(470.9329,875.18987162)(470.9329,886.6921059)
\curveto(470.9329,898.19434018)(481.05743865,907.5187434)(493.54669,907.5187434)
\curveto(506.03594135,907.5187434)(516.16048,898.19434018)(516.16048,886.6921059)
\closepath
}
}
{
\newrgbcolor{curcolor}{0 0 0}
\pscustom[linewidth=1.63443683,linecolor=curcolor]
{
\newpath
\moveto(516.16048,886.6921059)
\curveto(516.16048,875.18987162)(506.03594135,865.8654684)(493.54669,865.8654684)
\curveto(481.05743865,865.8654684)(470.9329,875.18987162)(470.9329,886.6921059)
\curveto(470.9329,898.19434018)(481.05743865,907.5187434)(493.54669,907.5187434)
\curveto(506.03594135,907.5187434)(516.16048,898.19434018)(516.16048,886.6921059)
\closepath
}
}
{
\newrgbcolor{curcolor}{1 1 1}
\pscustom[linestyle=none,fillstyle=solid,fillcolor=curcolor]
{
\newpath
\moveto(516.16048,752.2505059)
\curveto(516.16048,740.74827162)(506.03594135,731.4238684)(493.54669,731.4238684)
\curveto(481.05743865,731.4238684)(470.9329,740.74827162)(470.9329,752.2505059)
\curveto(470.9329,763.75274018)(481.05743865,773.0771434)(493.54669,773.0771434)
\curveto(506.03594135,773.0771434)(516.16048,763.75274018)(516.16048,752.2505059)
\closepath
}
}
{
\newrgbcolor{curcolor}{0 0 0}
\pscustom[linewidth=1.63443683,linecolor=curcolor]
{
\newpath
\moveto(516.16048,752.2505059)
\curveto(516.16048,740.74827162)(506.03594135,731.4238684)(493.54669,731.4238684)
\curveto(481.05743865,731.4238684)(470.9329,740.74827162)(470.9329,752.2505059)
\curveto(470.9329,763.75274018)(481.05743865,773.0771434)(493.54669,773.0771434)
\curveto(506.03594135,773.0771434)(516.16048,763.75274018)(516.16048,752.2505059)
\closepath
}
}
{
\newrgbcolor{curcolor}{0 0 0}
\pscustom[linewidth=1.31458414,linecolor=curcolor]
{
\newpath
\moveto(236.04839,979.78620262)
\lineto(236.03339,669.25502262)
\lineto(580.77272,668.52279262)
\lineto(580.77272,668.52279262)
}
}
{
\newrgbcolor{curcolor}{0 0 0}
\pscustom[linewidth=1.4242028,linecolor=curcolor]
{
\newpath
\moveto(236.04903,668.52279262)
\lineto(580.77288,979.78620262)
}
}
{
\newrgbcolor{curcolor}{0 0 0}
\pscustom[linestyle=none,fillstyle=solid,fillcolor=curcolor]
{
\newpath
\moveto(571.10952888,677.42618612)
\lineto(560.63985892,683.79533416)
\lineto(549.29141516,678.80190865)
\lineto(552.7646939,689.76083689)
\lineto(543.99828736,697.96317744)
\lineto(556.61456164,698.36701953)
\lineto(562.5450682,708.42977028)
\lineto(566.86907577,697.72043018)
\lineto(579.30073694,695.73721162)
\lineto(569.3568463,688.71463335)
\closepath
}
}
{
\newrgbcolor{curcolor}{0 0 0}
\pscustom[linestyle=none,fillstyle=solid,fillcolor=curcolor]
{
\newpath
\moveto(240.36951546,648.92511765)
\curveto(239.52642821,648.55506892)(238.65115366,648.27753303)(237.74368918,648.09250915)
\curveto(236.83620309,647.90748518)(235.8994544,647.81497322)(234.93344032,647.81497299)
\curveto(232.74963695,647.81497322)(231.01957923,648.3647586)(229.74326196,649.46433078)
\curveto(228.46693907,650.56918652)(227.82877903,652.06523712)(227.82877992,653.95248708)
\curveto(227.82877903,655.86087081)(228.47864843,657.36220782)(229.77839007,658.45650259)
\curveto(231.07812602,659.55077654)(232.85794852,660.09791873)(235.1178629,660.09793078)
\curveto(235.99020193,660.09791873)(236.82449373,660.02390916)(237.6207408,659.87590185)
\curveto(238.42282117,659.72787087)(239.17807479,659.50848536)(239.88650394,659.21774465)
\lineto(239.88650394,656.76749677)
\curveto(239.15465608,657.1428223)(238.42574851,657.42300139)(237.69977905,657.60803488)
\curveto(236.97964273,657.79304924)(236.25658984,657.88556121)(235.53061821,657.88557105)
\curveto(234.18403338,657.88556121)(233.14482781,657.54458854)(232.41299838,656.86265203)
\curveto(231.68701267,656.18598428)(231.32402255,655.21593027)(231.32402694,653.95248708)
\curveto(231.32402255,652.69960487)(231.67530331,651.73219406)(232.37787027,651.05025173)
\curveto(233.08042633,650.3683034)(234.07864915,650.02733074)(235.37254171,650.02733272)
\curveto(235.72381402,650.02733074)(236.04874872,650.04583313)(236.34734678,650.08283995)
\curveto(236.65178069,650.1251291)(236.92402328,650.18856587)(237.16407536,650.27315047)
\lineto(237.16407536,652.57273585)
\lineto(235.10029885,652.57273585)
\lineto(235.10029885,654.61857388)
\lineto(240.36951546,654.61857388)
\lineto(240.36951546,648.92511765)
}
}
{
\newrgbcolor{curcolor}{0 0 0}
\pscustom[linestyle=none,fillstyle=solid,fillcolor=curcolor]
{
\newpath
\moveto(253.01563574,652.50929901)
\lineto(253.01563574,651.70047932)
\lineto(245.66507857,651.70047932)
\curveto(245.74118542,651.03438953)(246.00757333,650.53482493)(246.46424309,650.20178402)
\curveto(246.92090329,649.8687388)(247.55906333,649.70221726)(248.37872512,649.70221892)
\curveto(249.04029719,649.70221726)(249.71651264,649.78944283)(250.40737352,649.96389588)
\curveto(251.10407163,650.14363149)(251.8183425,650.41323778)(252.55018828,650.77271557)
\lineto(252.55018828,648.58414465)
\curveto(251.80663314,648.33039701)(251.06308888,648.14008669)(250.31955324,648.01321311)
\curveto(249.57600034,647.88105319)(248.83245607,647.81497322)(248.08891821,647.81497299)
\curveto(246.30908931,647.81497322)(244.92445766,648.22202585)(243.93501911,649.03613212)
\curveto(242.95143075,649.8555228)(242.45963769,651.00267115)(242.45963846,652.47758059)
\curveto(242.45963769,653.92604918)(242.94264873,655.06526793)(243.90867303,655.89524025)
\curveto(244.88054757,656.72519687)(246.21541444,657.1401791)(247.91327766,657.1401882)
\curveto(249.45890676,657.1401791)(250.69424408,656.71991047)(251.61929334,655.87938104)
\curveto(252.55017741,655.03883594)(253.01562441,653.91547639)(253.01563574,652.50929901)
\moveto(249.78384955,653.45292198)
\curveto(249.78384146,653.99212916)(249.60820108,654.42561378)(249.25692789,654.75337716)
\curveto(248.91149424,655.08641352)(248.4577566,655.25293506)(247.8957136,655.25294226)
\curveto(247.28682075,655.25293506)(246.79210035,655.09698632)(246.41155092,654.78509558)
\curveto(246.03099204,654.47847776)(245.79387753,654.03442034)(245.70020668,653.45292198)
\lineto(249.78384955,653.45292198)
}
}
{
\newrgbcolor{curcolor}{0 0 0}
\pscustom[linestyle=none,fillstyle=solid,fillcolor=curcolor]
{
\newpath
\moveto(265.2753452,653.45292198)
\lineto(265.2753452,648.04493152)
\lineto(262.11381523,648.04493152)
\lineto(262.11381523,648.92511765)
\lineto(262.11381523,652.18418522)
\curveto(262.11380699,652.95070877)(262.09331561,653.47934856)(262.05234103,653.77010617)
\curveto(262.01720478,654.06085233)(261.95280331,654.27495144)(261.85913643,654.41240416)
\curveto(261.73618018,654.59742172)(261.56932182,654.74015446)(261.35856085,654.84060282)
\curveto(261.14778491,654.94632398)(260.90774306,654.99918796)(260.63843458,654.99919491)
\curveto(259.98270374,654.99918796)(259.46749196,654.76922965)(259.0927977,654.3093193)
\curveto(258.71809301,653.85468281)(258.53074328,653.22295826)(258.53074793,652.41414375)
\lineto(258.53074793,648.04493152)
\lineto(255.38678202,648.04493152)
\lineto(255.38678202,656.92608887)
\lineto(258.53074793,656.92608887)
\lineto(258.53074793,655.62563369)
\curveto(259.0049723,656.1436931)(259.50847472,656.52431375)(260.04125669,656.76749677)
\curveto(260.57402634,657.01594875)(261.16242161,657.1401791)(261.80644426,657.1401882)
\curveto(262.94224411,657.1401791)(263.80288196,656.82563843)(264.3883604,656.19656523)
\curveto(264.97967249,655.56747573)(265.2753338,654.65292889)(265.2753452,653.45292198)
}
}
{
\newrgbcolor{curcolor}{0 0 0}
\pscustom[linestyle=none,fillstyle=solid,fillcolor=curcolor]
{
\newpath
\moveto(278.02685142,652.50929901)
\lineto(278.02685142,651.70047932)
\lineto(270.67629424,651.70047932)
\curveto(270.75240109,651.03438953)(271.018789,650.53482493)(271.47545876,650.20178402)
\curveto(271.93211897,649.8687388)(272.57027901,649.70221726)(273.3899408,649.70221892)
\curveto(274.05151286,649.70221726)(274.72772832,649.78944283)(275.41858919,649.96389588)
\curveto(276.11528731,650.14363149)(276.82955818,650.41323778)(277.56140395,650.77271557)
\lineto(277.56140395,648.58414465)
\curveto(276.81784882,648.33039701)(276.07430455,648.14008669)(275.33076892,648.01321311)
\curveto(274.58721602,647.88105319)(273.84367175,647.81497322)(273.10013388,647.81497299)
\curveto(271.32030498,647.81497322)(269.93567333,648.22202585)(268.94623479,649.03613212)
\curveto(267.96264642,649.8555228)(267.47085336,651.00267115)(267.47085413,652.47758059)
\curveto(267.47085336,653.92604918)(267.9538644,655.06526793)(268.9198887,655.89524025)
\curveto(269.89176324,656.72519687)(271.22663011,657.1401791)(272.92449333,657.1401882)
\curveto(274.47012243,657.1401791)(275.70545976,656.71991047)(276.63050901,655.87938104)
\curveto(277.56139309,655.03883594)(278.02684009,653.91547639)(278.02685142,652.50929901)
\moveto(274.79506523,653.45292198)
\curveto(274.79505713,653.99212916)(274.61941675,654.42561378)(274.26814357,654.75337716)
\curveto(273.92270992,655.08641352)(273.46897227,655.25293506)(272.90692927,655.25294226)
\curveto(272.29803642,655.25293506)(271.80331602,655.09698632)(271.42276659,654.78509558)
\curveto(271.04220772,654.47847776)(270.80509321,654.03442034)(270.71142235,653.45292198)
\lineto(274.79506523,653.45292198)
}
}
{
\newrgbcolor{curcolor}{0 0 0}
\pscustom[linestyle=none,fillstyle=solid,fillcolor=curcolor]
{
\newpath
\moveto(287.70464573,654.50755941)
\curveto(287.42946699,654.62385371)(287.15429706,654.70843607)(286.87913513,654.76130677)
\curveto(286.60981189,654.81945043)(286.33756931,654.84852562)(286.06240656,654.84853242)
\curveto(285.25445364,654.84852562)(284.6309303,654.61328091)(284.19183466,654.1427976)
\curveto(283.75858309,653.67758848)(283.54195995,653.00885915)(283.54196461,652.13660759)
\lineto(283.54196461,648.04493152)
\lineto(280.3979987,648.04493152)
\lineto(280.3979987,656.92608887)
\lineto(283.54196461,656.92608887)
\lineto(283.54196461,655.46704159)
\curveto(283.94593282,656.04853794)(284.40845249,656.47144977)(284.92952498,656.73577836)
\curveto(285.45644008,657.00537596)(286.0858181,657.1401791)(286.81766094,657.1401882)
\curveto(286.92303723,657.1401791)(287.03720348,657.13489271)(287.16016002,657.12432899)
\curveto(287.28310001,657.11903351)(287.46166773,657.10317432)(287.69586371,657.07675136)
\lineto(287.70464573,654.50755941)
}
}
{
\newrgbcolor{curcolor}{0 0 0}
\pscustom[linestyle=none,fillstyle=solid,fillcolor=curcolor]
{
\newpath
\moveto(289.28540834,656.92608887)
\lineto(292.42937425,656.92608887)
\lineto(292.42937425,648.04493152)
\lineto(289.28540834,648.04493152)
\lineto(289.28540834,656.92608887)
\moveto(289.28540834,660.38339655)
\lineto(292.42937425,660.38339655)
\lineto(292.42937425,658.06795196)
\lineto(289.28540834,658.06795196)
\lineto(289.28540834,660.38339655)
}
}
{
\newrgbcolor{curcolor}{0 0 0}
\pscustom[linestyle=none,fillstyle=solid,fillcolor=curcolor]
{
\newpath
\moveto(303.41569378,656.6485527)
\lineto(303.41569378,654.33310811)
\curveto(302.98829273,654.59742172)(302.55797381,654.79301844)(302.12473571,654.91989886)
\curveto(301.69733595,655.04676554)(301.25238033,655.11020231)(300.7898675,655.11020938)
\curveto(299.91165878,655.11020231)(299.2266613,654.8776008)(298.73487302,654.41240416)
\curveto(298.24892986,653.95248117)(298.00596067,653.30754063)(298.00596472,652.47758059)
\curveto(298.00596067,651.64761169)(298.24892986,651.00002795)(298.73487302,650.53482742)
\curveto(299.2266613,650.07490832)(299.91165878,649.84495001)(300.7898675,649.84495181)
\curveto(301.28165373,649.84495001)(301.74710073,649.91102998)(302.1862099,650.04319193)
\curveto(302.6311573,650.17534988)(303.04098485,650.3709466)(303.41569378,650.62998268)
\lineto(303.41569378,648.30660848)
\curveto(302.92389126,648.14272989)(302.42331618,648.02114273)(301.91396704,647.94184666)
\curveto(301.41045667,647.8572644)(300.90402691,647.81497322)(300.39467625,647.81497299)
\curveto(298.620702,647.81497322)(297.23314301,648.22466905)(296.23199513,649.04406173)
\curveto(295.2308427,649.8687388)(294.73026762,651.01324394)(294.73026839,652.47758059)
\curveto(294.73026762,653.94190838)(295.2308427,655.08377032)(296.23199513,655.90316985)
\curveto(297.23314301,656.72784007)(298.620702,657.1401791)(300.39467625,657.1401882)
\curveto(300.90988159,657.1401791)(301.41631135,657.09788792)(301.91396704,657.01331452)
\curveto(302.4174615,656.93400959)(302.91803658,656.81242243)(303.41569378,656.6485527)
}
}
{
\newrgbcolor{curcolor}{0 0 0}
\pscustom[linestyle=none,fillstyle=solid,fillcolor=curcolor]
{
\newpath
\moveto(239.64938919,639.58404323)
\lineto(248.77391596,639.58404323)
\lineto(248.77391596,637.27652824)
\lineto(243.03046985,637.27652824)
\lineto(243.03046985,635.07209812)
\lineto(248.43141688,635.07209812)
\lineto(248.43141688,632.76458313)
\lineto(243.03046985,632.76458313)
\lineto(243.03046985,630.05265829)
\lineto(248.96712057,630.05265829)
\lineto(248.96712057,627.74514331)
\lineto(239.64938919,627.74514331)
\lineto(239.64938919,639.58404323)
}
}
{
\newrgbcolor{curcolor}{0 0 0}
\pscustom[linestyle=none,fillstyle=solid,fillcolor=curcolor]
{
\newpath
\moveto(260.91946043,635.15139416)
\curveto(261.31756799,635.70117214)(261.78886967,636.11879757)(262.33336689,636.40427172)
\curveto(262.8836947,636.69501494)(263.48672666,636.84039088)(264.14246459,636.84039998)
\curveto(265.27240384,636.84039088)(266.13304169,636.52585021)(266.72438073,635.89677701)
\curveto(267.3156869,635.26768751)(267.61134821,634.35314067)(267.61136553,633.15313376)
\lineto(267.61136553,627.74514331)
\lineto(264.44983556,627.74514331)
\lineto(264.44983556,632.37603249)
\curveto(264.45567608,632.44475103)(264.45860342,632.51611741)(264.45861759,632.59013182)
\curveto(264.4644581,632.66413655)(264.46738544,632.76986451)(264.46739961,632.90731601)
\curveto(264.46738544,633.5363922)(264.36492855,633.99102242)(264.16002864,634.27120803)
\curveto(263.955101,634.55666699)(263.62431162,634.69939974)(263.16765952,634.69940669)
\curveto(262.57046936,634.69939974)(262.10794969,634.47737103)(261.78009914,634.03331989)
\curveto(261.45808029,633.58925618)(261.29122193,632.94695883)(261.27952356,632.10642593)
\lineto(261.27952356,627.74514331)
\lineto(258.11799359,627.74514331)
\lineto(258.11799359,632.37603249)
\curveto(258.11798577,633.35929787)(258.0243109,633.99102242)(257.83696871,634.27120803)
\curveto(257.64961143,634.55666699)(257.31589471,634.69939974)(256.83581755,634.69940669)
\curveto(256.23277904,634.69939974)(255.76733204,634.47472783)(255.43947515,634.02539029)
\curveto(255.11160796,633.58132658)(254.94767694,632.94431564)(254.9476816,632.11435554)
\lineto(254.9476816,627.74514331)
\lineto(251.78615163,627.74514331)
\lineto(251.78615163,636.62630065)
\lineto(254.9476816,636.62630065)
\lineto(254.9476816,635.32584547)
\curveto(255.33408578,635.82804569)(255.77611406,636.20602314)(256.27376778,636.45977895)
\curveto(256.77726422,636.71351734)(257.33053141,636.84039088)(257.93357101,636.84039998)
\curveto(258.61270617,636.84039088)(259.21281079,636.69237174)(259.73388669,636.39634211)
\curveto(260.2549437,636.10029518)(260.65013455,635.68531294)(260.91946043,635.15139416)
}
}
{
\newrgbcolor{curcolor}{0 0 0}
\pscustom[linestyle=none,fillstyle=solid,fillcolor=curcolor]
{
\newpath
\moveto(274.97070409,631.74166411)
\curveto(274.31497409,631.74166011)(273.82025369,631.64121855)(273.48654141,631.44033913)
\curveto(273.15867493,631.23945231)(272.99474391,630.94341403)(272.99474786,630.5522234)
\curveto(272.99474391,630.19274553)(273.1264742,629.90992324)(273.3899391,629.70375568)
\curveto(273.65925001,629.50287061)(274.03102214,629.40242905)(274.50525662,629.4024307)
\curveto(275.09657377,629.40242905)(275.59422151,629.59273937)(275.99820133,629.97336225)
\curveto(276.40216725,630.35926706)(276.60415368,630.84032927)(276.60416124,631.41655032)
\lineto(276.60416124,631.74166411)
\lineto(274.97070409,631.74166411)
\moveto(279.77447323,632.81216076)
\lineto(279.77447323,627.74514331)
\lineto(276.60416124,627.74514331)
\lineto(276.60416124,629.0614577)
\curveto(276.18261678,628.5222438)(275.70838776,628.12840715)(275.18147275,627.87994659)
\curveto(274.65454549,627.63677215)(274.01345811,627.515185)(273.25820869,627.51518477)
\curveto(272.23949029,627.515185)(271.41105317,627.78214809)(270.77289485,628.31607485)
\curveto(270.14058777,628.85528686)(269.82443509,629.55309139)(269.82443586,630.40949051)
\curveto(269.82443509,631.45090823)(270.21962594,632.21479273)(271.0100096,632.70114629)
\curveto(271.80624402,633.18748994)(273.05329071,633.43066424)(274.75115339,633.43066993)
\lineto(276.60416124,633.43066993)
\lineto(276.60416124,633.65269886)
\curveto(276.60415368,634.10203677)(276.40802193,634.42979344)(276.01576538,634.63596985)
\curveto(275.62349491,634.84741888)(275.01168092,634.95314684)(274.18032159,634.95315404)
\curveto(273.50702835,634.95314684)(272.88057767,634.89235326)(272.30096767,634.77077313)
\curveto(271.72135117,634.64917896)(271.18272068,634.46679823)(270.68507457,634.22363041)
\lineto(270.68507457,636.38841251)
\curveto(271.35836106,636.53642301)(272.03457651,636.64743736)(272.71372297,636.72145591)
\curveto(273.3928621,636.8007429)(274.0720049,636.84039088)(274.75115339,636.84039998)
\curveto(276.52511551,636.84039088)(277.80436293,636.52320701)(278.58889949,635.88884741)
\curveto(279.37927166,635.25975791)(279.77446251,634.23419672)(279.77447323,632.81216076)
}
}
{
\newrgbcolor{curcolor}{0 0 0}
\pscustom[linestyle=none,fillstyle=solid,fillcolor=curcolor]
{
\newpath
\moveto(282.71645152,636.62630065)
\lineto(285.86041743,636.62630065)
\lineto(285.86041743,627.74514331)
\lineto(282.71645152,627.74514331)
\lineto(282.71645152,636.62630065)
\moveto(282.71645152,640.08360833)
\lineto(285.86041743,640.08360833)
\lineto(285.86041743,637.76816374)
\lineto(282.71645152,637.76816374)
\lineto(282.71645152,640.08360833)
}
}
{
\newrgbcolor{curcolor}{0 0 0}
\pscustom[linestyle=none,fillstyle=solid,fillcolor=curcolor]
{
\newpath
\moveto(288.8990019,640.08360833)
\lineto(292.04296781,640.08360833)
\lineto(292.04296781,627.74514331)
\lineto(288.8990019,627.74514331)
\lineto(288.8990019,640.08360833)
}
}
{
\newrgbcolor{curcolor}{0 0 0}
\pscustom[linestyle=none,fillstyle=solid,fillcolor=curcolor]
{
\newpath
\moveto(477.43436554,658.10310864)
\lineto(483.04608123,658.10310864)
\curveto(484.71465756,658.1030968)(485.99390498,657.76741053)(486.88382733,657.09604883)
\curveto(487.77958216,656.42995187)(488.22746512,655.47840024)(488.22747756,654.24139111)
\curveto(488.22746512,652.99907963)(487.77958216,652.04224161)(486.88382733,651.37087419)
\curveto(485.99390498,650.70478295)(484.71465756,650.37173988)(483.04608123,650.37174399)
\lineto(480.8154462,650.37174399)
\lineto(480.8154462,646.26420871)
\lineto(477.43436554,646.26420871)
\lineto(477.43436554,658.10310864)
\moveto(480.8154462,655.89074891)
\lineto(480.8154462,652.58410372)
\lineto(482.68601809,652.58410372)
\curveto(483.34173527,652.5840974)(483.84816503,652.72683014)(484.20530888,653.01230237)
\curveto(484.5624359,653.30304751)(484.74100361,653.71274335)(484.74101257,654.24139111)
\curveto(484.74100361,654.77002293)(484.5624359,655.17707556)(484.20530888,655.46255025)
\curveto(483.84816503,655.74800654)(483.34173527,655.89073928)(482.68601809,655.89074891)
\lineto(480.8154462,655.89074891)
}
}
{
\newrgbcolor{curcolor}{0 0 0}
\pscustom[linestyle=none,fillstyle=solid,fillcolor=curcolor]
{
\newpath
\moveto(500.28520148,650.7285762)
\lineto(500.28520148,649.91975651)
\lineto(492.9346443,649.91975651)
\curveto(493.01075116,649.25366672)(493.27713906,648.75410212)(493.73380882,648.42106121)
\curveto(494.19046903,648.08801599)(494.82862907,647.92149445)(495.64829086,647.92149611)
\curveto(496.30986293,647.92149445)(496.98607838,648.00872002)(497.67693926,648.18317307)
\curveto(498.37363737,648.36290868)(499.08790824,648.63251497)(499.81975401,648.99199275)
\lineto(499.81975401,646.80342184)
\curveto(499.07619888,646.5496742)(498.33265461,646.35936387)(497.58911898,646.23249029)
\curveto(496.84556608,646.10033038)(496.10202181,646.0342504)(495.35848395,646.03425017)
\curveto(493.57865505,646.0342504)(492.1940234,646.44130304)(491.20458485,647.25540931)
\curveto(490.22099648,648.07479999)(489.72920343,649.22194834)(489.7292042,650.69685778)
\curveto(489.72920343,652.14532637)(490.21221446,653.28454512)(491.17823877,654.11451744)
\curveto(492.1501133,654.94447406)(493.48498018,655.35945629)(495.18284339,655.35946539)
\curveto(496.72847249,655.35945629)(497.96380982,654.93918766)(498.88885908,654.09865823)
\curveto(499.81974315,653.25811313)(500.28519015,652.13475358)(500.28520148,650.7285762)
\moveto(497.05341529,651.67219917)
\curveto(497.05340719,652.21140635)(496.87776682,652.64489097)(496.52649363,652.97265435)
\curveto(496.18105998,653.30569071)(495.72732234,653.47221224)(495.16527934,653.47221945)
\curveto(494.55638648,653.47221224)(494.06166609,653.31626351)(493.68111666,653.00437277)
\curveto(493.30055778,652.69775495)(493.06344327,652.25369753)(492.96977242,651.67219917)
\lineto(497.05341529,651.67219917)
}
}
{
\newrgbcolor{curcolor}{0 0 0}
\pscustom[linestyle=none,fillstyle=solid,fillcolor=curcolor]
{
\newpath
\moveto(509.9629958,652.7268366)
\curveto(509.68781705,652.84313089)(509.41264713,652.92771326)(509.13748519,652.98058396)
\curveto(508.86816196,653.03872762)(508.59591937,653.0678028)(508.32075662,653.06780961)
\curveto(507.5128037,653.0678028)(506.88928036,652.8325581)(506.45018472,652.36207478)
\curveto(506.01693315,651.89686567)(505.80031002,651.22813634)(505.80031467,650.35588478)
\lineto(505.80031467,646.26420871)
\lineto(502.65634876,646.26420871)
\lineto(502.65634876,655.14536606)
\lineto(505.80031467,655.14536606)
\lineto(505.80031467,653.68631878)
\curveto(506.20428289,654.26781513)(506.66680255,654.69072696)(507.18787505,654.95505554)
\curveto(507.71479014,655.22465315)(508.34416816,655.35945629)(509.076011,655.35946539)
\curveto(509.1813873,655.35945629)(509.29555354,655.35416989)(509.41851008,655.34360618)
\curveto(509.54145007,655.3383107)(509.72001779,655.32245151)(509.95421377,655.29602855)
\lineto(509.9629958,652.7268366)
}
}
{
\newrgbcolor{curcolor}{0 0 0}
\pscustom[linestyle=none,fillstyle=solid,fillcolor=curcolor]
{
\newpath
\moveto(519.22803464,654.86782989)
\lineto(519.22803464,652.71097739)
\curveto(518.55473733,652.96471805)(517.90486793,653.15502837)(517.27842449,653.28190894)
\curveto(516.65196657,653.40877547)(516.06064396,653.47221224)(515.5044549,653.47221945)
\curveto(514.90727214,653.47221224)(514.46231652,653.40348907)(514.16958669,653.26604973)
\curveto(513.88270327,653.13388278)(513.73926363,652.92771326)(513.73926733,652.64754056)
\curveto(513.73926363,652.42021906)(513.8475752,652.24576793)(514.06420236,652.12418664)
\curveto(514.28667614,652.00259363)(514.68186699,651.91272486)(515.2497761,651.85458008)
\lineto(515.80304384,651.78321363)
\curveto(517.41307487,651.59818419)(518.49619054,651.29421631)(519.05239409,650.87130909)
\curveto(519.6085796,650.44839265)(519.88667686,649.78494971)(519.88668672,648.88097829)
\curveto(519.88667686,647.93471045)(519.50026803,647.22368993)(518.72745906,646.7479146)
\curveto(517.9546327,646.27213831)(516.80126089,646.0342504)(515.26734015,646.03425017)
\curveto(514.61746552,646.0342504)(513.9441774,646.08182799)(513.24747378,646.17698306)
\curveto(512.55661842,646.26685191)(511.84527488,646.40429826)(511.11344106,646.58932251)
\lineto(511.11344106,648.74617501)
\curveto(511.73989066,648.47127983)(512.38097804,648.26511032)(513.03670512,648.12766583)
\curveto(513.69828087,647.99021763)(514.36864165,647.92149445)(515.04778946,647.92149611)
\curveto(515.66252577,647.92149445)(516.12504543,647.99814722)(516.43534984,648.15145465)
\curveto(516.74564143,648.3047583)(516.90079044,648.53207341)(516.9007973,648.83340066)
\curveto(516.90079044,649.08714519)(516.79247887,649.27481231)(516.57586228,649.3964026)
\curveto(516.36508728,649.52327302)(515.94062303,649.62107138)(515.30246826,649.68979797)
\lineto(514.74920052,649.75323481)
\curveto(513.34992746,649.91182326)(512.36926868,650.20521834)(511.80722124,650.63342094)
\curveto(511.24517026,651.0616148)(510.96414565,651.71184174)(510.96414659,652.58410372)
\curveto(510.96414565,653.52507622)(511.32128109,654.22288074)(512.03555396,654.67751938)
\curveto(512.74982283,655.13214118)(513.84464786,655.35945629)(515.32003232,655.35946539)
\curveto(515.89964028,655.35945629)(516.50852692,655.31980831)(517.14669408,655.24052132)
\curveto(517.78484701,655.16121637)(518.4786265,655.03698602)(519.22803464,654.86782989)
}
}
{
\newrgbcolor{curcolor}{0 0 0}
\pscustom[linestyle=none,fillstyle=solid,fillcolor=curcolor]
{
\newpath
\moveto(526.93865361,653.32948657)
\curveto(526.24194059,653.3294795)(525.70916477,653.10216439)(525.34032457,652.64754056)
\curveto(524.97732986,652.19819035)(524.79583481,651.54796341)(524.79583885,650.69685778)
\curveto(524.79583481,649.84574329)(524.97732986,649.19287315)(525.34032457,648.7382454)
\curveto(525.70916477,648.28889911)(526.24194059,648.0642272)(526.93865361,648.064229)
\curveto(527.62364489,648.0642272)(528.14763869,648.28889911)(528.51063657,648.7382454)
\curveto(528.87361892,649.19287315)(529.05511398,649.84574329)(529.05512228,650.69685778)
\curveto(529.05511398,651.54796341)(528.87361892,652.19819035)(528.51063657,652.64754056)
\curveto(528.14763869,653.10216439)(527.62364489,653.3294795)(526.93865361,653.32948657)
\moveto(526.93865361,655.35946539)
\curveto(528.63064973,655.35945629)(529.9508799,654.94711726)(530.8993481,654.12244704)
\curveto(531.85365067,653.29776111)(532.33080703,652.15589917)(532.33081861,650.69685778)
\curveto(532.33080703,649.23780753)(531.85365067,648.09594558)(530.8993481,647.27126852)
\curveto(529.9508799,646.44658944)(528.63064973,646.0342504)(526.93865361,646.03425017)
\curveto(525.24079043,646.0342504)(523.91177824,646.44658944)(522.95161304,647.27126852)
\curveto(521.99729811,648.09594558)(521.52014175,649.23780753)(521.52014253,650.69685778)
\curveto(521.52014175,652.15589917)(521.99729811,653.29776111)(522.95161304,654.12244704)
\curveto(523.91177824,654.94711726)(525.24079043,655.35945629)(526.93865361,655.35946539)
}
}
{
\newrgbcolor{curcolor}{0 0 0}
\pscustom[linestyle=none,fillstyle=solid,fillcolor=curcolor]
{
\newpath
\moveto(544.51149277,651.67219917)
\lineto(544.51149277,646.26420871)
\lineto(541.3499628,646.26420871)
\lineto(541.3499628,647.14439484)
\lineto(541.3499628,650.4034624)
\curveto(541.34995457,651.16998596)(541.32946319,651.69862575)(541.28848861,651.98938336)
\curveto(541.25335236,652.28012952)(541.18895089,652.49422863)(541.095284,652.63168135)
\curveto(540.97232775,652.8166989)(540.80546939,652.95943165)(540.59470842,653.05988)
\curveto(540.38393249,653.16560117)(540.14389064,653.21846514)(539.87458215,653.2184721)
\curveto(539.21885131,653.21846514)(538.70363954,652.98850684)(538.32894528,652.52859648)
\curveto(537.95424059,652.07396)(537.76689085,651.44223545)(537.76689551,650.63342094)
\lineto(537.76689551,646.26420871)
\lineto(534.62292959,646.26420871)
\lineto(534.62292959,655.14536606)
\lineto(537.76689551,655.14536606)
\lineto(537.76689551,653.84491088)
\curveto(538.24111987,654.36297029)(538.74462229,654.74359094)(539.27740427,654.98677396)
\curveto(539.81017392,655.23522594)(540.39856919,655.35945629)(541.04259184,655.35946539)
\curveto(542.17839168,655.35945629)(543.03902954,655.04491562)(543.62450798,654.41584242)
\curveto(544.21582007,653.78675292)(544.51148137,652.87220608)(544.51149277,651.67219917)
}
}
{
\newrgbcolor{curcolor}{0 0 0}
\pscustom[linestyle=none,fillstyle=solid,fillcolor=curcolor]
{
\newpath
\moveto(551.85326592,650.26072952)
\curveto(551.19753592,650.26072552)(550.70281552,650.16028396)(550.36910324,649.95940454)
\curveto(550.04123677,649.75851772)(549.87730575,649.46247944)(549.87730969,649.0712888)
\curveto(549.87730575,648.71181094)(550.00903603,648.42898865)(550.27250094,648.22282109)
\curveto(550.54181184,648.02193601)(550.91358398,647.92149445)(551.38781845,647.92149611)
\curveto(551.97913561,647.92149445)(552.47678334,648.11180478)(552.88076316,648.49242765)
\curveto(553.28472908,648.87833247)(553.48671552,649.35939468)(553.48672307,649.93561572)
\lineto(553.48672307,650.26072952)
\lineto(551.85326592,650.26072952)
\moveto(556.65703507,651.33122616)
\lineto(556.65703507,646.26420871)
\lineto(553.48672307,646.26420871)
\lineto(553.48672307,647.58052311)
\curveto(553.06517861,647.0413092)(552.59094959,646.64747256)(552.06403459,646.39901199)
\curveto(551.53710732,646.15583756)(550.89601994,646.0342504)(550.14077052,646.03425017)
\curveto(549.12205212,646.0342504)(548.293615,646.3012135)(547.65545669,646.83514026)
\curveto(547.0231496,647.37435227)(546.70699692,648.07215679)(546.70699769,648.92855592)
\curveto(546.70699692,649.96997364)(547.10218777,650.73385813)(547.89257143,651.2202117)
\curveto(548.68880586,651.70655535)(549.93585254,651.94972965)(551.63371523,651.94973533)
\lineto(553.48672307,651.94973533)
\lineto(553.48672307,652.17176427)
\curveto(553.48671552,652.62110218)(553.29058376,652.94885885)(552.89832722,653.15503526)
\curveto(552.50605674,653.36648429)(551.89424276,653.47221224)(551.06288343,653.47221945)
\curveto(550.38959018,653.47221224)(549.7631395,653.41141867)(549.1835295,653.28983854)
\curveto(548.60391301,653.16824436)(548.06528251,652.98586364)(547.56763641,652.74269581)
\lineto(547.56763641,654.90747792)
\curveto(548.24092289,655.05548841)(548.91713835,655.16650277)(549.5962848,655.24052132)
\curveto(550.27542394,655.31980831)(550.95456673,655.35945629)(551.63371523,655.35946539)
\curveto(553.40767735,655.35945629)(554.68692477,655.04227242)(555.47146133,654.40791281)
\curveto(556.26183349,653.77882332)(556.65702434,652.75326213)(556.65703507,651.33122616)
}
}
{
\newrgbcolor{curcolor}{0 0 0}
\pscustom[linestyle=none,fillstyle=solid,fillcolor=curcolor]
{
\newpath
\moveto(559.59901938,658.60267374)
\lineto(562.74298529,658.60267374)
\lineto(562.74298529,646.26420871)
\lineto(559.59901938,646.26420871)
\lineto(559.59901938,658.60267374)
}
}
{
\newrgbcolor{curcolor}{0 0 0}
\pscustom[linestyle=none,fillstyle=solid,fillcolor=curcolor]
{
\newpath
\moveto(565.78156173,655.14536606)
\lineto(568.92552764,655.14536606)
\lineto(568.92552764,646.26420871)
\lineto(565.78156173,646.26420871)
\lineto(565.78156173,655.14536606)
\moveto(565.78156173,658.60267374)
\lineto(568.92552764,658.60267374)
\lineto(568.92552764,656.28722915)
\lineto(565.78156173,656.28722915)
\lineto(565.78156173,658.60267374)
}
}
{
\newrgbcolor{curcolor}{0 0 0}
\pscustom[linestyle=none,fillstyle=solid,fillcolor=curcolor]
{
\newpath
\moveto(571.48110058,655.14536606)
\lineto(580.06114163,655.14536606)
\lineto(580.06114163,653.16296487)
\lineto(574.8797453,648.29418753)
\lineto(580.06114163,648.29418753)
\lineto(580.06114163,646.26420871)
\lineto(571.26154989,646.26420871)
\lineto(571.26154989,648.24660991)
\lineto(576.44294623,653.11538724)
\lineto(571.48110058,653.11538724)
\lineto(571.48110058,655.14536606)
}
}
{
\newrgbcolor{curcolor}{0 0 0}
\pscustom[linestyle=none,fillstyle=solid,fillcolor=curcolor]
{
\newpath
\moveto(592.2505952,650.7285762)
\lineto(592.2505952,649.91975651)
\lineto(584.90003803,649.91975651)
\curveto(584.97614488,649.25366672)(585.24253279,648.75410212)(585.69920255,648.42106121)
\curveto(586.15586275,648.08801599)(586.79402279,647.92149445)(587.61368458,647.92149611)
\curveto(588.27525665,647.92149445)(588.9514721,648.00872002)(589.64233298,648.18317307)
\curveto(590.33903109,648.36290868)(591.05330196,648.63251497)(591.78514774,648.99199275)
\lineto(591.78514774,646.80342184)
\curveto(591.0415926,646.5496742)(590.29804834,646.35936387)(589.5545127,646.23249029)
\curveto(588.8109598,646.10033038)(588.06741553,646.0342504)(587.32387767,646.03425017)
\curveto(585.54404877,646.0342504)(584.15941712,646.44130304)(583.16997857,647.25540931)
\curveto(582.18639021,648.07479999)(581.69459715,649.22194834)(581.69459792,650.69685778)
\curveto(581.69459715,652.14532637)(582.17760819,653.28454512)(583.14363249,654.11451744)
\curveto(584.11550703,654.94447406)(585.4503739,655.35945629)(587.14823712,655.35946539)
\curveto(588.69386622,655.35945629)(589.92920354,654.93918766)(590.8542528,654.09865823)
\curveto(591.78513687,653.25811313)(592.25058387,652.13475358)(592.2505952,650.7285762)
\moveto(589.01880901,651.67219917)
\curveto(589.01880092,652.21140635)(588.84316054,652.64489097)(588.49188735,652.97265435)
\curveto(588.1464537,653.30569071)(587.69271606,653.47221224)(587.13067306,653.47221945)
\curveto(586.52178021,653.47221224)(586.02705981,653.31626351)(585.64651038,653.00437277)
\curveto(585.2659515,652.69775495)(585.02883699,652.25369753)(584.93516614,651.67219917)
\lineto(589.01880901,651.67219917)
}
}
{
\newrgbcolor{curcolor}{0 0 0}
\pscustom[linestyle=none,fillstyle=solid,fillcolor=curcolor]
{
\newpath
\moveto(601.31364357,653.84491088)
\lineto(601.31364357,658.60267374)
\lineto(604.47517354,658.60267374)
\lineto(604.47517354,646.26420871)
\lineto(601.31364357,646.26420871)
\lineto(601.31364357,647.54880469)
\curveto(600.8803891,647.02545001)(600.40323274,646.64218616)(599.88217305,646.39901199)
\curveto(599.36109983,646.15583756)(598.75806786,646.0342504)(598.07307535,646.03425017)
\curveto(596.86115178,646.0342504)(595.8658563,646.46773503)(595.08718594,647.33470536)
\curveto(594.30851162,648.20695994)(593.91917545,649.32767629)(593.91917625,650.69685778)
\curveto(593.91917545,652.0660304)(594.30851162,653.18410356)(595.08718594,654.0510806)
\curveto(595.8658563,654.92332847)(596.86115178,655.35945629)(598.07307535,655.35946539)
\curveto(598.75221318,655.35945629)(599.35231781,655.23522594)(599.87339103,654.98677396)
\curveto(600.4003054,654.74359094)(600.8803891,654.36297029)(601.31364357,653.84491088)
\moveto(599.24108503,648.09594742)
\curveto(599.91436702,648.09594558)(600.42665146,648.3179743)(600.77793988,648.76203422)
\curveto(601.13506765,649.20608914)(601.31363536,649.85102968)(601.31364357,650.69685778)
\curveto(601.31363536,651.54267701)(601.13506765,652.18761755)(600.77793988,652.63168135)
\curveto(600.42665146,653.0757324)(599.91436702,653.29776111)(599.24108503,653.29776815)
\curveto(598.57364547,653.29776111)(598.06136103,653.0757324)(597.70423019,652.63168135)
\curveto(597.35294484,652.18761755)(597.17730446,651.54267701)(597.17730853,650.69685778)
\curveto(597.17730446,649.85102968)(597.35294484,649.20608914)(597.70423019,648.76203422)
\curveto(598.06136103,648.3179743)(598.57364547,648.09594558)(599.24108503,648.09594742)
}
}
{
\newrgbcolor{curcolor}{0 0 0}
\pscustom[linestyle=none,fillstyle=solid,fillcolor=curcolor]
{
\newpath
\moveto(514.75798321,637.80332042)
\lineto(523.88250998,637.80332042)
\lineto(523.88250998,635.49580543)
\lineto(518.13906387,635.49580543)
\lineto(518.13906387,633.2913753)
\lineto(523.5400109,633.2913753)
\lineto(523.5400109,630.98386031)
\lineto(518.13906387,630.98386031)
\lineto(518.13906387,628.27193548)
\lineto(524.07571459,628.27193548)
\lineto(524.07571459,625.96442049)
\lineto(514.75798321,625.96442049)
\lineto(514.75798321,637.80332042)
}
}
{
\newrgbcolor{curcolor}{0 0 0}
\pscustom[linestyle=none,fillstyle=solid,fillcolor=curcolor]
{
\newpath
\moveto(536.02805445,633.37067135)
\curveto(536.42616201,633.92044933)(536.89746369,634.33807476)(537.44196091,634.62354891)
\curveto(537.99228872,634.91429213)(538.59532068,635.05966807)(539.25105861,635.05967717)
\curveto(540.38099786,635.05966807)(541.24163571,634.7451274)(541.83297475,634.1160542)
\curveto(542.42428093,633.4869647)(542.71994223,632.57241786)(542.71995955,631.37241095)
\lineto(542.71995955,625.96442049)
\lineto(539.55842958,625.96442049)
\lineto(539.55842958,630.59530968)
\curveto(539.5642701,630.66402822)(539.56719744,630.73539459)(539.56721161,630.80940901)
\curveto(539.57305212,630.88341374)(539.57597946,630.98914169)(539.57599363,631.1265932)
\curveto(539.57597946,631.75566939)(539.47352257,632.21029961)(539.26862267,632.49048522)
\curveto(539.06369502,632.77594418)(538.73290565,632.91867693)(538.27625354,632.91868388)
\curveto(537.67906338,632.91867693)(537.21654371,632.69664821)(536.88869316,632.25259708)
\curveto(536.56667431,631.80853337)(536.39981596,631.16623602)(536.38811758,630.32570312)
\lineto(536.38811758,625.96442049)
\lineto(533.22658762,625.96442049)
\lineto(533.22658762,630.59530968)
\curveto(533.22657979,631.57857506)(533.13290492,632.21029961)(532.94556273,632.49048522)
\curveto(532.75820545,632.77594418)(532.42448873,632.91867693)(531.94441157,632.91868388)
\curveto(531.34137307,632.91867693)(530.87592606,632.69400501)(530.54806917,632.24466747)
\curveto(530.22020198,631.80060377)(530.05627097,631.16359282)(530.05627562,630.33363272)
\lineto(530.05627562,625.96442049)
\lineto(526.89474565,625.96442049)
\lineto(526.89474565,634.84557784)
\lineto(530.05627562,634.84557784)
\lineto(530.05627562,633.54512266)
\curveto(530.4426798,634.04732288)(530.88470808,634.42530033)(531.3823618,634.67905614)
\curveto(531.88585824,634.93279452)(532.43912543,635.05966807)(533.04216503,635.05967717)
\curveto(533.72130019,635.05966807)(534.32140481,634.91164893)(534.84248071,634.6156193)
\curveto(535.36353772,634.31957237)(535.75872858,633.90459013)(536.02805445,633.37067135)
}
}
{
\newrgbcolor{curcolor}{0 0 0}
\pscustom[linestyle=none,fillstyle=solid,fillcolor=curcolor]
{
\newpath
\moveto(550.07929811,629.9609413)
\curveto(549.42356811,629.9609373)(548.92884771,629.86049574)(548.59513543,629.65961632)
\curveto(548.26726895,629.4587295)(548.10333793,629.16269122)(548.10334188,628.77150058)
\curveto(548.10333793,628.41202272)(548.23506822,628.12920043)(548.49853312,627.92303287)
\curveto(548.76784403,627.72214779)(549.13961617,627.62170623)(549.61385064,627.62170789)
\curveto(550.20516779,627.62170623)(550.70281553,627.81201656)(551.10679535,628.19263943)
\curveto(551.51076127,628.57854425)(551.7127477,629.05960646)(551.71275526,629.6358275)
\lineto(551.71275526,629.9609413)
\lineto(550.07929811,629.9609413)
\moveto(554.88306725,631.03143794)
\lineto(554.88306725,625.96442049)
\lineto(551.71275526,625.96442049)
\lineto(551.71275526,627.28073489)
\curveto(551.2912108,626.74152098)(550.81698178,626.34768434)(550.29006677,626.09922377)
\curveto(549.76313951,625.85604934)(549.12205213,625.73446219)(548.36680271,625.73446196)
\curveto(547.34808431,625.73446219)(546.51964719,626.00142528)(545.88148887,626.53535204)
\curveto(545.24918179,627.07456405)(544.93302911,627.77236857)(544.93302988,628.6287677)
\curveto(544.93302911,629.67018542)(545.32821996,630.43406991)(546.11860362,630.92042348)
\curveto(546.91483804,631.40676713)(548.16188473,631.64994143)(549.85974742,631.64994712)
\lineto(551.71275526,631.64994712)
\lineto(551.71275526,631.87197605)
\curveto(551.7127477,632.32131396)(551.51661595,632.64907063)(551.1243594,632.85524704)
\curveto(550.73208893,633.06669607)(550.12027494,633.17242402)(549.28891562,633.17243123)
\curveto(548.61562237,633.17242402)(547.98917169,633.11163045)(547.40956169,632.99005032)
\curveto(546.82994519,632.86845615)(546.2913147,632.68607542)(545.7936686,632.44290759)
\lineto(545.7936686,634.6076897)
\curveto(546.46695508,634.75570019)(547.14317053,634.86671455)(547.82231699,634.9407331)
\curveto(548.50145612,635.02002009)(549.18059892,635.05966807)(549.85974742,635.05967717)
\curveto(551.63370953,635.05966807)(552.91295696,634.7424842)(553.69749352,634.1081246)
\curveto(554.48786568,633.4790351)(554.88305653,632.45347391)(554.88306725,631.03143794)
}
}
{
\newrgbcolor{curcolor}{0 0 0}
\pscustom[linestyle=none,fillstyle=solid,fillcolor=curcolor]
{
\newpath
\moveto(557.82504554,634.84557784)
\lineto(560.96901145,634.84557784)
\lineto(560.96901145,625.96442049)
\lineto(557.82504554,625.96442049)
\lineto(557.82504554,634.84557784)
\moveto(557.82504554,638.30288552)
\lineto(560.96901145,638.30288552)
\lineto(560.96901145,635.98744093)
\lineto(557.82504554,635.98744093)
\lineto(557.82504554,638.30288552)
}
}
{
\newrgbcolor{curcolor}{0 0 0}
\pscustom[linestyle=none,fillstyle=solid,fillcolor=curcolor]
{
\newpath
\moveto(564.00759592,638.30288552)
\lineto(567.15156183,638.30288552)
\lineto(567.15156183,625.96442049)
\lineto(564.00759592,625.96442049)
\lineto(564.00759592,638.30288552)
}
}
{
\newrgbcolor{curcolor}{0 0 0}
\pscustom[linestyle=none,fillstyle=solid,fillcolor=curcolor]
{
\newpath
\moveto(164.04474148,707.96227442)
\lineto(167.42582214,707.96227442)
\lineto(167.42582214,698.43088948)
\lineto(173.36247285,698.43088948)
\lineto(173.36247285,696.12337449)
\lineto(164.04474148,696.12337449)
\lineto(164.04474148,707.96227442)
}
}
{
\newrgbcolor{curcolor}{0 0 0}
\pscustom[linestyle=none,fillstyle=solid,fillcolor=curcolor]
{
\newpath
\moveto(185.17430033,700.58774198)
\lineto(185.17430033,699.77892229)
\lineto(177.82374315,699.77892229)
\curveto(177.89985001,699.1128325)(178.16623791,698.6132679)(178.62290767,698.28022699)
\curveto(179.07956788,697.94718177)(179.71772792,697.78066023)(180.53738971,697.78066189)
\curveto(181.19896178,697.78066023)(181.87517723,697.8678858)(182.56603811,698.04233885)
\curveto(183.26273622,698.22207446)(183.97700709,698.49168075)(184.70885286,698.85115853)
\lineto(184.70885286,696.66258762)
\curveto(183.96529773,696.40883998)(183.22175346,696.21852965)(182.47821783,696.09165607)
\curveto(181.73466493,695.95949616)(180.99112066,695.89341618)(180.2475828,695.89341595)
\curveto(178.4677539,695.89341618)(177.08312225,696.30046882)(176.0936837,697.11457509)
\curveto(175.11009533,697.93396577)(174.61830227,699.08111412)(174.61830305,700.55602356)
\curveto(174.61830227,702.00449215)(175.10131331,703.1437109)(176.06733762,703.97368322)
\curveto(177.03921215,704.80363984)(178.37407903,705.21862207)(180.07194224,705.21863117)
\curveto(181.61757134,705.21862207)(182.85290867,704.79835344)(183.77795793,703.95782401)
\curveto(184.708842,703.11727891)(185.174289,701.99391936)(185.17430033,700.58774198)
\moveto(181.94251414,701.53136495)
\curveto(181.94250604,702.07057213)(181.76686567,702.50405675)(181.41559248,702.83182013)
\curveto(181.07015883,703.16485649)(180.61642119,703.33137802)(180.05437819,703.33138523)
\curveto(179.44548533,703.33137802)(178.95076494,703.17542929)(178.57021551,702.86353855)
\curveto(178.18965663,702.55692073)(177.95254212,702.11286331)(177.85887126,701.53136495)
\lineto(181.94251414,701.53136495)
}
}
{
\newrgbcolor{curcolor}{0 0 0}
\pscustom[linestyle=none,fillstyle=solid,fillcolor=curcolor]
{
\newpath
\moveto(195.22972284,704.72699567)
\lineto(195.22972284,702.57014317)
\curveto(194.55642553,702.82388383)(193.90655613,703.01419415)(193.28011269,703.14107472)
\curveto(192.65365477,703.26794125)(192.06233216,703.33137802)(191.5061431,703.33138523)
\curveto(190.90896034,703.33137802)(190.46400472,703.26265485)(190.17127489,703.12521551)
\curveto(189.88439147,702.99304856)(189.74095183,702.78687904)(189.74095554,702.50670633)
\curveto(189.74095183,702.27938484)(189.8492634,702.10493371)(190.06589056,701.98335242)
\curveto(190.28836434,701.86175941)(190.68355519,701.77189064)(191.2514643,701.71374586)
\lineto(191.80473204,701.64237941)
\curveto(193.41476307,701.45734997)(194.49787874,701.15338209)(195.05408229,700.73047487)
\curveto(195.6102678,700.30755843)(195.88836506,699.64411549)(195.88837492,698.74014407)
\curveto(195.88836506,697.79387623)(195.50195623,697.08285571)(194.72914726,696.60708038)
\curveto(193.9563209,696.13130409)(192.80294909,695.89341618)(191.26902835,695.89341595)
\curveto(190.61915372,695.89341618)(189.9458656,695.94099377)(189.24916199,696.03614884)
\curveto(188.55830662,696.12601769)(187.84696309,696.26346404)(187.11512926,696.44848829)
\lineto(187.11512926,698.60534079)
\curveto(187.74157886,698.33044561)(188.38266624,698.1242761)(189.03839332,697.98683161)
\curveto(189.69996907,697.84938341)(190.37032985,697.78066023)(191.04947766,697.78066189)
\curveto(191.66421397,697.78066023)(192.12673363,697.857313)(192.43703804,698.01062043)
\curveto(192.74732964,698.16392408)(192.90247864,698.39123919)(192.9024855,698.69256644)
\curveto(192.90247864,698.94631097)(192.79416707,699.13397809)(192.57755048,699.25556838)
\curveto(192.36677548,699.3824388)(191.94231124,699.48023716)(191.30415646,699.54896375)
\lineto(190.75088872,699.61240059)
\curveto(189.35161566,699.77098904)(188.37095688,700.06438412)(187.80890944,700.49258672)
\curveto(187.24685846,700.92078058)(186.96583386,701.57100752)(186.96583479,702.4432695)
\curveto(186.96583386,703.384242)(187.32296929,704.08204652)(188.03724216,704.53668516)
\curveto(188.75151103,704.99130696)(189.84633606,705.21862207)(191.32172052,705.21863117)
\curveto(191.90132848,705.21862207)(192.51021512,705.17897409)(193.14838228,705.0996871)
\curveto(193.78653521,705.02038215)(194.4803147,704.8961518)(195.22972284,704.72699567)
}
}
{
\newrgbcolor{curcolor}{0 0 0}
\pscustom[linestyle=none,fillstyle=solid,fillcolor=curcolor]
{
\newpath
\moveto(205.94379528,704.72699567)
\lineto(205.94379528,702.57014317)
\curveto(205.27049797,702.82388383)(204.62062857,703.01419415)(203.99418513,703.14107472)
\curveto(203.36772721,703.26794125)(202.7764046,703.33137802)(202.22021554,703.33138523)
\curveto(201.62303278,703.33137802)(201.17807716,703.26265485)(200.88534733,703.12521551)
\curveto(200.59846391,702.99304856)(200.45502427,702.78687904)(200.45502798,702.50670633)
\curveto(200.45502427,702.27938484)(200.56333584,702.10493371)(200.779963,701.98335242)
\curveto(201.00243678,701.86175941)(201.39762763,701.77189064)(201.96553674,701.71374586)
\lineto(202.51880448,701.64237941)
\curveto(204.12883551,701.45734997)(205.21195118,701.15338209)(205.76815473,700.73047487)
\curveto(206.32434024,700.30755843)(206.6024375,699.64411549)(206.60244736,698.74014407)
\curveto(206.6024375,697.79387623)(206.21602867,697.08285571)(205.4432197,696.60708038)
\curveto(204.67039334,696.13130409)(203.51702153,695.89341618)(201.98310079,695.89341595)
\curveto(201.33322616,695.89341618)(200.65993804,695.94099377)(199.96323442,696.03614884)
\curveto(199.27237906,696.12601769)(198.56103553,696.26346404)(197.8292017,696.44848829)
\lineto(197.8292017,698.60534079)
\curveto(198.4556513,698.33044561)(199.09673868,698.1242761)(199.75246576,697.98683161)
\curveto(200.41404151,697.84938341)(201.08440229,697.78066023)(201.7635501,697.78066189)
\curveto(202.37828641,697.78066023)(202.84080607,697.857313)(203.15111048,698.01062043)
\curveto(203.46140208,698.16392408)(203.61655108,698.39123919)(203.61655794,698.69256644)
\curveto(203.61655108,698.94631097)(203.50823951,699.13397809)(203.29162292,699.25556838)
\curveto(203.08084792,699.3824388)(202.65638368,699.48023716)(202.0182289,699.54896375)
\lineto(201.46496116,699.61240059)
\curveto(200.0656881,699.77098904)(199.08502932,700.06438412)(198.52298188,700.49258672)
\curveto(197.9609309,700.92078058)(197.67990629,701.57100752)(197.67990723,702.4432695)
\curveto(197.67990629,703.384242)(198.03704173,704.08204652)(198.7513146,704.53668516)
\curveto(199.46558347,704.99130696)(200.5604085,705.21862207)(202.03579296,705.21863117)
\curveto(202.61540092,705.21862207)(203.22428756,705.17897409)(203.86245472,705.0996871)
\curveto(204.50060765,705.02038215)(205.19438714,704.8961518)(205.94379528,704.72699567)
}
}
{
\newrgbcolor{curcolor}{0 0 0}
\pscustom[linestyle=none,fillstyle=solid,fillcolor=curcolor]
{
\newpath
\moveto(157.66898937,687.6624862)
\lineto(166.79351614,687.6624862)
\lineto(166.79351614,685.35497121)
\lineto(161.05007003,685.35497121)
\lineto(161.05007003,683.15054108)
\lineto(166.45101706,683.15054108)
\lineto(166.45101706,680.84302609)
\lineto(161.05007003,680.84302609)
\lineto(161.05007003,678.13110126)
\lineto(166.98672075,678.13110126)
\lineto(166.98672075,675.82358627)
\lineto(157.66898937,675.82358627)
\lineto(157.66898937,687.6624862)
}
}
{
\newrgbcolor{curcolor}{0 0 0}
\pscustom[linestyle=none,fillstyle=solid,fillcolor=curcolor]
{
\newpath
\moveto(176.29567027,688.1620513)
\lineto(176.29567027,686.29859418)
\lineto(174.55682879,686.29859418)
\curveto(174.11186692,686.2985837)(173.80156892,686.22457413)(173.62593386,686.07656525)
\curveto(173.45028816,685.93382225)(173.36246798,685.68271835)(173.36247303,685.32325279)
\lineto(173.36247303,684.70474362)
\lineto(176.0497735,684.70474362)
\lineto(176.0497735,682.6747648)
\lineto(173.36247303,682.6747648)
\lineto(173.36247303,675.82358627)
\lineto(170.21850711,675.82358627)
\lineto(170.21850711,682.6747648)
\lineto(168.65530618,682.6747648)
\lineto(168.65530618,684.70474362)
\lineto(170.21850711,684.70474362)
\lineto(170.21850711,685.32325279)
\curveto(170.21850521,686.29065411)(170.51709385,687.00431782)(171.11427394,687.46424608)
\curveto(171.71144842,687.92943745)(172.63648775,688.16203896)(173.88939469,688.1620513)
\lineto(176.29567027,688.1620513)
}
}
{
\newrgbcolor{curcolor}{0 0 0}
\pscustom[linestyle=none,fillstyle=solid,fillcolor=curcolor]
{
\newpath
\moveto(184.12923893,688.1620513)
\lineto(184.12923893,686.29859418)
\lineto(182.39039745,686.29859418)
\curveto(181.94543558,686.2985837)(181.63513758,686.22457413)(181.45950252,686.07656525)
\curveto(181.28385682,685.93382225)(181.19603664,685.68271835)(181.19604168,685.32325279)
\lineto(181.19604168,684.70474362)
\lineto(183.88334216,684.70474362)
\lineto(183.88334216,682.6747648)
\lineto(181.19604168,682.6747648)
\lineto(181.19604168,675.82358627)
\lineto(178.05207577,675.82358627)
\lineto(178.05207577,682.6747648)
\lineto(176.48887484,682.6747648)
\lineto(176.48887484,684.70474362)
\lineto(178.05207577,684.70474362)
\lineto(178.05207577,685.32325279)
\curveto(178.05207387,686.29065411)(178.35066251,687.00431782)(178.9478426,687.46424608)
\curveto(179.54501708,687.92943745)(180.47005641,688.16203896)(181.72296335,688.1620513)
\lineto(184.12923893,688.1620513)
}
}
{
\newrgbcolor{curcolor}{0 0 0}
\pscustom[linestyle=none,fillstyle=solid,fillcolor=curcolor]
{
\newpath
\moveto(190.17127394,682.88886413)
\curveto(189.47456092,682.88885706)(188.94178511,682.66154195)(188.5729449,682.20691812)
\curveto(188.2099502,681.75756791)(188.02845514,681.10734097)(188.02845919,680.25623534)
\curveto(188.02845514,679.40512085)(188.2099502,678.75225071)(188.5729449,678.29762296)
\curveto(188.94178511,677.84827667)(189.47456092,677.62360476)(190.17127394,677.62360656)
\curveto(190.85626523,677.62360476)(191.38025902,677.84827667)(191.7432569,678.29762296)
\curveto(192.10623925,678.75225071)(192.28773431,679.40512085)(192.28774262,680.25623534)
\curveto(192.28773431,681.10734097)(192.10623925,681.75756791)(191.7432569,682.20691812)
\curveto(191.38025902,682.66154195)(190.85626523,682.88885706)(190.17127394,682.88886413)
\moveto(190.17127394,684.91884295)
\curveto(191.86327006,684.91883385)(193.18350024,684.50649482)(194.13196843,683.6818246)
\curveto(195.086271,682.85713867)(195.56342736,681.71527673)(195.56343894,680.25623534)
\curveto(195.56342736,678.79718509)(195.086271,677.65532314)(194.13196843,676.83064608)
\curveto(193.18350024,676.005967)(191.86327006,675.59362797)(190.17127394,675.59362774)
\curveto(188.47341076,675.59362797)(187.14439857,676.005967)(186.18423337,676.83064608)
\curveto(185.22991845,677.65532314)(184.75276209,678.79718509)(184.75276286,680.25623534)
\curveto(184.75276209,681.71527673)(185.22991845,682.85713867)(186.18423337,683.6818246)
\curveto(187.14439857,684.50649482)(188.47341076,684.91883385)(190.17127394,684.91884295)
}
}
{
\newrgbcolor{curcolor}{0 0 0}
\pscustom[linestyle=none,fillstyle=solid,fillcolor=curcolor]
{
\newpath
\moveto(205.16219696,682.28621416)
\curveto(204.88701822,682.40250845)(204.6118483,682.48709082)(204.33668636,682.53996152)
\curveto(204.06736312,682.59810518)(203.79512054,682.62718037)(203.51995779,682.62718717)
\curveto(202.71200487,682.62718037)(202.08848153,682.39193566)(201.64938589,681.92145234)
\curveto(201.21613432,681.45624323)(200.99951119,680.7875139)(200.99951584,679.91526234)
\lineto(200.99951584,675.82358627)
\lineto(197.85554993,675.82358627)
\lineto(197.85554993,684.70474362)
\lineto(200.99951584,684.70474362)
\lineto(200.99951584,683.24569634)
\curveto(201.40348406,683.82719269)(201.86600372,684.25010452)(202.38707622,684.5144331)
\curveto(202.91399131,684.78403071)(203.54336933,684.91883385)(204.27521217,684.91884295)
\curveto(204.38058847,684.91883385)(204.49475471,684.91354746)(204.61771125,684.90298374)
\curveto(204.74065124,684.89768826)(204.91921896,684.88182907)(205.15341494,684.85540611)
\lineto(205.16219696,682.28621416)
}
}
{
\newrgbcolor{curcolor}{0 0 0}
\pscustom[linestyle=none,fillstyle=solid,fillcolor=curcolor]
{
\newpath
\moveto(210.1767324,687.22635794)
\lineto(210.1767324,684.70474362)
\lineto(213.41730062,684.70474362)
\lineto(213.41730062,682.6747648)
\lineto(210.1767324,682.6747648)
\lineto(210.1767324,678.90820253)
\curveto(210.17672746,678.49586041)(210.26747498,678.21568132)(210.44897526,678.06766442)
\curveto(210.6304651,677.92492944)(210.99052787,677.85356307)(211.52916466,677.8535651)
\lineto(213.14505776,677.8535651)
\lineto(213.14505776,675.82358627)
\lineto(210.44897526,675.82358627)
\curveto(209.20777804,675.82358627)(208.32664881,676.05618778)(207.80558492,676.52139149)
\curveto(207.29037057,676.99188021)(207.03276469,677.78748309)(207.03276649,678.90820253)
\lineto(207.03276649,682.6747648)
\lineto(205.46956556,682.6747648)
\lineto(205.46956556,684.70474362)
\lineto(207.03276649,684.70474362)
\lineto(207.03276649,687.22635794)
\lineto(210.1767324,687.22635794)
}
}
{
\newrgbcolor{curcolor}{0 0 0}
\pscustom[linestyle=none,fillstyle=solid,fillcolor=curcolor]
{
\newpath
\moveto(164.84499199,975.95509837)
\lineto(169.14818556,975.95509837)
\lineto(172.13407497,969.61934416)
\lineto(175.13752844,975.95509837)
\lineto(179.43193998,975.95509837)
\lineto(179.43193998,964.11619844)
\lineto(176.2352819,964.11619844)
\lineto(176.2352819,972.77532686)
\lineto(173.21426438,966.39199501)
\lineto(171.07144962,966.39199501)
\lineto(168.0504321,972.77532686)
\lineto(168.0504321,964.11619844)
\lineto(164.84499199,964.11619844)
\lineto(164.84499199,975.95509837)
}
}
{
\newrgbcolor{curcolor}{0 0 0}
\pscustom[linestyle=none,fillstyle=solid,fillcolor=curcolor]
{
\newpath
\moveto(187.30063721,971.1814763)
\curveto(186.60392419,971.18146923)(186.07114838,970.95415412)(185.70230817,970.49953029)
\curveto(185.33931347,970.05018008)(185.15781841,969.39995314)(185.15782246,968.54884751)
\curveto(185.15781841,967.69773302)(185.33931347,967.04486288)(185.70230817,966.59023513)
\curveto(186.07114838,966.14088884)(186.60392419,965.91621693)(187.30063721,965.91621873)
\curveto(187.9856285,965.91621693)(188.50962229,966.14088884)(188.87262017,966.59023513)
\curveto(189.23560252,967.04486288)(189.41709758,967.69773302)(189.41710589,968.54884751)
\curveto(189.41709758,969.39995314)(189.23560252,970.05018008)(188.87262017,970.49953029)
\curveto(188.50962229,970.95415412)(187.9856285,971.18146923)(187.30063721,971.1814763)
\moveto(187.30063721,973.21145512)
\curveto(188.99263333,973.21144602)(190.31286351,972.79910699)(191.2613317,971.97443677)
\curveto(192.21563427,971.14975084)(192.69279063,970.0078889)(192.69280222,968.54884751)
\curveto(192.69279063,967.08979726)(192.21563427,965.94793531)(191.2613317,965.12325825)
\curveto(190.31286351,964.29857917)(188.99263333,963.88624013)(187.30063721,963.8862399)
\curveto(185.60277403,963.88624013)(184.27376184,964.29857917)(183.31359664,965.12325825)
\curveto(182.35928172,965.94793531)(181.88212536,967.08979726)(181.88212613,968.54884751)
\curveto(181.88212536,970.0078889)(182.35928172,971.14975084)(183.31359664,971.97443677)
\curveto(184.27376184,972.79910699)(185.60277403,973.21144602)(187.30063721,973.21145512)
}
}
{
\newrgbcolor{curcolor}{0 0 0}
\pscustom[linestyle=none,fillstyle=solid,fillcolor=curcolor]
{
\newpath
\moveto(202.29155823,970.57882633)
\curveto(202.01637949,970.69512062)(201.74120956,970.77970299)(201.46604763,970.83257369)
\curveto(201.19672439,970.89071735)(200.9244818,970.91979254)(200.64931905,970.91979934)
\curveto(199.84136614,970.91979254)(199.21784279,970.68454783)(198.77874715,970.21406451)
\curveto(198.34549558,969.7488554)(198.12887245,969.08012607)(198.1288771,968.20787451)
\lineto(198.1288771,964.11619844)
\lineto(194.98491119,964.11619844)
\lineto(194.98491119,972.99735579)
\lineto(198.1288771,972.99735579)
\lineto(198.1288771,971.53830851)
\curveto(198.53284532,972.11980486)(198.99536498,972.54271669)(199.51643748,972.80704527)
\curveto(200.04335257,973.07664288)(200.67273059,973.21144602)(201.40457343,973.21145512)
\curveto(201.50994973,973.21144602)(201.62411597,973.20615962)(201.74707251,973.19559591)
\curveto(201.8700125,973.19030043)(202.04858022,973.17444124)(202.2827762,973.14801828)
\lineto(202.29155823,970.57882633)
}
}
{
\newrgbcolor{curcolor}{0 0 0}
\pscustom[linestyle=none,fillstyle=solid,fillcolor=curcolor]
{
\newpath
\moveto(213.69063181,968.58056593)
\lineto(213.69063181,967.77174624)
\lineto(206.34007463,967.77174624)
\curveto(206.41618148,967.10565645)(206.68256939,966.60609185)(207.13923915,966.27305094)
\curveto(207.59589936,965.94000572)(208.2340594,965.77348418)(209.05372119,965.77348584)
\curveto(209.71529325,965.77348418)(210.39150871,965.86070975)(211.08236958,966.0351628)
\curveto(211.7790677,966.21489841)(212.49333857,966.4845047)(213.22518434,966.84398249)
\lineto(213.22518434,964.65541157)
\curveto(212.48162921,964.40166393)(211.73808494,964.21135361)(210.99454931,964.08448002)
\curveto(210.25099641,963.95232011)(209.50745214,963.88624013)(208.76391427,963.8862399)
\curveto(206.98408537,963.88624013)(205.59945373,964.29329277)(204.61001518,965.10739904)
\curveto(203.62642681,965.92678972)(203.13463375,967.07393807)(203.13463453,968.54884751)
\curveto(203.13463375,969.9973161)(203.61764479,971.13653485)(204.58366909,971.96650717)
\curveto(205.55554363,972.79646379)(206.8904105,973.21144602)(208.58827372,973.21145512)
\curveto(210.13390282,973.21144602)(211.36924015,972.79117739)(212.2942894,971.95064796)
\curveto(213.22517348,971.11010286)(213.69062048,969.98674331)(213.69063181,968.58056593)
\moveto(210.45884562,969.5241889)
\curveto(210.45883752,970.06339608)(210.28319714,970.4968807)(209.93192396,970.82464408)
\curveto(209.58649031,971.15768044)(209.13275267,971.32420197)(208.57070966,971.32420918)
\curveto(207.96181681,971.32420197)(207.46709641,971.16825324)(207.08654699,970.8563625)
\curveto(206.70598811,970.54974468)(206.4688736,970.10568726)(206.37520274,969.5241889)
\lineto(210.45884562,969.5241889)
}
}
{
\newrgbcolor{curcolor}{0 0 0}
\pscustom[linestyle=none,fillstyle=solid,fillcolor=curcolor]
{
\newpath
\moveto(161.6132058,955.65531015)
\lineto(170.73773257,955.65531015)
\lineto(170.73773257,953.34779516)
\lineto(164.99428646,953.34779516)
\lineto(164.99428646,951.14336503)
\lineto(170.39523349,951.14336503)
\lineto(170.39523349,948.83585005)
\lineto(164.99428646,948.83585005)
\lineto(164.99428646,946.12392521)
\lineto(170.93093718,946.12392521)
\lineto(170.93093718,943.81641022)
\lineto(161.6132058,943.81641022)
\lineto(161.6132058,955.65531015)
}
}
{
\newrgbcolor{curcolor}{0 0 0}
\pscustom[linestyle=none,fillstyle=solid,fillcolor=curcolor]
{
\newpath
\moveto(180.2398867,956.15487525)
\lineto(180.2398867,954.29141813)
\lineto(178.50104522,954.29141813)
\curveto(178.05608335,954.29140765)(177.74578535,954.21739808)(177.57015028,954.0693892)
\curveto(177.39450459,953.9266462)(177.3066844,953.6755423)(177.30668945,953.31607674)
\lineto(177.30668945,952.69756757)
\lineto(179.99398993,952.69756757)
\lineto(179.99398993,950.66758875)
\lineto(177.30668945,950.66758875)
\lineto(177.30668945,943.81641022)
\lineto(174.16272354,943.81641022)
\lineto(174.16272354,950.66758875)
\lineto(172.59952261,950.66758875)
\lineto(172.59952261,952.69756757)
\lineto(174.16272354,952.69756757)
\lineto(174.16272354,953.31607674)
\curveto(174.16272163,954.28347806)(174.46131028,954.99714177)(175.05849036,955.45707003)
\curveto(175.65566485,955.92226141)(176.58070417,956.15486291)(177.83361111,956.15487525)
\lineto(180.2398867,956.15487525)
}
}
{
\newrgbcolor{curcolor}{0 0 0}
\pscustom[linestyle=none,fillstyle=solid,fillcolor=curcolor]
{
\newpath
\moveto(188.07345536,956.15487525)
\lineto(188.07345536,954.29141813)
\lineto(186.33461388,954.29141813)
\curveto(185.88965201,954.29140765)(185.57935401,954.21739808)(185.40371894,954.0693892)
\curveto(185.22807325,953.9266462)(185.14025306,953.6755423)(185.14025811,953.31607674)
\lineto(185.14025811,952.69756757)
\lineto(187.82755859,952.69756757)
\lineto(187.82755859,950.66758875)
\lineto(185.14025811,950.66758875)
\lineto(185.14025811,943.81641022)
\lineto(181.9962922,943.81641022)
\lineto(181.9962922,950.66758875)
\lineto(180.43309127,950.66758875)
\lineto(180.43309127,952.69756757)
\lineto(181.9962922,952.69756757)
\lineto(181.9962922,953.31607674)
\curveto(181.99629029,954.28347806)(182.29487894,954.99714177)(182.89205902,955.45707003)
\curveto(183.48923351,955.92226141)(184.41427283,956.15486291)(185.66717977,956.15487525)
\lineto(188.07345536,956.15487525)
}
}
{
\newrgbcolor{curcolor}{0 0 0}
\pscustom[linestyle=none,fillstyle=solid,fillcolor=curcolor]
{
\newpath
\moveto(194.11549037,950.88168808)
\curveto(193.41877735,950.88168101)(192.88600153,950.6543659)(192.51716133,950.19974207)
\curveto(192.15416662,949.75039186)(191.97267157,949.10016492)(191.97267562,948.24905929)
\curveto(191.97267157,947.3979448)(192.15416662,946.74507466)(192.51716133,946.29044691)
\curveto(192.88600153,945.84110062)(193.41877735,945.61642871)(194.11549037,945.61643051)
\curveto(194.80048165,945.61642871)(195.32447545,945.84110062)(195.68747333,946.29044691)
\curveto(196.05045568,946.74507466)(196.23195074,947.3979448)(196.23195904,948.24905929)
\curveto(196.23195074,949.10016492)(196.05045568,949.75039186)(195.68747333,950.19974207)
\curveto(195.32447545,950.6543659)(194.80048165,950.88168101)(194.11549037,950.88168808)
\moveto(194.11549037,952.9116669)
\curveto(195.80748649,952.9116578)(197.12771667,952.49931877)(198.07618486,951.67464855)
\curveto(199.03048743,950.84996262)(199.50764379,949.70810068)(199.50765537,948.24905929)
\curveto(199.50764379,946.79000904)(199.03048743,945.6481471)(198.07618486,944.82347003)
\curveto(197.12771667,943.99879095)(195.80748649,943.58645192)(194.11549037,943.58645169)
\curveto(192.41762719,943.58645192)(191.088615,943.99879095)(190.1284498,944.82347003)
\curveto(189.17413488,945.6481471)(188.69697851,946.79000904)(188.69697929,948.24905929)
\curveto(188.69697851,949.70810068)(189.17413488,950.84996262)(190.1284498,951.67464855)
\curveto(191.088615,952.49931877)(192.41762719,952.9116578)(194.11549037,952.9116669)
}
}
{
\newrgbcolor{curcolor}{0 0 0}
\pscustom[linestyle=none,fillstyle=solid,fillcolor=curcolor]
{
\newpath
\moveto(209.10641339,950.27903811)
\curveto(208.83123465,950.39533241)(208.55606472,950.47991477)(208.28090279,950.53278547)
\curveto(208.01157955,950.59092913)(207.73933697,950.62000432)(207.46417421,950.62001112)
\curveto(206.6562213,950.62000432)(206.03269796,950.38475961)(205.59360232,949.91427629)
\curveto(205.16035075,949.44906718)(204.94372761,948.78033785)(204.94373227,947.90808629)
\lineto(204.94373227,943.81641022)
\lineto(201.79976636,943.81641022)
\lineto(201.79976636,952.69756757)
\lineto(204.94373227,952.69756757)
\lineto(204.94373227,951.23852029)
\curveto(205.34770048,951.82001664)(205.81022015,952.24292847)(206.33129264,952.50725706)
\curveto(206.85820774,952.77685466)(207.48758576,952.9116578)(208.2194286,952.9116669)
\curveto(208.32480489,952.9116578)(208.43897114,952.90637141)(208.56192768,952.89580769)
\curveto(208.68486767,952.89051221)(208.86343539,952.87465302)(209.09763136,952.84823006)
\lineto(209.10641339,950.27903811)
}
}
{
\newrgbcolor{curcolor}{0 0 0}
\pscustom[linestyle=none,fillstyle=solid,fillcolor=curcolor]
{
\newpath
\moveto(214.12094883,955.21918189)
\lineto(214.12094883,952.69756757)
\lineto(217.36151704,952.69756757)
\lineto(217.36151704,950.66758875)
\lineto(214.12094883,950.66758875)
\lineto(214.12094883,946.90102648)
\curveto(214.12094388,946.48868436)(214.21169141,946.20850527)(214.39319169,946.06048837)
\curveto(214.57468153,945.91775339)(214.9347443,945.84638702)(215.47338109,945.84638905)
\lineto(217.08927419,945.84638905)
\lineto(217.08927419,943.81641022)
\lineto(214.39319169,943.81641022)
\curveto(213.15199446,943.81641022)(212.27086523,944.04901173)(211.74980135,944.51421544)
\curveto(211.234587,944.98470416)(210.97698111,945.78030704)(210.97698291,946.90102648)
\lineto(210.97698291,950.66758875)
\lineto(209.41378199,950.66758875)
\lineto(209.41378199,952.69756757)
\lineto(210.97698291,952.69756757)
\lineto(210.97698291,955.21918189)
\lineto(214.12094883,955.21918189)
}
}
{
\newrgbcolor{curcolor}{0 0 0}
\pscustom[linestyle=none,fillstyle=solid,fillcolor=curcolor]
{
\newpath
\moveto(501.099832,751.85875153)
\curveto(501.099832,748.12166375)(497.74465548,745.09215553)(493.6058355,745.09215553)
\curveto(489.46701552,745.09215553)(486.111839,748.12166375)(486.111839,751.85875153)
\curveto(486.111839,755.59583931)(489.46701552,758.62534753)(493.6058355,758.62534753)
\curveto(497.74465548,758.62534753)(501.099832,755.59583931)(501.099832,751.85875153)
\closepath
}
}
{
\newrgbcolor{curcolor}{0 0 0}
\pscustom[linewidth=1.42420289,linecolor=curcolor]
{
\newpath
\moveto(501.099832,751.85875153)
\curveto(501.099832,748.12166375)(497.74465548,745.09215553)(493.6058355,745.09215553)
\curveto(489.46701552,745.09215553)(486.111839,748.12166375)(486.111839,751.85875153)
\curveto(486.111839,755.59583931)(489.46701552,758.62534753)(493.6058355,758.62534753)
\curveto(497.74465548,758.62534753)(501.099832,755.59583931)(501.099832,751.85875153)
\closepath
}
}
{
\newrgbcolor{curcolor}{0 0 0}
\pscustom[linestyle=none,fillstyle=solid,fillcolor=curcolor]
{
\newpath
\moveto(521.88737974,781.86066168)
\lineto(524.54833413,781.86066168)
\lineto(526.07640695,773.88347928)
\lineto(531.15241896,781.86066168)
\lineto(533.88362957,781.86066168)
\lineto(531.32805951,770.02176176)
\lineto(529.60678208,770.02176176)
\lineto(531.83741712,780.4016144)
\lineto(526.69114889,772.29755833)
\lineto(524.96108943,772.29755833)
\lineto(523.29250417,780.43333282)
\lineto(521.06186914,770.02176176)
\lineto(519.33180969,770.02176176)
\lineto(521.88737974,781.86066168)
}
}
{
\newrgbcolor{curcolor}{0 0 0}
\pscustom[linestyle=none,fillstyle=solid,fillcolor=curcolor]
{
\newpath
\moveto(538.83669295,769.19708286)
\curveto(538.08728954,768.06050814)(537.47547556,767.35213082)(537.00124916,767.07194878)
\curveto(536.53287219,766.78648624)(535.93276757,766.6437535)(535.20093348,766.64375012)
\lineto(533.92753947,766.64375012)
\lineto(534.19978233,767.86490926)
\lineto(535.13067726,767.86490926)
\curveto(535.58148681,767.86491142)(535.9649683,767.97592577)(536.28112289,768.19795266)
\curveto(536.59727367,768.41998319)(536.95148176,768.85346782)(537.34374824,769.49840784)
\lineto(537.84432382,770.34687555)
\lineto(535.66638095,778.9029191)
\lineto(537.37887635,778.9029191)
\lineto(539.0123335,772.09931821)
\lineto(543.52629573,778.9029191)
\lineto(545.22122708,778.9029191)
\lineto(538.83669295,769.19708286)
}
}
{
\newrgbcolor{curcolor}{0 0 0}
\pscustom[linestyle=none,fillstyle=solid,fillcolor=curcolor]
{
\newpath
\moveto(553.03723181,777.55488629)
\curveto(552.87329277,777.63417473)(552.68594303,777.6949683)(552.47518204,777.7372672)
\curveto(552.26440613,777.77955067)(552.03900097,777.80069626)(551.7989659,777.80070404)
\curveto(550.93832127,777.80069626)(550.18599498,777.50465798)(549.54198479,776.9125883)
\curveto(548.89796554,776.32579125)(548.47057396,775.53811796)(548.25980874,774.54956608)
\lineto(547.26743962,770.02176176)
\lineto(545.65154652,770.02176176)
\lineto(547.57481058,778.9029191)
\lineto(549.19070368,778.9029191)
\lineto(548.88333271,777.52316787)
\curveto(549.31072043,778.03594097)(549.82007753,778.42977761)(550.41140553,778.70467898)
\curveto(551.00857742,778.97956299)(551.64381012,779.11700934)(552.31710554,779.11701843)
\curveto(552.49273862,779.11700934)(552.66545166,779.10643654)(552.83524517,779.08530001)
\curveto(553.00502305,779.06943175)(553.17480875,779.04035657)(553.34460278,778.99807436)
\lineto(553.03723181,777.55488629)
}
}
{
\newrgbcolor{curcolor}{0 0 0}
\pscustom[linestyle=none,fillstyle=solid,fillcolor=curcolor]
{
\newpath
\moveto(555.72453277,782.36022678)
\lineto(557.34042587,782.36022678)
\lineto(556.94523462,780.51262887)
\lineto(555.32934153,780.51262887)
\lineto(555.72453277,782.36022678)
\moveto(554.98684245,778.9029191)
\lineto(556.60273554,778.9029191)
\lineto(554.67947148,770.02176176)
\lineto(553.06357838,770.02176176)
\lineto(554.98684245,778.9029191)
}
}
{
\newrgbcolor{curcolor}{0 0 0}
\pscustom[linestyle=none,fillstyle=solid,fillcolor=curcolor]
{
\newpath
\moveto(567.07969262,775.08877921)
\lineto(565.98193915,770.02176176)
\lineto(564.36604606,770.02176176)
\lineto(564.664635,771.36979457)
\curveto(564.19039874,770.84115343)(563.6488409,770.44467359)(563.03995988,770.18035385)
\curveto(562.43692229,769.9213202)(561.76363418,769.79180345)(561.02009351,769.79180322)
\curveto(560.18287077,769.79180345)(559.49494596,770.02176176)(558.956317,770.48167883)
\curveto(558.42353965,770.94688139)(558.15715175,771.54160115)(558.15715248,772.26583991)
\curveto(558.15715175,773.30197165)(558.61381673,774.12136332)(559.5271488,774.72401739)
\curveto(560.44633134,775.32666204)(561.71094206,775.62798672)(563.32098476,775.62799233)
\lineto(565.56918385,775.62799233)
\lineto(565.65700413,776.01654297)
\curveto(565.66870525,776.05882815)(565.67748727,776.10376254)(565.68335021,776.15134625)
\curveto(565.68919663,776.2042041)(565.69212397,776.28350006)(565.69213224,776.38923439)
\curveto(565.69212397,776.85971743)(565.47842817,777.22447889)(565.05104422,777.48351985)
\curveto(564.62949968,777.74783228)(564.03524973,777.87999223)(563.2682926,777.88000009)
\curveto(562.74136562,777.87999223)(562.19980778,777.81919865)(561.64361748,777.69761918)
\curveto(561.0932734,777.57602435)(560.52536951,777.39364362)(559.9399041,777.15047645)
\lineto(560.22092899,778.49850926)
\curveto(560.82981283,778.7046703)(561.42406278,778.85797584)(562.00368061,778.95842634)
\curveto(562.58914395,779.06414536)(563.1541205,779.11700934)(563.69861196,779.11701843)
\curveto(564.85783217,779.11700934)(565.7389614,778.88969423)(566.34200229,778.43507242)
\curveto(566.95088001,777.98043379)(567.25532333,777.31963405)(567.25533317,776.45267123)
\curveto(567.25532333,776.27821367)(567.24068664,776.07204415)(567.21142303,775.83416206)
\curveto(567.18213984,775.60155474)(567.13822975,775.35309403)(567.07969262,775.08877921)
\moveto(565.33206911,774.48612924)
\lineto(563.71617601,774.48612924)
\curveto(562.3930122,774.48612478)(561.41235342,774.32488964)(560.77419674,774.00242335)
\curveto(560.14188802,773.6852355)(559.82573534,773.1909573)(559.82573774,772.51958726)
\curveto(559.82573534,772.05438175)(559.98673902,771.68962029)(560.30874927,771.4253018)
\curveto(560.63660842,771.1609805)(561.08741872,771.02882056)(561.66118153,771.02882156)
\curveto(562.53937918,771.02882056)(563.30634217,771.31164284)(563.96207279,771.87728927)
\curveto(564.61779032,772.44821839)(565.05103659,773.20945969)(565.26181288,774.16101545)
\lineto(565.33206911,774.48612924)
}
}
{
\newrgbcolor{curcolor}{0 0 0}
\pscustom[linestyle=none,fillstyle=solid,fillcolor=curcolor]
{
\newpath
\moveto(575.9671044,771.35393536)
\curveto(575.53385062,770.84115343)(575.02156618,770.45260319)(574.43024956,770.18828346)
\curveto(573.83892097,769.9239634)(573.18319689,769.79180345)(572.46307535,769.79180322)
\curveto(571.47948522,769.79180345)(570.70081288,770.09312813)(570.12705599,770.69577816)
\curveto(569.55915042,771.30371325)(569.27519848,772.13367772)(569.2751993,773.18567406)
\curveto(569.27519848,774.06321295)(569.44791152,774.89846382)(569.79333894,775.69142917)
\curveto(570.13876367,776.48966958)(570.63933875,777.2033333)(571.29506567,777.83242246)
\curveto(571.72830909,778.25004008)(572.21717481,778.56722395)(572.7616643,778.78397503)
\curveto(573.30614516,779.00599498)(573.87697638,779.11700934)(574.47415969,779.11701843)
\curveto(575.10645903,779.11700934)(575.66265356,778.97956299)(576.14274496,778.70467898)
\curveto(576.62867564,778.42977761)(577.00337512,778.03594097)(577.2668445,777.52316787)
\lineto(578.32068782,782.36022678)
\lineto(579.94536294,782.36022678)
\lineto(577.28440855,770.02176176)
\lineto(575.65973343,770.02176176)
\lineto(575.9671044,771.35393536)
\moveto(570.95256659,773.44735102)
\curveto(570.95256409,772.6808199)(571.14284116,772.08345694)(571.52339839,771.65526034)
\curveto(571.90980415,771.22706048)(572.44257996,771.01296136)(573.12172743,771.01296235)
\curveto(573.62522518,771.01296136)(574.09067218,771.12133252)(574.51806983,771.33807615)
\curveto(574.95131003,771.56010354)(575.33186418,771.88521701)(575.65973343,772.31341754)
\curveto(576.0051523,772.75747267)(576.27739489,773.27025326)(576.47646201,773.85176086)
\curveto(576.67551308,774.4332608)(576.77504262,775.00683497)(576.77505095,775.5724851)
\curveto(576.77504262,776.30728885)(576.58183821,776.88086303)(576.19543712,777.29320933)
\curveto(575.81487522,777.7055411)(575.28795409,777.91171062)(574.61467214,777.91171851)
\curveto(574.10530888,777.91171062)(573.63107985,777.80333946)(573.19198365,777.58660471)
\curveto(572.75287796,777.36985483)(572.37817849,777.05795736)(572.06788411,776.65091135)
\curveto(571.72830909,776.21213369)(571.45606651,775.7019963)(571.25115553,775.12049763)
\curveto(571.05209364,774.53898876)(570.95256409,773.98127378)(570.95256659,773.44735102)
}
}
{
\newrgbcolor{curcolor}{0 0 0}
\pscustom[linestyle=none,fillstyle=solid,fillcolor=curcolor]
{
\newpath
\moveto(500.113802,886.30034153)
\curveto(500.113802,882.56325375)(496.75862548,879.53374553)(492.6198055,879.53374553)
\curveto(488.48098552,879.53374553)(485.125809,882.56325375)(485.125809,886.30034153)
\curveto(485.125809,890.03742931)(488.48098552,893.06693753)(492.6198055,893.06693753)
\curveto(496.75862548,893.06693753)(500.113802,890.03742931)(500.113802,886.30034153)
\closepath
}
}
{
\newrgbcolor{curcolor}{0 0 0}
\pscustom[linewidth=1.42420289,linecolor=curcolor]
{
\newpath
\moveto(500.113802,886.30034153)
\curveto(500.113802,882.56325375)(496.75862548,879.53374553)(492.6198055,879.53374553)
\curveto(488.48098552,879.53374553)(485.125809,882.56325375)(485.125809,886.30034153)
\curveto(485.125809,890.03742931)(488.48098552,893.06693753)(492.6198055,893.06693753)
\curveto(496.75862548,893.06693753)(500.113802,890.03742931)(500.113802,886.30034153)
\closepath
}
}
{
\newrgbcolor{curcolor}{0 0 0}
\pscustom[linestyle=none,fillstyle=solid,fillcolor=curcolor]
{
\newpath
\moveto(515.97109525,916.30223261)
\lineto(517.75384687,916.30223261)
\lineto(516.70878557,911.44931449)
\lineto(523.14601187,911.44931449)
\lineto(524.19107316,916.30223261)
\lineto(525.97382478,916.30223261)
\lineto(523.41825473,904.46333269)
\lineto(521.63550311,904.46333269)
\lineto(522.85620496,910.10128168)
\lineto(516.41897866,910.10128168)
\lineto(515.19827681,904.46333269)
\lineto(513.41552519,904.46333269)
\lineto(515.97109525,916.30223261)
}
}
{
\newrgbcolor{curcolor}{0 0 0}
\pscustom[linestyle=none,fillstyle=solid,fillcolor=curcolor]
{
\newpath
\moveto(535.10713337,909.69687184)
\curveto(535.12468876,909.78144897)(535.13639812,909.86867453)(535.14226148,909.95854879)
\curveto(535.15396215,910.04841206)(535.15981683,910.13828083)(535.15982554,910.22815536)
\curveto(535.15981683,910.87309013)(534.94904838,911.38322753)(534.52751954,911.75856908)
\curveto(534.11182924,912.13389603)(533.54392535,912.32156316)(532.82380617,912.32157102)
\curveto(532.02170874,912.32156316)(531.31329255,912.09160485)(530.69855547,911.6316954)
\curveto(530.0838099,911.17705801)(529.6183629,910.52947427)(529.30221307,909.68894223)
\lineto(535.10713337,909.69687184)
\moveto(536.5122578,908.55500875)
\lineto(528.9948421,908.55500875)
\curveto(528.95971148,908.35412154)(528.93629277,908.1955296)(528.92458588,908.07923246)
\curveto(528.91287405,907.96292809)(528.90701937,907.86248653)(528.90702182,907.77790748)
\curveto(528.90701937,907.04309486)(529.15584324,906.47480709)(529.65349417,906.07304245)
\curveto(530.1569934,905.67127461)(530.86540959,905.47039148)(531.77874488,905.47039249)
\curveto(532.48130107,905.47039148)(533.14580716,905.54175786)(533.77226516,905.68449182)
\curveto(534.39870853,905.82722334)(534.98124912,906.03603606)(535.51988867,906.3109306)
\lineto(535.2125177,904.86774253)
\curveto(534.6328957,904.65628621)(534.03279107,904.49769427)(533.41220203,904.39196624)
\curveto(532.79745375,904.28623836)(532.17100306,904.23337438)(531.5328481,904.23337415)
\curveto(530.16870275,904.23337438)(529.11778782,904.52676946)(528.38010016,905.11356028)
\curveto(527.64826333,905.70563619)(527.28234587,906.54353026)(527.2823467,907.62724499)
\curveto(527.28234587,908.55236146)(527.46384093,909.41140112)(527.82683242,910.20436654)
\curveto(528.19567584,911.00260688)(528.73430633,911.7136274)(529.44272551,912.33743023)
\curveto(529.89938751,912.7286158)(530.44094534,913.02994048)(531.06740063,913.24140517)
\curveto(531.69970138,913.45285231)(532.37006216,913.55858027)(533.07848497,913.55858936)
\curveto(534.19086741,913.55858027)(535.07492398,913.25725559)(535.73065734,912.65461442)
\curveto(536.39222682,912.05195687)(536.7230162,911.24313799)(536.72302646,910.22815536)
\curveto(536.7230162,909.98497529)(536.70545216,909.72329859)(536.6703343,909.44312448)
\curveto(536.63519601,909.16822681)(536.5825039,908.87218853)(536.5122578,908.55500875)
}
}
{
\newrgbcolor{curcolor}{0 0 0}
\pscustom[linestyle=none,fillstyle=solid,fillcolor=curcolor]
{
\newpath
\moveto(540.81545192,916.80179771)
\lineto(542.43134502,916.80179771)
\lineto(539.77039063,904.46333269)
\lineto(538.15449753,904.46333269)
\lineto(540.81545192,916.80179771)
}
}
{
\newrgbcolor{curcolor}{0 0 0}
\pscustom[linestyle=none,fillstyle=solid,fillcolor=curcolor]
{
\newpath
\moveto(551.43292345,909.93475998)
\curveto(551.43291452,910.711855)(551.24556479,911.30657476)(550.87087367,911.71892105)
\curveto(550.49616584,912.13125283)(549.95753535,912.33742235)(549.25498058,912.33743023)
\curveto(548.77489014,912.33742235)(548.31529781,912.226408)(547.87620223,912.00438683)
\curveto(547.4429506,911.78235057)(547.05654177,911.4598803)(546.71697458,911.03697504)
\curveto(546.38325365,910.61934304)(546.11686575,910.11449204)(545.91781006,909.52242053)
\curveto(545.71874756,908.93033891)(545.61921801,908.34354874)(545.61922112,907.76204827)
\curveto(545.61921801,907.02194927)(545.80656774,906.45101829)(546.18127089,906.04925364)
\curveto(546.55596669,905.65277221)(547.08874251,905.45453229)(547.77959993,905.45453328)
\curveto(548.28895176,905.45453229)(548.76025344,905.56290345)(549.19350639,905.77964708)
\curveto(549.63260065,905.99638808)(550.00730012,906.31092875)(550.31760593,906.72327005)
\curveto(550.65131484,907.15675241)(550.92063009,907.66688981)(551.12555248,908.25368377)
\curveto(551.33045764,908.84047014)(551.43291452,909.40082832)(551.43292345,909.93475998)
\moveto(546.42716766,911.99645722)
\curveto(546.86041001,912.50923028)(547.36976711,912.89778053)(547.95524048,913.16210912)
\curveto(548.54655764,913.42642032)(549.20228172,913.55858027)(549.92241468,913.55858936)
\curveto(550.92941211,913.55858027)(551.71101179,913.25989878)(552.26721608,912.66254402)
\curveto(552.82340085,912.07045926)(553.10149812,911.23785159)(553.10150871,910.16471852)
\curveto(553.10149812,909.28717077)(552.92878508,908.4492767)(552.58336907,907.6510338)
\curveto(552.23793292,906.85807093)(551.74028519,906.14705042)(551.09042437,905.51797012)
\curveto(550.65716952,905.10034363)(550.1683038,904.78051656)(549.62382574,904.55848794)
\curveto(549.07933346,904.34174553)(548.50557489,904.23337438)(547.90254832,904.23337415)
\curveto(547.20583609,904.23337438)(546.62036816,904.36289113)(546.14614278,904.62192478)
\curveto(545.67191012,904.88624452)(545.31770203,905.27743796)(545.08351743,905.79550629)
\lineto(544.07358424,901.08532105)
\lineto(542.45769115,901.08532105)
\lineto(545.10108148,913.34449003)
\lineto(546.71697458,913.34449003)
\lineto(546.42716766,911.99645722)
}
}
{
\newrgbcolor{curcolor}{0 0 0}
\pscustom[linestyle=none,fillstyle=solid,fillcolor=curcolor]
{
}
}
{
\newrgbcolor{curcolor}{0 0 0}
\pscustom[linestyle=none,fillstyle=solid,fillcolor=curcolor]
{
\newpath
\moveto(562.69148401,916.30223261)
\lineto(566.50288403,916.30223261)
\curveto(568.52859621,916.30222077)(570.0566675,915.91367053)(571.08710248,915.13658071)
\curveto(572.12336928,914.36475594)(572.64150839,913.2149644)(572.64152138,911.68720263)
\curveto(572.64150839,910.67749341)(572.44537664,909.7047962)(572.05312553,908.76910808)
\curveto(571.66084962,907.83869774)(571.11929178,907.05366765)(570.4284504,906.41401546)
\curveto(569.7317328,905.75850017)(568.84767623,905.26950836)(567.77627804,904.94703858)
\curveto(566.70486361,904.62456782)(565.43732555,904.46333269)(563.97366005,904.46333269)
\lineto(560.13591395,904.46333269)
\lineto(562.69148401,916.30223261)
\moveto(564.18442872,914.98591822)
\lineto(562.20847249,905.77964708)
\lineto(564.5269278,905.77964708)
\curveto(566.5116592,905.77964576)(568.04851251,906.29506956)(569.13749233,907.32592001)
\curveto(570.22645319,908.35676474)(570.77093837,909.81052416)(570.77094948,911.68720263)
\curveto(570.77093837,912.82377096)(570.41965761,913.65637863)(569.71710616,914.18502814)
\curveto(569.01453458,914.7189446)(567.91385488,914.9859077)(566.41506375,914.98591822)
\lineto(564.18442872,914.98591822)
}
}
{
\newrgbcolor{curcolor}{0 0 0}
\pscustom[linestyle=none,fillstyle=solid,fillcolor=curcolor]
{
\newpath
\moveto(582.14367423,909.69687184)
\curveto(582.16122962,909.78144897)(582.17293897,909.86867453)(582.17880234,909.95854879)
\curveto(582.19050301,910.04841206)(582.19635769,910.13828083)(582.19636639,910.22815536)
\curveto(582.19635769,910.87309013)(581.98558924,911.38322753)(581.5640604,911.75856908)
\curveto(581.1483701,912.13389603)(580.58046621,912.32156316)(579.86034703,912.32157102)
\curveto(579.0582496,912.32156316)(578.34983341,912.09160485)(577.73509633,911.6316954)
\curveto(577.12035076,911.17705801)(576.65490376,910.52947427)(576.33875393,909.68894223)
\lineto(582.14367423,909.69687184)
\moveto(583.54879866,908.55500875)
\lineto(576.03138296,908.55500875)
\curveto(575.99625234,908.35412154)(575.97283363,908.1955296)(575.96112674,908.07923246)
\curveto(575.94941491,907.96292809)(575.94356023,907.86248653)(575.94356268,907.77790748)
\curveto(575.94356023,907.04309486)(576.1923841,906.47480709)(576.69003503,906.07304245)
\curveto(577.19353425,905.67127461)(577.90195045,905.47039148)(578.81528573,905.47039249)
\curveto(579.51784192,905.47039148)(580.18234802,905.54175786)(580.80880602,905.68449182)
\curveto(581.43524939,905.82722334)(582.01778997,906.03603606)(582.55642953,906.3109306)
\lineto(582.24905856,904.86774253)
\curveto(581.66943656,904.65628621)(581.06933193,904.49769427)(580.44874288,904.39196624)
\curveto(579.83399461,904.28623836)(579.20754392,904.23337438)(578.56938896,904.23337415)
\curveto(577.20524361,904.23337438)(576.15432868,904.52676946)(575.41664102,905.11356028)
\curveto(574.68480419,905.70563619)(574.31888673,906.54353026)(574.31888756,907.62724499)
\curveto(574.31888673,908.55236146)(574.50038179,909.41140112)(574.86337327,910.20436654)
\curveto(575.2322167,911.00260688)(575.77084719,911.7136274)(576.47926637,912.33743023)
\curveto(576.93592837,912.7286158)(577.4774862,913.02994048)(578.10394149,913.24140517)
\curveto(578.73624224,913.45285231)(579.40660302,913.55858027)(580.11502583,913.55858936)
\curveto(581.22740827,913.55858027)(582.11146484,913.25725559)(582.76719819,912.65461442)
\curveto(583.42876768,912.05195687)(583.75955706,911.24313799)(583.75956732,910.22815536)
\curveto(583.75955706,909.98497529)(583.74199302,909.72329859)(583.70687516,909.44312448)
\curveto(583.67173687,909.16822681)(583.61904475,908.87218853)(583.54879866,908.55500875)
}
}
{
\newrgbcolor{curcolor}{0 0 0}
\pscustom[linestyle=none,fillstyle=solid,fillcolor=curcolor]
{
\newpath
\moveto(593.55153076,913.08281307)
\lineto(593.24415979,911.70306184)
\curveto(592.81675952,911.91451052)(592.36594921,912.07310246)(591.89172753,912.17883813)
\curveto(591.41749117,912.28455837)(590.92862545,912.33742235)(590.4251289,912.33743023)
\curveto(589.57619454,912.33742235)(588.90583377,912.2052624)(588.41404456,911.94094999)
\curveto(587.92810233,911.68190901)(587.68513314,911.32772035)(587.68513626,910.87838295)
\curveto(587.68513314,910.35502314)(588.25303703,909.9532569)(589.38884964,909.67308302)
\curveto(589.47666499,909.65193222)(589.54106647,909.63607303)(589.58205424,909.62550539)
\lineto(590.10019388,909.48277251)
\curveto(591.17744932,909.2131612)(591.89464753,908.93033891)(592.25179066,908.6343048)
\curveto(592.61477308,908.33826234)(592.79626814,907.9338529)(592.79627638,907.42107527)
\curveto(592.79626814,906.48009348)(592.38058591,905.71356579)(591.54922845,905.12148988)
\curveto(590.72371168,904.52941266)(589.64059601,904.23337438)(588.2998782,904.23337415)
\curveto(587.77880801,904.23337438)(587.23139549,904.27830876)(586.65763902,904.36817743)
\curveto(586.08387836,904.45804629)(585.451573,904.59813583)(584.76072104,904.78844648)
\lineto(585.07687404,906.29507139)
\curveto(585.66819613,906.02017687)(586.25073672,905.81136415)(586.82449755,905.66863261)
\curveto(587.39825385,905.52589866)(587.9485937,905.45453229)(588.47551876,905.45453328)
\curveto(589.26589654,905.45453229)(589.90698392,905.60783783)(590.39878282,905.91445036)
\curveto(590.89642472,906.22105999)(591.14524859,906.60961023)(591.14525517,907.08010226)
\curveto(591.14524859,907.58759384)(590.49537919,907.99728968)(589.19564503,908.309191)
\lineto(589.0287865,908.34883902)
\lineto(588.47551876,908.4757127)
\curveto(587.65585974,908.67130541)(587.05575512,908.92769571)(586.67520308,909.24488436)
\curveto(586.29464681,909.56734985)(586.10436973,909.97704569)(586.10437128,910.4739731)
\curveto(586.10436973,911.42023232)(586.49663325,912.17090082)(587.28116299,912.72598086)
\curveto(588.07154197,913.28104438)(589.14294828,913.55858027)(590.49538512,913.55858936)
\curveto(591.028155,913.55858027)(591.54629412,913.51893228)(592.04980403,913.43964529)
\curveto(592.55915363,913.36034034)(593.05972871,913.24139639)(593.55153076,913.08281307)
}
}
{
\newrgbcolor{curcolor}{0 0 0}
\pscustom[linestyle=none,fillstyle=solid,fillcolor=curcolor]
{
\newpath
\moveto(597.23119703,916.80179771)
\lineto(598.84709012,916.80179771)
\lineto(597.30145325,909.60964618)
\lineto(602.84291272,913.34449003)
\lineto(604.94181734,913.34449003)
\lineto(598.75926985,909.09422187)
\lineto(603.27323208,904.46333269)
\lineto(601.3148399,904.46333269)
\lineto(597.11703067,908.8008265)
\lineto(596.18613573,904.46333269)
\lineto(594.57024264,904.46333269)
\lineto(597.23119703,916.80179771)
}
}
{
\newrgbcolor{curcolor}{0 0 0}
\pscustom[linestyle=none,fillstyle=solid,fillcolor=curcolor]
{
\newpath
\moveto(345.303603,791.03379153)
\curveto(345.303603,787.29670375)(341.94842648,784.26719553)(337.8096065,784.26719553)
\curveto(333.67078652,784.26719553)(330.31561,787.29670375)(330.31561,791.03379153)
\curveto(330.31561,794.77087931)(333.67078652,797.80038753)(337.8096065,797.80038753)
\curveto(341.94842648,797.80038753)(345.303603,794.77087931)(345.303603,791.03379153)
\closepath
}
}
{
\newrgbcolor{curcolor}{0 0 0}
\pscustom[linewidth=1.42420289,linecolor=curcolor]
{
\newpath
\moveto(345.303603,791.03379153)
\curveto(345.303603,787.29670375)(341.94842648,784.26719553)(337.8096065,784.26719553)
\curveto(333.67078652,784.26719553)(330.31561,787.29670375)(330.31561,791.03379153)
\curveto(330.31561,794.77087931)(333.67078652,797.80038753)(337.8096065,797.80038753)
\curveto(341.94842648,797.80038753)(345.303603,794.77087931)(345.303603,791.03379153)
\closepath
}
}
{
\newrgbcolor{curcolor}{0 0 0}
\pscustom[linestyle=none,fillstyle=solid,fillcolor=curcolor]
{
\newpath
\moveto(359.1887853,822.8163439)
\lineto(367.47901944,822.8163439)
\lineto(367.18921253,821.46831109)
\lineto(360.66416596,821.46831109)
\lineto(359.90891157,817.96342578)
\lineto(366.17049732,817.96342578)
\lineto(365.8806904,816.61539297)
\lineto(359.61910466,816.61539297)
\lineto(358.70577378,812.32547678)
\lineto(365.38011482,812.32547678)
\lineto(365.09030791,810.97744397)
\lineto(356.63321525,810.97744397)
\lineto(359.1887853,822.8163439)
}
}
{
\newrgbcolor{curcolor}{0 0 0}
\pscustom[linestyle=none,fillstyle=solid,fillcolor=curcolor]
{
\newpath
\moveto(383.68186018,816.3378568)
\lineto(382.53141456,810.97744397)
\lineto(380.91552146,810.97744397)
\lineto(382.04840303,816.29027917)
\curveto(382.09522595,816.51758897)(382.13035403,816.71318569)(382.15378737,816.87706992)
\curveto(382.17719146,817.04094236)(382.18890082,817.1810319)(382.18891548,817.29733898)
\curveto(382.18890082,817.77310847)(382.04253384,818.14315632)(381.74981409,818.40748364)
\curveto(381.45706591,818.67179611)(381.04723836,818.80395606)(380.52033022,818.80396388)
\curveto(379.7357902,818.80395606)(379.03615603,818.53699296)(378.4214256,818.0030738)
\curveto(377.80667338,817.47442699)(377.40855519,816.78455206)(377.22706983,815.93344695)
\lineto(376.14688043,810.97744397)
\lineto(374.53098733,810.97744397)
\lineto(375.68143296,816.29027917)
\curveto(375.72826224,816.48587058)(375.76339032,816.66560811)(375.78681729,816.82949229)
\curveto(375.81022775,816.99865117)(375.82193711,817.14931351)(375.8219454,817.28147977)
\curveto(375.82193711,817.76253567)(375.67557013,818.13522672)(375.38284402,818.39955404)
\curveto(375.0901022,818.66915291)(374.68612933,818.80395606)(374.1709242,818.80396388)
\curveto(373.37468118,818.80395606)(372.66919232,818.53699296)(372.05445552,818.0030738)
\curveto(371.43970968,817.47442699)(371.04159149,816.78455206)(370.86009976,815.93344695)
\lineto(369.77991035,810.97744397)
\lineto(368.16401726,810.97744397)
\lineto(370.08728132,819.85860132)
\lineto(371.70317442,819.85860132)
\lineto(371.39580345,818.47885009)
\curveto(371.83490053,818.99690958)(372.34718496,819.39074622)(372.93265829,819.6603612)
\curveto(373.5239755,819.93524521)(374.15628086,820.07269155)(374.82957627,820.07270065)
\curveto(375.54383985,820.07269155)(376.12930777,819.90352682)(376.58598181,819.56520594)
\curveto(377.04849242,819.22686789)(377.32658968,818.75637847)(377.42027444,818.15373629)
\curveto(377.91205761,818.77752407)(378.4799615,819.25329988)(379.12398781,819.58106515)
\curveto(379.77384562,819.90881322)(380.45884309,820.07269155)(381.17898229,820.07270065)
\curveto(382.02204246,820.07269155)(382.67191186,819.85330604)(383.12859244,819.41454345)
\curveto(383.5910965,818.97576399)(383.82235634,818.35461223)(383.82237263,817.55108633)
\curveto(383.82235634,817.37662862)(383.81064698,817.1836751)(383.78724452,816.97222518)
\curveto(383.76380954,816.76604967)(383.72868147,816.55459375)(383.68186018,816.3378568)
}
}
{
\newrgbcolor{curcolor}{0 0 0}
\pscustom[linestyle=none,fillstyle=solid,fillcolor=curcolor]
{
\newpath
\moveto(394.72086873,816.04446142)
\lineto(393.62311527,810.97744397)
\lineto(392.00722217,810.97744397)
\lineto(392.30581112,812.32547678)
\curveto(391.83157485,811.79683565)(391.29001702,811.4003558)(390.68113599,811.13603607)
\curveto(390.07809841,810.87700241)(389.40481029,810.74748566)(388.66126962,810.74748543)
\curveto(387.82404689,810.74748566)(387.13612207,810.97744397)(386.59749312,811.43736105)
\curveto(386.06471577,811.9025636)(385.79832786,812.49728337)(385.7983286,813.22152212)
\curveto(385.79832786,814.25765387)(386.25499284,815.07704554)(387.16832492,815.6796996)
\curveto(388.08750745,816.28234426)(389.35211818,816.58366894)(390.96216088,816.58367455)
\lineto(393.21035997,816.58367455)
\lineto(393.29818024,816.97222518)
\curveto(393.30988136,817.01451037)(393.31866338,817.05944475)(393.32452633,817.10702846)
\curveto(393.33037274,817.15988631)(393.33330008,817.23918228)(393.33330835,817.3449166)
\curveto(393.33330008,817.81539965)(393.11960429,818.1801611)(392.69222033,818.43920206)
\curveto(392.27067579,818.7035145)(391.67642585,818.83567444)(390.90946871,818.8356823)
\curveto(390.38254173,818.83567444)(389.8409839,818.77488087)(389.28479359,818.65330139)
\curveto(388.73444951,818.53170656)(388.16654563,818.34932584)(387.58108022,818.10615866)
\lineto(387.8621051,819.45419147)
\curveto(388.47098895,819.66035251)(389.06523889,819.81365805)(389.64485673,819.91410855)
\curveto(390.23032007,820.01982757)(390.79529662,820.07269155)(391.33978807,820.07270065)
\curveto(392.49900829,820.07269155)(393.38013752,819.84537644)(393.9831784,819.39075464)
\curveto(394.59205613,818.936116)(394.89649945,818.27531627)(394.89650928,817.40835344)
\curveto(394.89649945,817.23389588)(394.88186275,817.02772636)(394.85259914,816.78984427)
\curveto(394.82331596,816.55723695)(394.77940586,816.30877625)(394.72086873,816.04446142)
\moveto(392.97324522,815.44181146)
\lineto(391.35735212,815.44181146)
\curveto(390.03418831,815.44180699)(389.05352953,815.28057186)(388.41537285,814.95810557)
\curveto(387.78306413,814.64091771)(387.46691145,814.14663951)(387.46691386,813.47526947)
\curveto(387.46691145,813.01006396)(387.62791513,812.64530251)(387.94992538,812.38098402)
\curveto(388.27778453,812.11666272)(388.72859484,811.98450277)(389.30235765,811.98450378)
\curveto(390.18055529,811.98450277)(390.94751828,812.26732506)(391.6032489,812.83297149)
\curveto(392.25896644,813.40390061)(392.6922127,814.1651419)(392.902989,815.11669766)
\lineto(392.97324522,815.44181146)
}
}
{
\newrgbcolor{curcolor}{0 0 0}
\pscustom[linestyle=none,fillstyle=solid,fillcolor=curcolor]
{
\newpath
\moveto(399.3841252,823.315909)
\lineto(401.00001829,823.315909)
\lineto(400.60482705,821.46831109)
\lineto(398.98893395,821.46831109)
\lineto(399.3841252,823.315909)
\moveto(398.64643487,819.85860132)
\lineto(400.26232797,819.85860132)
\lineto(398.3390639,810.97744397)
\lineto(396.72317081,810.97744397)
\lineto(398.64643487,819.85860132)
}
}
{
\newrgbcolor{curcolor}{0 0 0}
\pscustom[linestyle=none,fillstyle=solid,fillcolor=curcolor]
{
\newpath
\moveto(404.37231697,823.315909)
\lineto(405.98821007,823.315909)
\lineto(403.32725568,810.97744397)
\lineto(401.71136258,810.97744397)
\lineto(404.37231697,823.315909)
}
}
{
\newrgbcolor{curcolor}{0 0 0}
\pscustom[linestyle=none,fillstyle=solid,fillcolor=curcolor]
{
}
}
{
\newrgbcolor{curcolor}{0 0 0}
\pscustom[linestyle=none,fillstyle=solid,fillcolor=curcolor]
{
\newpath
\moveto(414.83171061,822.8163439)
\lineto(417.492665,822.8163439)
\lineto(419.02073782,814.8391615)
\lineto(424.09674982,822.8163439)
\lineto(426.82796043,822.8163439)
\lineto(424.27239037,810.97744397)
\lineto(422.55111295,810.97744397)
\lineto(424.78174798,821.35729662)
\lineto(419.63547975,813.25324054)
\lineto(417.9054203,813.25324054)
\lineto(416.23683504,821.38901504)
\lineto(414.00620001,810.97744397)
\lineto(412.27614055,810.97744397)
\lineto(414.83171061,822.8163439)
}
}
{
\newrgbcolor{curcolor}{0 0 0}
\pscustom[linestyle=none,fillstyle=solid,fillcolor=curcolor]
{
\newpath
\moveto(436.97998922,816.04446142)
\lineto(435.88223576,810.97744397)
\lineto(434.26634267,810.97744397)
\lineto(434.56493161,812.32547678)
\curveto(434.09069534,811.79683565)(433.54913751,811.4003558)(432.94025649,811.13603607)
\curveto(432.3372189,810.87700241)(431.66393078,810.74748566)(430.92039012,810.74748543)
\curveto(430.08316738,810.74748566)(429.39524257,810.97744397)(428.85661361,811.43736105)
\curveto(428.32383626,811.9025636)(428.05744835,812.49728337)(428.05744909,813.22152212)
\curveto(428.05744835,814.25765387)(428.51411334,815.07704554)(429.42744541,815.6796996)
\curveto(430.34662795,816.28234426)(431.61123867,816.58366894)(433.22128137,816.58367455)
\lineto(435.46948046,816.58367455)
\lineto(435.55730074,816.97222518)
\curveto(435.56900186,817.01451037)(435.57778388,817.05944475)(435.58364682,817.10702846)
\curveto(435.58949324,817.15988631)(435.59242057,817.23918228)(435.59242885,817.3449166)
\curveto(435.59242057,817.81539965)(435.37872478,818.1801611)(434.95134083,818.43920206)
\curveto(434.52979629,818.7035145)(433.93554634,818.83567444)(433.16858921,818.8356823)
\curveto(432.64166222,818.83567444)(432.10010439,818.77488087)(431.54391408,818.65330139)
\curveto(430.99357001,818.53170656)(430.42566612,818.34932584)(429.84020071,818.10615866)
\lineto(430.1212256,819.45419147)
\curveto(430.73010944,819.66035251)(431.32435939,819.81365805)(431.90397722,819.91410855)
\curveto(432.48944056,820.01982757)(433.05441711,820.07269155)(433.59890856,820.07270065)
\curveto(434.75812878,820.07269155)(435.63925801,819.84537644)(436.2422989,819.39075464)
\curveto(436.85117662,818.936116)(437.15561994,818.27531627)(437.15562978,817.40835344)
\curveto(437.15561994,817.23389588)(437.14098324,817.02772636)(437.11171964,816.78984427)
\curveto(437.08243645,816.55723695)(437.03852635,816.30877625)(436.97998922,816.04446142)
\moveto(435.23236571,815.44181146)
\lineto(433.61647262,815.44181146)
\curveto(432.29330881,815.44180699)(431.31265003,815.28057186)(430.67449334,814.95810557)
\curveto(430.04218463,814.64091771)(429.72603195,814.14663951)(429.72603435,813.47526947)
\curveto(429.72603195,813.01006396)(429.88703563,812.64530251)(430.20904587,812.38098402)
\curveto(430.53690502,812.11666272)(430.98771533,811.98450277)(431.56147814,811.98450378)
\curveto(432.43967579,811.98450277)(433.20663877,812.26732506)(433.86236939,812.83297149)
\curveto(434.51808693,813.40390061)(434.95133319,814.1651419)(435.16210949,815.11669766)
\lineto(435.23236571,815.44181146)
}
}
{
\newrgbcolor{curcolor}{0 0 0}
\pscustom[linestyle=none,fillstyle=solid,fillcolor=curcolor]
{
\newpath
\moveto(446.36797257,818.51056851)
\curveto(446.20403353,818.58985694)(446.0166838,818.65065052)(445.8059228,818.69294942)
\curveto(445.59514689,818.73523288)(445.36974174,818.75637847)(445.12970667,818.75638625)
\curveto(444.26906204,818.75637847)(443.51673575,818.46034019)(442.87272555,817.86827052)
\curveto(442.22870631,817.28147346)(441.80131472,816.49380018)(441.59054951,815.5052483)
\lineto(440.59818038,810.97744397)
\lineto(438.98228729,810.97744397)
\lineto(440.90555135,819.85860132)
\lineto(442.52144444,819.85860132)
\lineto(442.21407348,818.47885009)
\curveto(442.6414612,818.99162318)(443.15081829,819.38545982)(443.74214629,819.6603612)
\curveto(444.33931819,819.93524521)(444.97455089,820.07269155)(445.6478463,820.07270065)
\curveto(445.82347938,820.07269155)(445.99619242,820.06211876)(446.16598594,820.04098223)
\curveto(446.33576382,820.02511397)(446.50554952,819.99603878)(446.67534354,819.95375657)
\lineto(446.36797257,818.51056851)
}
}
{
\newrgbcolor{curcolor}{0 0 0}
\pscustom[linestyle=none,fillstyle=solid,fillcolor=curcolor]
{
\newpath
\moveto(449.05527354,823.315909)
\lineto(450.67116663,823.315909)
\lineto(449.12552976,816.12375747)
\lineto(454.66698923,819.85860132)
\lineto(456.76589385,819.85860132)
\lineto(450.58334636,815.60833316)
\lineto(455.09730859,810.97744397)
\lineto(453.13891641,810.97744397)
\lineto(448.94110718,815.31493778)
\lineto(448.01021224,810.97744397)
\lineto(446.39431915,810.97744397)
\lineto(449.05527354,823.315909)
}
}
{
\newrgbcolor{curcolor}{0 0 0}
\pscustom[linestyle=none,fillstyle=solid,fillcolor=curcolor]
{
\newpath
\moveto(464.84535968,816.21098312)
\curveto(464.86291506,816.29556025)(464.87462442,816.38278582)(464.88048779,816.47266008)
\curveto(464.89218846,816.56252335)(464.89804314,816.65239211)(464.89805184,816.74226664)
\curveto(464.89804314,817.38720142)(464.68727469,817.89733882)(464.26574585,818.27268036)
\curveto(463.85005555,818.64800732)(463.28215166,818.83567444)(462.56203248,818.8356823)
\curveto(461.75993505,818.83567444)(461.05151886,818.60571613)(460.43678178,818.14580669)
\curveto(459.82203621,817.6911693)(459.35658921,817.04358556)(459.04043937,816.20305352)
\lineto(464.84535968,816.21098312)
\moveto(466.25048411,815.06912003)
\lineto(458.7330684,815.06912003)
\curveto(458.69793779,814.86823282)(458.67451907,814.70964089)(458.66281218,814.59334375)
\curveto(458.65110036,814.47703938)(458.64524568,814.37659782)(458.64524813,814.29201877)
\curveto(458.64524568,813.55720614)(458.89406955,812.98891837)(459.39172048,812.58715374)
\curveto(459.8952197,812.18538589)(460.60363589,811.98450277)(461.51697118,811.98450378)
\curveto(462.21952737,811.98450277)(462.88403347,812.05586914)(463.51049147,812.19860311)
\curveto(464.13693483,812.34133463)(464.71947542,812.55014735)(465.25811498,812.82504188)
\lineto(464.95074401,811.38185381)
\curveto(464.37112201,811.17039749)(463.77101738,811.01180556)(463.15042833,810.90607753)
\curveto(462.53568005,810.80034964)(461.90922937,810.74748566)(461.27107441,810.74748543)
\curveto(459.90692906,810.74748566)(458.85601413,811.04088075)(458.11832647,811.62767156)
\curveto(457.38648963,812.21974748)(457.02057218,813.05764154)(457.02057301,814.14135628)
\curveto(457.02057218,815.06647274)(457.20206724,815.9255124)(457.56505872,816.71847783)
\curveto(457.93390215,817.51671817)(458.47253264,818.22773869)(459.18095182,818.85154151)
\curveto(459.63761381,819.24272708)(460.17917165,819.54405176)(460.80562694,819.75551646)
\curveto(461.43792769,819.96696359)(462.10828847,820.07269155)(462.81671128,820.07270065)
\curveto(463.92909372,820.07269155)(464.81315029,819.77136687)(465.46888364,819.1687257)
\curveto(466.13045313,818.56606815)(466.4612425,817.75724927)(466.46125277,816.74226664)
\curveto(466.4612425,816.49908657)(466.44367847,816.23740988)(466.4085606,815.95723577)
\curveto(466.37342232,815.6823381)(466.3207302,815.38629982)(466.25048411,815.06912003)
}
}
{
\newrgbcolor{curcolor}{0 0 0}
\pscustom[linestyle=none,fillstyle=solid,fillcolor=curcolor]
{
\newpath
\moveto(474.86565583,819.85860132)
\lineto(474.61975906,818.72466783)
\lineto(471.3967549,818.72466783)
\lineto(470.3516936,813.90346813)
\curveto(470.31656243,813.72372768)(470.29021638,813.57306534)(470.27265535,813.45148066)
\curveto(470.2550883,813.32989103)(470.24630628,813.23473587)(470.24630927,813.16601489)
\curveto(470.24630628,812.82768323)(470.35754519,812.58186573)(470.58002632,812.42856164)
\curveto(470.80835549,812.27525465)(471.17134561,812.19860189)(471.66899775,812.19860311)
\lineto(473.3024549,812.19860311)
\lineto(473.03021205,810.97744397)
\lineto(471.48457517,810.97744397)
\curveto(470.52440355,810.97744397)(469.80720534,811.1466087)(469.33297839,811.48493868)
\curveto(468.86460198,811.82326763)(468.6304148,812.33604823)(468.63041617,813.023282)
\curveto(468.6304148,813.14486711)(468.63919682,813.27702706)(468.65676226,813.41976224)
\curveto(468.6743249,813.56777894)(468.70067096,813.72901408)(468.73580051,813.90346813)
\lineto(469.7808618,818.72466783)
\lineto(468.41086548,818.72466783)
\lineto(468.66554428,819.85860132)
\lineto(470.00919452,819.85860132)
\lineto(470.55368024,822.38021564)
\lineto(472.16957333,822.38021564)
\lineto(471.63386964,819.85860132)
\lineto(474.86565583,819.85860132)
}
}
{
\newrgbcolor{curcolor}{0 0 0}
\pscustom[linestyle=none,fillstyle=solid,fillcolor=curcolor]
{
\newpath
\moveto(477.61442709,823.315909)
\lineto(479.23032018,823.315909)
\lineto(478.83512894,821.46831109)
\lineto(477.21923584,821.46831109)
\lineto(477.61442709,823.315909)
\moveto(476.87673676,819.85860132)
\lineto(478.49262986,819.85860132)
\lineto(476.56936579,810.97744397)
\lineto(474.9534727,810.97744397)
\lineto(476.87673676,819.85860132)
}
}
{
\newrgbcolor{curcolor}{0 0 0}
\pscustom[linestyle=none,fillstyle=solid,fillcolor=curcolor]
{
\newpath
\moveto(489.32965609,816.3378568)
\lineto(488.17921046,810.97744397)
\lineto(486.55453534,810.97744397)
\lineto(487.70498097,816.28234956)
\curveto(487.75766468,816.53080496)(487.79864744,816.75019047)(487.82792935,816.94050676)
\curveto(487.85719423,817.13081112)(487.87183093,817.28147346)(487.87183949,817.39249423)
\curveto(487.87183093,817.83654524)(487.71668193,818.1828043)(487.40639202,818.43127246)
\curveto(487.09608593,818.67972571)(486.66283966,818.80395606)(486.10665193,818.80396388)
\curveto(485.2401526,818.80395606)(484.49368099,818.54227936)(483.86723487,818.01893301)
\curveto(483.24077963,817.50085897)(482.83095208,816.79512486)(482.63775099,815.90172854)
\lineto(481.55756158,810.97744397)
\lineto(479.94166849,810.97744397)
\lineto(481.8473685,819.85860132)
\lineto(483.46326159,819.85860132)
\lineto(483.13832657,818.46299088)
\curveto(483.58913304,818.97576399)(484.13069087,819.37224383)(484.76300169,819.65243159)
\curveto(485.3953016,819.93260201)(486.05980769,820.07269155)(486.75652197,820.07270065)
\curveto(487.62300706,820.07269155)(488.29336784,819.86123564)(488.76760632,819.43833226)
\curveto(489.24768056,819.01541197)(489.48772241,818.42069221)(489.48773259,817.65417119)
\curveto(489.48772241,817.46385419)(489.47308571,817.26032787)(489.44382245,817.04359162)
\curveto(489.4203936,816.82684324)(489.38233818,816.59159854)(489.32965609,816.3378568)
}
}
{
\newrgbcolor{curcolor}{0 0 0}
\pscustom[linestyle=none,fillstyle=solid,fillcolor=curcolor]
{
\newpath
\moveto(501.44885274,819.85860132)
\lineto(499.77148545,812.07965903)
\curveto(499.44946904,810.57303453)(498.83180038,809.44967498)(497.91847761,808.70957701)
\curveto(497.01099513,807.96948357)(495.78736716,807.59943572)(494.24759003,807.59943234)
\curveto(493.67968262,807.59943572)(493.15276149,807.6390837)(492.66682505,807.71837641)
\curveto(492.18088473,807.79238924)(491.72714709,807.90868999)(491.30561076,808.06727902)
\lineto(491.6041997,809.48667827)
\curveto(492.00231701,809.25407826)(492.42385392,809.08227033)(492.86881169,808.97125396)
\curveto(493.31376517,808.86024161)(493.78799419,808.80473444)(494.29150017,808.80473226)
\curveto(495.31606548,808.80473444)(496.15621195,809.05848154)(496.81194212,809.56597432)
\curveto(497.47351479,810.06818353)(497.9096884,810.79242005)(498.12046424,811.73868603)
\lineto(498.26097669,812.40477283)
\curveto(497.81015885,811.93956839)(497.28616505,811.58537973)(496.68899373,811.34220579)
\curveto(496.09181048,811.09903112)(495.44779576,810.97744397)(494.75694764,810.97744397)
\curveto(493.76164813,810.97744397)(492.97712111,811.27348225)(492.40336422,811.86555971)
\curveto(491.83545865,812.46292178)(491.55150671,813.28231345)(491.55150753,814.32373719)
\curveto(491.55150671,815.14312551)(491.72421975,815.94401479)(492.06964717,816.72640743)
\curveto(492.4150719,817.51407497)(492.90101028,818.21187949)(493.52746376,818.81982309)
\curveto(493.94314319,819.22158149)(494.42615423,819.53083577)(494.97649833,819.74758685)
\curveto(495.53268861,819.96432039)(496.11230186,820.07269155)(496.71533981,820.07270065)
\curveto(497.37691258,820.07269155)(497.95359849,819.93260201)(498.44539927,819.65243159)
\curveto(498.93718461,819.37753023)(499.30310206,818.98633678)(499.54315273,818.47885009)
\lineto(499.82417762,819.85860132)
\lineto(501.44885274,819.85860132)
\moveto(499.01623107,816.59953376)
\curveto(499.01622278,817.31847825)(498.82887304,817.87355003)(498.4541813,818.26475076)
\curveto(498.07947409,818.65593691)(497.54669828,818.85153364)(496.85585226,818.85154151)
\curveto(496.42845454,818.85153364)(496.02155433,818.77488087)(495.63515041,818.62158297)
\curveto(495.24873667,818.46826979)(494.91794729,818.25417067)(494.64278128,817.97928499)
\curveto(494.19782174,817.52464777)(493.84946832,816.98807838)(493.59771998,816.36957522)
\curveto(493.35182058,815.75634767)(493.22887232,815.12197992)(493.22887482,814.46647007)
\curveto(493.22887232,813.73694367)(493.41622206,813.1765855)(493.79092459,812.78539386)
\curveto(494.17147568,812.39419861)(494.71596085,812.19860189)(495.42438174,812.19860311)
\curveto(496.44894592,812.19860189)(497.30372909,812.61622732)(497.98873383,813.45148066)
\curveto(498.67372404,814.29201545)(499.01622278,815.34136543)(499.01623107,816.59953376)
}
}
{
\newrgbcolor{curcolor}{0 0 0}
\pscustom[linestyle=none,fillstyle=solid,fillcolor=curcolor]
{
\newpath
\moveto(257.544952,715.35475453)
\curveto(257.544952,711.61766675)(254.18977548,708.58815853)(250.0509555,708.58815853)
\curveto(245.91213552,708.58815853)(242.556959,711.61766675)(242.556959,715.35475453)
\curveto(242.556959,719.09184231)(245.91213552,722.12135053)(250.0509555,722.12135053)
\curveto(254.18977548,722.12135053)(257.544952,719.09184231)(257.544952,715.35475453)
\closepath
}
}
{
\newrgbcolor{curcolor}{0 0 0}
\pscustom[linewidth=1.42420289,linecolor=curcolor]
{
\newpath
\moveto(257.544952,715.35475453)
\curveto(257.544952,711.61766675)(254.18977548,708.58815853)(250.0509555,708.58815853)
\curveto(245.91213552,708.58815853)(242.556959,711.61766675)(242.556959,715.35475453)
\curveto(242.556959,719.09184231)(245.91213552,722.12135053)(250.0509555,722.12135053)
\curveto(254.18977548,722.12135053)(257.544952,719.09184231)(257.544952,715.35475453)
\closepath
}
}
{
\newrgbcolor{curcolor}{0 0 0}
\pscustom[linestyle=none,fillstyle=solid,fillcolor=curcolor]
{
\newpath
\moveto(279.23734897,745.85846678)
\lineto(278.89484989,744.29633464)
\curveto(278.30351678,744.58179023)(277.71512151,744.79588935)(277.12966232,744.93863262)
\curveto(276.55004034,745.08664123)(275.98799113,745.1606508)(275.44351301,745.16066156)
\curveto(274.38380901,745.1606508)(273.54073519,744.95183809)(272.91428903,744.53422278)
\curveto(272.28783383,744.11658722)(271.97460849,743.56151544)(271.97461207,742.86900578)
\curveto(271.97460849,742.48837667)(272.08877473,742.19498158)(272.31711115,741.98881965)
\curveto(272.55129439,741.78792895)(273.14847168,741.57118663)(274.1086448,741.33859205)
\lineto(275.17127015,741.10070391)
\curveto(276.37147262,740.82051813)(277.20576442,740.46368627)(277.67414804,740.03020727)
\curveto(278.1425131,739.60200342)(278.37670027,738.99935406)(278.37671025,738.22225738)
\curveto(278.37670027,737.02752764)(277.85563381,736.05483043)(276.81350933,735.30416282)
\curveto(275.77722267,734.55349343)(274.41015506,734.17815917)(272.7123024,734.17815894)
\curveto(272.01559124,734.17815917)(271.31595707,734.24159595)(270.61339778,734.36846946)
\curveto(269.91083404,734.49005665)(269.20534519,734.67772378)(268.4969291,734.9314714)
\lineto(268.85699224,736.58082919)
\curveto(269.50686117,736.21606556)(270.15673057,735.94117287)(270.80660239,735.75615029)
\curveto(271.46232405,735.57112502)(272.11512079,735.47861306)(272.76499456,735.47861413)
\curveto(273.87152457,735.47861306)(274.75850848,735.70064177)(275.42594895,736.14470093)
\curveto(276.09337535,736.58875662)(276.42709207,737.16761718)(276.42710011,737.88128437)
\curveto(276.42709207,738.35705671)(276.29243445,738.71653177)(276.02312683,738.95971062)
\curveto(275.75965864,739.20816677)(275.18882741,739.43283868)(274.31063143,739.63372703)
\lineto(273.24800608,739.87954478)
\curveto(272.03608262,740.16500479)(271.2076455,740.49276146)(270.76269225,740.86281577)
\curveto(270.32358893,741.23814356)(270.10403846,741.76149696)(270.10404017,742.43287751)
\curveto(270.10403846,743.61173622)(270.60461354,744.57650384)(271.60576691,745.32718326)
\curveto(272.61276853,746.08312724)(273.92128934,746.46110469)(275.53133328,746.46111674)
\curveto(276.15777683,746.46110469)(276.77837283,746.41088391)(277.39312315,746.31045425)
\curveto(278.00785548,746.21000079)(278.6225968,746.05933845)(279.23734897,745.85846678)
}
}
{
\newrgbcolor{curcolor}{0 0 0}
\pscustom[linestyle=none,fillstyle=solid,fillcolor=curcolor]
{
\newpath
\moveto(281.01131901,737.928862)
\lineto(282.16176464,743.28927483)
\lineto(283.78643976,743.28927483)
\lineto(282.63599413,737.98436924)
\curveto(282.57744451,737.72533216)(282.53353442,737.50330345)(282.50426372,737.31828243)
\curveto(282.4808423,737.1332556)(282.46913294,736.97995006)(282.4691356,736.85836536)
\curveto(282.46913294,736.40901909)(282.6213546,736.06011683)(282.92580104,735.81165753)
\curveto(283.23024125,735.56848182)(283.66056018,735.44689467)(284.21675911,735.44689571)
\curveto(285.08324724,735.44689467)(285.83264619,735.71121456)(286.4649582,736.23985619)
\curveto(287.09725691,736.76849414)(287.5100118,737.47951466)(287.70322411,738.37291987)
\lineto(288.78341351,743.28927483)
\lineto(290.39930661,743.28927483)
\lineto(288.4936066,734.40811748)
\lineto(286.8777135,734.40811748)
\lineto(287.20264853,735.80372792)
\curveto(286.74597615,735.28565953)(286.20149098,734.88389329)(285.56919138,734.598428)
\curveto(284.94273494,734.31824872)(284.2753015,734.17815917)(283.56688907,734.17815894)
\curveto(282.70624745,734.17815917)(282.03588668,734.38961509)(281.55580473,734.81252733)
\curveto(281.07571928,735.24072515)(280.83567743,735.83544491)(280.83567845,736.5966884)
\curveto(280.83567743,736.75527815)(280.85031413,736.95351807)(280.87958859,737.19140876)
\curveto(280.90886092,737.42929388)(280.95277101,737.67511138)(281.01131901,737.928862)
}
}
{
\newrgbcolor{curcolor}{0 0 0}
\pscustom[linestyle=none,fillstyle=solid,fillcolor=curcolor]
{
\newpath
\moveto(299.24280894,741.94124202)
\curveto(299.0788699,742.02053045)(298.89152016,742.08132403)(298.68075917,742.12362293)
\curveto(298.46998326,742.1659064)(298.2445781,742.18705199)(298.00454303,742.18705977)
\curveto(297.1438984,742.18705199)(296.39157211,741.8910137)(295.74756192,741.29894403)
\curveto(295.10354267,740.71214697)(294.67615109,739.92447369)(294.46538587,738.93592181)
\lineto(293.47301675,734.40811748)
\lineto(291.85712365,734.40811748)
\lineto(293.78038771,743.28927483)
\lineto(295.39628081,743.28927483)
\lineto(295.08890984,741.9095236)
\curveto(295.51629756,742.42229669)(296.02565466,742.81613334)(296.61698266,743.09103471)
\curveto(297.21415455,743.36591872)(297.84938725,743.50336506)(298.52268267,743.50337416)
\curveto(298.69831575,743.50336506)(298.87102879,743.49279227)(299.0408223,743.47165574)
\curveto(299.21060018,743.45578748)(299.38038588,743.42671229)(299.55017991,743.38443009)
\lineto(299.24280894,741.94124202)
}
}
{
\newrgbcolor{curcolor}{0 0 0}
\pscustom[linestyle=none,fillstyle=solid,fillcolor=curcolor]
{
\newpath
\moveto(299.92780759,743.28927483)
\lineto(301.64030299,743.28927483)
\lineto(303.08055553,735.74029109)
\lineto(307.78772237,743.28927483)
\lineto(309.50021777,743.28927483)
\lineto(303.90606613,734.40811748)
\lineto(301.72812327,734.40811748)
\lineto(299.92780759,743.28927483)
}
}
{
\newrgbcolor{curcolor}{0 0 0}
\pscustom[linestyle=none,fillstyle=solid,fillcolor=curcolor]
{
\newpath
\moveto(317.93096544,739.64165663)
\curveto(317.94852083,739.72623377)(317.96023019,739.81345933)(317.96609355,739.90333359)
\curveto(317.97779423,739.99319686)(317.98364891,740.08306562)(317.98365761,740.17294015)
\curveto(317.98364891,740.81787493)(317.77288045,741.32801233)(317.35135162,741.70335387)
\curveto(316.93566132,742.07868083)(316.36775743,742.26634796)(315.64763824,742.26635581)
\curveto(314.84554082,742.26634796)(314.13712463,742.03638965)(313.52238754,741.5764802)
\curveto(312.90764198,741.12184281)(312.44219498,740.47425907)(312.12604514,739.63372703)
\lineto(317.93096544,739.64165663)
\moveto(319.33608987,738.49979355)
\lineto(311.81867417,738.49979355)
\curveto(311.78354356,738.29890633)(311.76012484,738.1403144)(311.74841795,738.02401726)
\curveto(311.73670612,737.90771289)(311.73085144,737.80727133)(311.73085389,737.72269228)
\curveto(311.73085144,736.98787966)(311.97967531,736.41959188)(312.47732625,736.01782725)
\curveto(312.98082547,735.6160594)(313.68924166,735.41517628)(314.60257695,735.41517729)
\curveto(315.30513314,735.41517628)(315.96963924,735.48654265)(316.59609723,735.62927662)
\curveto(317.2225406,735.77200814)(317.80508119,735.98082086)(318.34372074,736.2557154)
\lineto(318.03634978,734.81252733)
\curveto(317.45672777,734.60107101)(316.85662315,734.44247907)(316.2360341,734.33675104)
\curveto(315.62128582,734.23102315)(314.99483514,734.17815917)(314.35668017,734.17815894)
\curveto(312.99253483,734.17815917)(311.9416199,734.47155426)(311.20393223,735.05834507)
\curveto(310.4720954,735.65042099)(310.10617795,736.48831506)(310.10617877,737.57202979)
\curveto(310.10617795,738.49714626)(310.287673,739.35618591)(310.65066449,740.14915134)
\curveto(311.01950791,740.94739168)(311.55813841,741.6584122)(312.26655758,742.28221502)
\curveto(312.72321958,742.67340059)(313.26477741,742.97472527)(313.89123271,743.18618997)
\curveto(314.52353346,743.3976371)(315.19389423,743.50336506)(315.90231705,743.50337416)
\curveto(317.01469949,743.50336506)(317.89875606,743.20204038)(318.55448941,742.59939921)
\curveto(319.21605889,741.99674166)(319.54684827,741.18792278)(319.54685854,740.17294015)
\curveto(319.54684827,739.92976009)(319.52928423,739.66808339)(319.49416637,739.38790928)
\curveto(319.45902808,739.11301161)(319.40633597,738.81697333)(319.33608987,738.49979355)
}
}
{
\newrgbcolor{curcolor}{0 0 0}
\pscustom[linestyle=none,fillstyle=solid,fillcolor=curcolor]
{
\newpath
\moveto(324.80729167,733.58343859)
\curveto(324.05788827,732.44686386)(323.44607428,731.73848655)(322.97184789,731.45830451)
\curveto(322.50347092,731.17284197)(321.90336629,731.03010923)(321.17153221,731.03010585)
\lineto(319.89813819,731.03010585)
\lineto(320.17038105,732.25126498)
\lineto(321.10127599,732.25126498)
\curveto(321.55208554,732.25126714)(321.93556703,732.3622815)(322.25172162,732.58430839)
\curveto(322.56787239,732.80633892)(322.92208049,733.23982355)(323.31434697,733.88476357)
\lineto(323.81492254,734.73323128)
\lineto(321.63697968,743.28927483)
\lineto(323.34947508,743.28927483)
\lineto(324.98293223,736.48567393)
\lineto(329.49689446,743.28927483)
\lineto(331.1918258,743.28927483)
\lineto(324.80729167,733.58343859)
}
}
{
\newrgbcolor{curcolor}{0 0 0}
\pscustom[linestyle=none,fillstyle=solid,fillcolor=curcolor]
{
\newpath
\moveto(339.98263661,743.02759787)
\lineto(339.67526564,741.64784664)
\curveto(339.24786537,741.85929532)(338.79705506,742.01788725)(338.32283338,742.12362293)
\curveto(337.84859702,742.22934317)(337.3597313,742.28220715)(336.85623475,742.28221502)
\curveto(336.00730039,742.28220715)(335.33693961,742.1500472)(334.84515041,741.88573478)
\curveto(334.35920818,741.62669381)(334.11623899,741.27250515)(334.11624211,740.82316774)
\curveto(334.11623899,740.29980794)(334.68414288,739.8980417)(335.81995548,739.61786782)
\curveto(335.90777084,739.59671702)(335.97217232,739.58085782)(336.01316009,739.57029019)
\lineto(336.53129973,739.4275573)
\curveto(337.60855517,739.15794599)(338.32575338,738.8751237)(338.68289651,738.57908959)
\curveto(339.04587893,738.28304714)(339.22737399,737.8786377)(339.22738223,737.36586006)
\curveto(339.22737399,736.42487828)(338.81169176,735.65835059)(337.9803343,735.06627468)
\curveto(337.15481753,734.47419746)(336.07170186,734.17815917)(334.73098405,734.17815894)
\curveto(334.20991386,734.17815917)(333.66250134,734.22309356)(333.08874487,734.31296223)
\curveto(332.51498421,734.40283109)(331.88267884,734.54292063)(331.19182689,734.73323128)
\lineto(331.50797989,736.23985619)
\curveto(332.09930198,735.96496166)(332.68184257,735.75614895)(333.2556034,735.61341741)
\curveto(333.8293597,735.47068346)(334.37969955,735.39931709)(334.90662461,735.39931808)
\curveto(335.69700239,735.39931709)(336.33808977,735.55262263)(336.82988867,735.85923516)
\curveto(337.32753057,736.16584478)(337.57635444,736.55439503)(337.57636102,737.02488706)
\curveto(337.57635444,737.53237864)(336.92648504,737.94207448)(335.62675088,738.2539758)
\lineto(335.45989235,738.29362382)
\lineto(334.90662461,738.4204975)
\curveto(334.08696559,738.61609021)(333.48686097,738.87248051)(333.10630893,739.18966916)
\curveto(332.72575266,739.51213465)(332.53547558,739.92183049)(332.53547713,740.4187579)
\curveto(332.53547558,741.36501711)(332.92773909,742.11568561)(333.71226884,742.67076566)
\curveto(334.50264782,743.22582917)(335.57405413,743.50336506)(336.92649097,743.50337416)
\curveto(337.45926085,743.50336506)(337.97739997,743.46371708)(338.48090987,743.38443009)
\curveto(338.99025948,743.30512514)(339.49083456,743.18618119)(339.98263661,743.02759787)
}
}
\end{pspicture}

	\end{pdfpic}
	\caption[Effort vs. Personalization and the Comparison of Available Solutions]{Effort vs. Personalization and the Comparison of Available Solutions}
	\label{fig:drawingEfforPersonalizationAnnotations}
\end{figure}

\section{Future Work}
\label{sec:6.2:FutuWork}

The provided assistant support in the final solution helps researchers share certain tasks in extracting information from emails, proofreading the primary researcher's replies before sending them, compose replies to recipients' emails, and to verify the rule-based actions before they are acted upon.
\vspace{1cm}

Along with the assistant support, the system could also provide an option to allow a crowd of anonymous workers to perform the tasks of assigned assistants, as a crowd assistant. If the system would be able to provide the required functions to allow the involvement of crowd assistants, the decisions on the possible tasks in a mass email communication can be done by anonymous crowd workers.
\vspace{1cm}

\cite{Surowiecki2005} stated that under the right conditions, groups can be remarkably smart, and even smarter than the smartest person within them. Therefore, if you try to solve a complicated problem or try to make a decision, the best thing to do is to ask a group instead of trying to find an expert. Therefore, taking advantage of crowd assistants can minimize the work a researcher needs to do in a personalized mass email communication.
\vspace{1cm}

Another improvement could be about the dynamic variables in the messages. \ac{KVP}s were used in the messages as dynamic variables to personalize the content of the messages as described in section \ref{subsec:5.2.4:CompEmaiMess}. However, when communicating with larger groups of people, some \ac{KVP}s might not be applicable to some recipients, or other irregular \ac{KVP}s needs to be used to personalize messages for some recipients in a large group. When we reviewed available products in the market as featured in chapter \ref{chp:3:EvalExisAppl}, Zendesk solved this issue by adding conditional logic to the dynamic variables, as illustrated in listing \ref{lst:MailChimCondMergTags}. Such extension would be an option for Myriad to avoid creating additional email templates for closely similar emails, with minor changes on \ac{KVP}s.
\vspace{1cm}

In chapter \ref{chp:3:EvalExisAppl}, we saw that that \ac{CRM} applications provide task management options to remind users about upcoming tasks. For example, Highrise allows association of a task to an email to let users easily browse through the source of the task in the provided task list. Currently, Myriad users can maximize the use of the \ac{KVP}s for the same purpose by adding a key named "task" and a value according to the recipients. However, this contradicts with one design principle, separation of concerns, since the logic behind the \ac{KVP}s is to store extracted information from the recipients' emails and use them to personalize emails. Besides, \ac{KVP}s connected to the recipients' profile in a campaign is not useful in adding task messages related to a message of a recipient. Therefore, the task management feature in Myriad could be useful for researchers in attaching reminder messages as a task along with the recipients' messages, and a list of these tasks are shown at the main page of a campaign.
\vspace{1cm}

One Myriad user mentioned about the email editor and its attachment handling in the user testimonials, as shown in listing \ref{lst:Testimonial1}. Currently, Myriad provides an \ac{HTML} editor to compose an email; however, it does not provide an option for adding attachments. This is due to the limited time for development, and adding this feature was not considered as a priority at the time of the development since users can still use Gmail in composing an email, and Myriad will able to import it, together with its attachment. Therefore, a better \ac{HTML} editor with an email attachment feature would necessary to accomplish all email composing tasks inside the Myriad system.
\vspace{1cm}

Next improvement could be on the visualization tree of the communication state and flow. The current visualization tree's nodes are arranged according to email templates and their order in the communication. However, a user can opt to create all email templates at the beginning of a campaign and use them according to the recipients answer later on. In this case, all the created templates at the beginning of the campaign will be considered as individual root nodes of a tree, losing the hierarchical structure in visualizing the communication state. Therefore, the possible improvement should let Myriad allow the rearrangement of those nodes in the visualization tree, together with a drag-and-drop ability for each node.
\vspace{1cm}

Furthermore, a better workflow can be implemented in the future by the provided rules created according to the decision on sending emails. Currently, Myriad saves filtering conditions of a recipient search as a rule in sending emails the next time, without having the need to search for matching conditions of the recipients again. This feature is quite unintuitive, since when user decides to browse through the rules view, no defined rules will be shown, unless the user uses the provided filtering options in getting a subset of the recipient list and then send an email to the recipients afterwards. To make this feature more intuitive and better define the workflow of a mass communication before-hand, Myriad could offer an option on creating rules right under the rules view. With this, a researcher can easily create the rules and assistants can verify and send emails according to the matching rules, or even opt to automate the process by the provided option under the rules view.
\vspace{1cm}

Finally, the last prospective features concluding this chapter and the study are the analytic reports and the mobile support. As we saw in chapter \ref{chp:3:EvalExisAppl}, the reviewed help desk and email marketing applications provide analytic reports to keep track of the success of a support team or a campaign, respectively. Such an analytic report is also beneficial for a mass email communication campaign to get a quick report about the statistics of a campaign. Myriad provides a table view with some statistics on the response rates and the amount of messages that were sent. However, providing a dashboard summarizing more information with graphics and charts would be a useful feature to help researchers get an overview about a campaign at a glance. Also, a mobile compatible version of Myriad would reduce the dependency on a single platform from the researchers' and assistants' perspective, and provide flexibility as we saw in other applications in chapter \ref{chp:3:EvalExisAppl}.

\clearemptydoublepage 