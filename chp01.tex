\chapter{Introduction}
\label{chp:Intro}
Increased Internet usage turned email as a tool for communication replacing telephone and regular mail \citep{Norman2000,Madden2003}. There are many use cases showing that email plays a huge role as a communication tool. Some of them includes marketing for engaging clients, customer support for offering assistance after sale, surveying people to get their opinion on a topic, and many other cases showing that email become essential part of our daily life.
\vspace{1cm}

However, when the amount of people you want to reach increases, the way how you compose the emails and extract the information change. Because, the personal effort will not be enough anymore to individually tailor the emails according to each recipients or reading out all the respondent's emails to extract the answers that you seek for. As a result, researchers tend to use online or software tools to send out generic emails to recipients with a non-adequate personalization, which is known as one of the important factor to increase response rates \citep{Dillman1991,Schaefer1998}. Such emails are treated with low priority, which results low response rates at the end \citep[page 272]{DillmanDonA.SmythJoleneD.Christian2009}.
\vspace{1cm}

There are several products in the market focusing on email communication and data collection. Customer relationship management (CRM) softwares keep track a company's communication with their clients. Help desk softwares offers a platform to solve customers' problems or provide guidance regarding products. Email marketing applications help out sending commercial messages to group of people. Finally, survey applications aid to conduct online surveys to get people's opinions and behaviors. One of the common property of all of these application is their dependency on email communication. However, none of these mentioned tools offers a complete workflow to help out a researcher to make an email communication possible with a great amount of people in a personalized manner and as easy as possible like communicating with an individual.
\vspace{1cm}

The goal of this thesis is to understand the possible workflow of a personalized mass email communication, and to show that it is possible to reach a great amount of people by keeping the communication personalized at the same time. A complete system, named Myriad, has been developed to demonstrate the practical aspects of the idea.

\section{Email as a data collection method}
\label{sec:1:EmailDataCol}

Nearly 600\% growth rate on world-wide internet usage between 2000 to 2012 makes Europe's 63\% and North America's 80\% overall population internet usage proportion \citep{InternetWorldStats2012}. Email is ranked as the most popular online activity a long with search engine usage with 92\% of online adult users \citep{Purcell2011}. Also, the connectivity and the flexibility have been increased with the introduction of smart phones and tablet devices \citep{Madden2008}. In addition to these facts, email has low cost and quick turnover compared to regular mail or telephone communication \citep{Zikmund2007}. Therefore, email as a part of communication is considered as a viable option for data collection as well \citep{Zikmund2007}.
\vspace{1cm}

There are several reasons of data collection depending on the situation. However, purposes of data collection can be group under the following three categories \citep[pages 92--94]{Babbie2010}:

\begin{compactenum}
	\item To explore and get information about a topic
	\item To describe the events and the situations
	\item To explain things by questioning
\end{compactenum}

To illustrate these purposes to see how we can use email to explore, describe, and explain things, let's suppose that we have an online learning platform offering various courses publicly:

\paragraph{Exploration}
Offering online courses is a relatively new trend, therefore we do not have a previous knowledge about the topic. To explore the popularity of the platform, we need to ask questions about the platform's users: Why are they attending our online courses? Have they taken any online courses before? What are their income level? Figuring out the answers of these question will help us to improve the system or to decide its future. For example, the aggregated answers to the income level question will make us to decide whether to charge the users for their usage or offer it for free but find some sponsors to make it viable.

\paragraph{Description}
Our goal can be to describe characteristics of the online learning platform's users. The questions helping us to describe this can be: Where do they come from? What are their age range? Have they attended to a college? At the end, we might end up with a description of users profile like users, at the age of 16 -- 22, who have never attended to a college, and coming from less developed countries. Knowing our users' portfolio according to this outcome can help us to attract organizations who have already had engagements to support those countries' young population. Hence, they can leverage our platform as a tool to reach those populations.

\paragraph{Explanation}
We figured out that our platform's users' age range is between 16 -- 22 in our descriptive study. The reasons of why this ended up like that make our explanatory purpose. The questions like how often they are connected online or have they attended a college or a similar high level education institute might help us to find out the answer of why young people use our platform than older people. Collecting such statistics may help us to develop an explanation to a topic.

\vspace{1cm}
Since all of our registered users provided their email addresses as a primary and mandatory contact medium, we can use email to conduct our data collection whether the reason is to explore, describe or explain the user trends on our online learning platform.

\section{Problem Statement}
\label{sec:2:Problem}
(paper E21 p 442)To date, researchers have used Web page-based surveys to study large groups of on-line users (e.g. Kehoe, Pitkow and Morton, 1997) and e-mail surveys to study smaller, more homogenous on-line user groups (e.g. Parker, 1992; Smith, 1997; Tse et al, 1995). However, it appears that a relatively untapped use for the Internet is to use e-mail is to survey broader Internet populations on both a national and international basis. WRITE the problems in here as well form that paper. Problems related from researchers perspective. Also mention briefly about the purpose and design of the those surveys.

As aforementioned studies showed that different forms of personalization increase the response rates in email communication. However, it has become very easy to add personalized information into email thanks to the softwares. Dillman, et al. (2009) stated that over-personalization using software tools might easily result impersonal messages.

 Moreover, experienced email users can identify if a message is written by a person or computer generated by looking appearance of one's name in certain locations, and similar patterns for other information \citep[page 272]{DillmanDonA.SmythJoleneD.Christian2009}. Therefore, it becomes difficult to have a correct amount and tone of personalization. The more daily interaction with digital devices will make the true authentic personalization more rare, hence achieving it will make it more important and effective \citep[page 238]{DillmanDonA.SmythJoleneD.Christian2009}.

\vspace{1cm}

The main goal of this thesis is that personalized communication with large groups is possible when a proper workflow is provided. To achieve this goal:

\begin{compactenum}
	\item Examine the workflow of an email communication with large groups and possible exceptional cases on this flow
	\item Investigate the effects of email content's personalization on the response rates
	\item Comparison of  existing products claiming to provide solutions on email communication and recipient's related information collection
	\item Provide an application satisfying the mentioned workflow to aid researchers
	\item Real life use cases of the application and its users opinions about the application
\end{compactenum}
\vspace{1cm}

This thesis also contributes on the following areas:

\begin{compactenum}
	\item Email as a data collection method
	\item Surveying with email
	\item Defining a workflow on a mass email communication
	\item Possible crowdsourced assistant usage
	\item Personalization of email content according to recipients
\end{compactenum}


\section{Outline}
\label{sec:3:Outline}
Outline goes here

%%% Local Variables: 
%%% mode: latex
%%% TeX-master: "thesis"
%%% End: 