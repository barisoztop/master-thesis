\chapter{Introduction}
\label{chp:Intro}
Increased Internet usage turned email into a tool for communication, replacing telephone calls and regular mail \citep{Norman2000,Madden2003}. Email is used in many ways, proving that it plays a huge role in the communication world. Email is popularly used in; marketing, for engaging with clients; customer support, for offering aftersales assistance; research, on gathering the opinion of people on a certain topic; and many other cases showing that email has essentially become a part of our daily lives.
\vspace{1cm}

However, as the number of people you want to reach increases, the way on how you compose emails and extract information changes. Because, each email response should be uniquely composed based on the flow of the conversation to effectively deliver which require an increased personal effort and time on the part of the researchers. As a result, researchers tend to use online tools or third-party applications on sending out generic emails to their recipients with non-adequate personalization, which is known as one important factor needed to increase response rates \citep{Dillman1991,Schaefer1998}. Such emails are treated with a low priority, which results to low response rates at the end \citep[page 272]{DillmanDonA.SmythJoleneD.Christian2009}.
\vspace{1cm}

There are several products in the market focusing on email communication and data collection. A \ac{CRM} application keeps track of a company's communication with their clients. A Help Desk application offers a platform on helping solve customers' problems and provide guidance regarding products. Email marketing applications help on sending out commercial messages to groups of people. Finally, survey applications aid on conducting online surveys in getting people's views and opinions. The similarity of all these applications is that they all focus on email communications. However, none of these mentioned tools are capable of offering a complete workflow in helping out a researcher communicate with a great amount of people on a personalized level, as easy as possible by email.
\vspace{1cm}

The goal of this study is to understand the possible workflow of a personalized mass email communication, and to show that it is possible to reach a great amount of people while keeping it personalized at the same time. A complete system named Myriad has been developed to demonstrate the practical aspects of this idea.

\section{Email as a Data Collection Method}
\label{sec:1:EmailDataCol}

There is nearly a 600\% growth rate in the world-wide internet usage between the years 2000 to 2012 that makes Europe's 63\% and North America's 80\% over-all population internet usage proportion \citep{InternetWorldStats2012}. Email is ranked as the most popular online activity, along with search engine usage with 92\% of online adult users \citep{Purcell2011}. Also, the connectivity and the flexibility have increased with the introduction of smartphones and tablet devices \citep{Madden2008}. In addition to these facts, email is low cost and has a quick turnover compared to regular mail or telephone communication \citep{Zikmund2007}. Therefore, email as a part of communication is considered a viable option for data collection as well \citep{Zikmund2007}.
\vspace{1cm}

There are several reasons for data collection depending on the situation. However, purposes of data collection can be classified under the following categories \citep{Sue2011} \citep[pages 92--94]{Babbie2010}:

\begin{compactenum}
	\item To explore and get information about a topic
	\item To describe the events and the situations
	\item To explain things by questioning
\end{compactenum}

To illustrate these purposes and to see how we can use email to explore, describe, and explain things, let's suppose that we have an online learning platform, offering various courses publicly:

\paragraph{Exploration}
Offering online courses is relatively a new trend; therefore we do not have much previous knowledge about the topic. To explore the popularity of the platform, we would need to ask the platform's users questions such as: Why are they attending our online courses? Have they taken any online courses before? What are their income levels? Figuring out the answers to these questions will help us improve the system or to decide on its future. For example, the aggregated answers to the income level question will help us decide whether to charge the users for their usage or to offer it for free and find some sponsors to make it more viable.

\paragraph{Description}
Our goal is to describe the characteristics of the online learning platform's users. The questions that can help us to describe this can be: Where do they come from? What age range do they belong to? Were they able to attend college? With these questions, we will end up with a user profile like: between ages 16 -- 22; have never attended college, and is coming from a less developed country. Knowing our users' portfolio according to this outcome can help us to attract organizations with necessary connections in supporting such countries' young population. Hence, they can take advantage of our platform as a tool in reaching those populations, in return they monetize our platform.

\paragraph{Explanation}
With our descriptive study, we discovered out that our platform's user's age range is between ages 16 -- 22. The reasons on how our platform's users' age range turned out to be between 16 -- 22 makes up our explanatory purpose. Questions about how often they are connected to the internet or have they attended college or a similar high level of an education institute might help us to figure out the answer on why do young people use our platform more frequently compared to older people. Collecting such statistics may help us to develop an explanation to a topic.

\vspace{1cm}
Since all of our registered users provided their email addresses as the primary and mandatory contact medium, we can send them emails to conduct our data collection whether the reason is to explore, describe or explain the user trends on our online learning platform.

\section{Problem Statement}
\label{sec:2:Problem}

To date, email, as a popular medium for communication, is utilized for many purposes such as reaching groups of people to explore, describe, and explain things. However, as the group's size gets larger, it becomes harder on the researchers' part to maintain the consistency and effectiveness of the flow of the exchange of emails as compared to that of small groups. Therefore, the researchers tend to write generic emails ignoring or using inadequate recipient specific information with the help of a third-party application or an online tool. This results into low response rates, since recipients do realize that because they are a part of a large group of people being responded to means that you will feel less important and less valued, and the chance of getting a reply is less likely to happen. On the other hand, if researchers individually tailor those emails according to their recipients, it will require a huge additional effort at an increased cost, hence reducing the advantage of using email as the primary communication medium.
\vspace{1cm}

Even though, there are many products available in the market supporting email communication, there is just no available product allowing anyone to reach larger groups via email, requiring minimum effort while keeping the communication personalized at the same time.
\vspace{1cm}

The main goal of this study is to show that a personalized email communication with large groups is possible if a proper workflow is provided. In order to achieve this goal, the study will:

\begin{compactenum}
	\item Examine the workflow of an email communication with large groups and possible exceptional cases on this flow.
	\item Investigate the effects of an email's content's personalization on the response rates.
	\item Describe how an adequate amount of personalization in emails can be supplied.
	\item Analyze the comparison of existing products claiming to provide solutions on email communication and collection of the respondents' information.
	\item Describe the design and implementation of an application satisfying the mentioned workflow to aid the researchers, including the initial prototype.
	\item Show how assistants can support the mentioned workflow. 
	\item Analyze real life usage of the application and its users' opinions about the application, and the latest statistical information giving an insight on how and in which way the application is used by its users.
\end{compactenum}
\vspace{1cm}

This study also contributes on the following areas:

\begin{compactenum}
	\item Email as a data collection method.
	\item Conducting surveys with the use of email.
	\item Defining a workflow on a mass email communication.
	\item Possible crowd-sourced assistant usage.
	\item Personalization of email content.
\end{compactenum}

\section{Outline}
\label{sec:3:Outline}

\paragraph{Chapter 1} The first chapter introduces the concept of the personalized mass email communication, defines email as a data collection medium, as well as its purposes and continues with the problem statement and the contributions of this study.
  
\paragraph{Chapter 2} The second chapter gives the necessary foundation on data collection by investigating related work about the email surveys, its errors, the factors affecting the response rates on a research and the studies on the personalization of emails.

\paragraph{Chapter 3} The third chapter is about existing applications available in the market and their connection with mass email communication as well as their features useful in reducing the efforts of a researcher on initiating a mass email campaign.

\paragraph{Chapter 4} The fourth chapter builds up a mass email communication scenario and introduces the prototype built upon to reflect the initial findings on a personalized mass email communication. The prototype will be reviewed, including its requirement analysis and architecture and finally, its evaluation.

\paragraph{Chapter 5} The fifth chapter will introduce the final solution, the developed and enhanced idea of the initial findings on a personalized mass email communication. The final solution will be reviewed, including its improved requirement analysis and architecture, and finally, the benefits of the solution will be described together with the experience of users with it will be brought in, with the statistical results.

\paragraph{Chapter 6} The last chapter will summarize the findings according to the chapters, and mention about the future work for the provided solution.

\begin{comment}


--> We use email also more daily communication, onlar bize soru soruyor iletisim olusuyor.

(paper E21 p 442, and E5)To date, researchers have used Web page-based surveys to study large groups of on-line users (e.g. Kehoe, Pitkow and Morton, 1997) and e-mail surveys to study smaller, more homogenous on-line user groups (e.g. Parker, 1992; Smith, 1997; Tse et al, 1995). However, it appears that a relatively untapped use for the Internet is to use e-mail is to survey broader Internet populations on both a national and international basis. WRITE the problems in here as well form that paper. Problems related from researchers perspective. Also mention briefly about the purpose and design of the those surveys.


Reaching out large-scale of people via internet is a fast and cost efficient way comparing with postal mail or telephone. Therefore, email has been used not just for research, but also for marketing, customer support, and other data collection purposes. However, getting an acceptable response rates on the sent out emails requires additional efforts from the researchers' side. This thesis investigates a communication system which contributes increasing the response rates while minimizing the burden on the researchers' side. 

As aforementioned studies showed that different forms of personalization increase the response rates in email communication. However, it has become very easy to add personalized information into email thanks to the softwares. Dillman, et al. (2009) stated that over-personalization using software tools might easily result impersonal messages.

Moreover, experienced email users can identify if a message is written by a person or computer generated by looking appearance of one's name in certain locations, and similar patterns for other information \citep[page 272]{DillmanDonA.SmythJoleneD.Christian2009}. Therefore, it becomes difficult to have a correct amount and tone of personalization. The more daily interaction with digital devices will make the true authentic personalization more rare, hence achieving it will make it more important and effective \citep[page 238]{DillmanDonA.SmythJoleneD.Christian2009}.

\end{comment}