\chapter{Foundation and Related Work}
\label{chp:FouRelWor}
This chapter presents the related work on the data collection domain. Even though technology is different for email surveys in collecting data from well-established regular mail-surveying methods, the nature of the communication is similar to self-administrated questionnaires \citep{Schaefer1998}. In lieu with this, this chapter will also investigate regular mail surveys in emphasizing points which are also related with email communications, and the earlier studies about response rate influences.

\section{Surveys and Data Collection}
\label{sec:1:SurDatCol}
A Survey is defined as a system for collecting information \citep[page 3]{Sue2011}. It helps to learn about people's opinions and behaviors \citep{DillmanDonA.SmythJoleneD.Christian2009}. The produced data during or at the completion of the survey belongs to the data collection process. Therefore, data collection is a fundamental step in producing useful data to enable analysis on the researcher's part \citep[page 149]{Groves2009}. These researches include -- but not limited to -- many disciplines like sociology, statistics, psychology, marketing, economics, and health sciences. 

\subsection{Email Surveys}
\label{sec:2.1.1:EmaSur}

Comparing many different characteristics of surveys and interviews, the concerns regarding speed and cost make the most powerful differences \citep{Sproull1986, Schaefer1998}. Email surveys offer more rapid surveying than other methods including regular mail and telephone surveys. In addition to that, email surveys are inexpensive since it removes postage, paper and printing, and interview costs \citep{Schaefer1998}.
\vspace{1cm}

\cite{Sproull1986} identified the characteristics of email with organizational research, within a Fortune 500 office products and systems manufacturer, who were using email for 12 years in the organization and over 80 percent of all employees in the selected unit had email access at the time of the research. Selected candidates were separated into two groups. The data collection protocol within the organization asked each of the group's participants series of questions regarding their 3-day old email inbox. Both groups filled out the questionnaire and answered open-ended questions either electronically or in writing.
\vspace{1cm}

The result of the study indicated that the average duration of the data collection process for the email version was less than a week, which is half of the duration of the written version. While the response rate of the email version was 73 percent, the conventional written version's rate was 87. The percentage of missing data in the questionnaires was .2 percent in the written version, and 1.4 in the email version. There were no differences in the nature of answers in the email version compared with the written questionnaire.
\vspace{1cm}

In another study by \cite{Sheehan2006}, they administered an email only survey to query individuals about their online behaviors, attitudes and opinions regarding their privacy concerns. They have reached the shortest response time of 3.65 days, compared with earlier studies conducted until that time (See table~\ref{tab:sur_res_met}).

\begin{center}
	%\renewcommand{\arraystretch}{2}
	\tiny
	\setlength{\tabcolsep}{5pt}
    \begin{longtable}{ | p{2cm} | p{2cm} | p{2cm} | p{0.75cm} | p{0.75cm} | p{1cm} | p{1cm} | p{0.5cm} | }
	\caption[Summary of Survey Research Methods Using E-mail]{Summary of Survey Research Methods Using E-mail \citep{Sheehan2006}} \label{tab:sur_res_met} \\
    
	\hline
	\multicolumn{1}{|p{2cm}|}{\textbf{Author}} & \multicolumn{1}{p{2cm}|}{\textbf{Response Sample}} & \multicolumn{1}{p{2cm}|}{\textbf{Survey Topic}} & \multicolumn{1}{p{0.75cm}|}{\textbf{Sample Size}} & \multicolumn{1}{p{0.75cm}|}{\textbf{Usable Sample}} & \multicolumn{1}{p{1cm}|}{\textbf{Method}} & \multicolumn{1}{p{1cm}|}{\textbf{Response Rate}} & \multicolumn{1}{p{0.5cm}|}{\textbf{Time (days)}} \\ \hline
	\endfirsthead
	
	\multicolumn{8}{c}%
	{{\bfseries \tablename\ \thetable{} -- continued from previous page}} \\
	\hline
	\multicolumn{1}{|p{2cm}|}{\textbf{Author}} & \multicolumn{1}{p{2cm}|}{\textbf{Response Sample}} & \multicolumn{1}{p{2cm}|}{\textbf{Survey Topic}} & \multicolumn{1}{p{0.75cm}|}{\textbf{Sample Size}} & \multicolumn{1}{p{0.75cm}|}{\textbf{Usable Sample}} & \multicolumn{1}{p{1cm}|}{\textbf{Method}} & \multicolumn{1}{p{1cm}|}{\textbf{Response Rate}} & \multicolumn{1}{p{0.5cm}|}{\textbf{Time (days)}} \\ \hline
	\endhead

	\multicolumn{8}{|r|}{{Continued on next page}} \\ \hline
	\endfoot

	\multicolumn{8}{|r|}{{*Differences not significant}} \\ \hline
	\endlastfoot

	
    \multirow{2}{2cm}{Kiesler \& Sproull (1986)}  & \multirow{2}{2cm}{Employees of a Fortune 500} & \multirow{2}{2cm}{Corporate Communication} & 115 & 77 & Mail & 67\% & 10.8 \\ \cline{4-8}
	&  &  & 115 & 86 & Email & 75\% & 9.6 \\ \hline
    \multirow{2}{2cm}{Parker (1992)}  & \multirow{2}{2cm}{Employees of AT\&T} & \multirow{2}{2cm}{Internal Communication} & 70 & 27 & Mail & 38\% & NA \\ \cline{4-8}
	&  &  & 70 & 48 & Email & 68\% & NA \\ \hline
    \multirow{2}{2cm}{Schuldt \& Totten (1994)}  & \multirow{2}{2cm}{Marketing \& MIS Professors (US)} & \multirow{2}{2cm}{Shareware Copying} & 200 & 113 & Mail & 56.5\% & NA \\ \cline{4-8}
	&  &  & 218 & 42 & Email & 19.3\% & NA \\ \hline
    \multirow{2}{2cm}{Mehta \& Sivadas (1995)}  & \multirow{2}{2cm}{Usenet Users} & \multirow{2}{2cm}{Internet Communication} & 309 & 173 & Mail & 56.5\%* & NA \\ \cline{4-8}
	&  &  & 182 & 99 & Email & 54.3\%* & NA \\ \hline
    \multirow{2}{2cm}{Tse, et al (1995)}  & \multirow{2}{2cm}{University Population (HK)} & \multirow{2}{2cm}{Business Ethics} & 200 & 54 & Mail & 27\% & 9.79 \\ \cline{4-8}
	&  &  & 200 & 12 & Email & 6\% & 8.09 \\ \hline
    \multirow{2}{2cm}{Bachman, Elfrink \& Vazzana (1996)} & \multirow{2}{2cm}{Business School Deans} & \multirow{2}{2cm}{TQM} & 224 & 147 & Mail & 65.6\% & 11.18 \\ \cline{4-8}
	&  &  & 224 & 117 & Email & 52.5\% & 4.68 \\ \hline
    Sheehan \& Hoy (1997) & University Population (Southeast US) & Privacy and New Technology & 580 & 274 & Email & 47.2\% & 4.7 \\ \hline
    \multirow{2}{2cm}{Smith (1997)} & \multirow{2}{2cm}{Web presence} & \multirow{2}{2cm}{Business Activities} & 150 & 11 & Email survey & 8\% & NA \\ \cline{4-8}
	&  &  & 150 & 42 & Email solicit & 11.3\% & NA \\ \hline
	\multirow{4}{2cm}{Schillewaert, Langerak and Duhamel (1998)} & \multirow{4}{2cm}{Web users in Belgium} & \multirow{4}{2cm}{Attitudes toward the Web} & 430 & 125 & Email & 31\% & NA \\ \cline{4-8}
	&  &  & 62.5M & 110 & Ad in magazine & 0\% & NA \\ \cline{4-8}
	&  &  & 4000 & 67 & USENET Posting & 2\% & NA \\ \cline{4-8}
	&  &  & 7500 & 51 & Hyperlinks & 0.68\% & NA \\ \hline 
    \multirow{4}{2cm}{Weible and Wallace (1998)} & \multirow{4}{2cm}{MIS Professors (US)} & \multirow{4}{2cm}{Internet Use} & 200 & 70 & Mail & 35.7\% & 12.9 \\ \cline{4-8}
	&  &  & 200 & 50 & Fax & 30.9\% & 8.8 \\ \cline{4-8}
	&  &  & 200 & 48 & Email & 29.8\% & 6.1 \\ \cline{4-8}
	&  &  & 200 & 52 & Web form & 32.7\% & 7.4 \\ \hline
    \multirow{2}{2cm}{Schaefer and Dillman (1998)} & \multirow{2}{2cm}{University Faculty} & \multirow{2}{2cm}{Unknown} & 226 & 130 & Mail & 57.5\%* & 14.39 \\ \cline{4-8}
	&  &  & 226 & 131 & Email & 58.0\%* & 9.16 \\ \hline
    \end{longtable}
\end{center}


In addition to the speedy response time of the email surveys, cost benefits have been indicated in \citeauthor{Sheehan2006}'s \citeyearpar{Sheehan2006} study. They also concluded that email is considered as an extremely cost-efficient method for data collection, where the total cost estimated at \$470 (\$30 for printing out the responses, \$440 for the 22-hour computer usage on downloading surveys for printing), while postal mail costs were estimated at \$6,500 (printing, postage, survey, and reminder mailing).
\vspace{1cm}

In another study from \cite{Mavis1998}, email survey was considered to be as nearly as seven times more cost efficient compared to a postal survey. This includes labor hours, survey materials like booklets, mailing labels, envelopes, and postage costs. Total time spent for the postal survey was 33 hours, but it only required 12 hours for the email survey. Final cost was \$503.36 for postal survey -- \$305.36 of which for postage and the remaining \$198 for student labor costs. Now on the other hand, email survey costs amounted to only \$72 in total.
\vspace{1cm}

Moreover, \cite{Paolo2009} reported that their respondents made longer comments to open-ended questions for the email version of the survey, compared to the regular mail version. While the average number of words per comment was 58.33\% in the mail version, the average for the email version was 75.40\%. \cite{BachmannD.ElfrinkJ.&Vazzana1999} had similar findings on their studies conducted on 1995 and 1998, where open-ended questions were more likely to be responded on by email recipients than by mail recipients. In the latter study conducted in 1998, the researchers also found out that email respondents were more likely to expand their answers, even if it was not suggested on the survey, resulting in a more candid set of responses compared to the set of responses on mail surveys. Responses to open-ended questions are one of most the important measures on determining the quality of the returned surveys.
\vspace{1cm}

Given these advantages and positive benefits of email surveys, the next section will provide information on survey errors.

\subsection{Survey Errors}
\label{sec:2.1.2:SurErr}
Sample surveys are quantitative estimations of the distribution of a characteristic in a population by obtaining this information from a small portion of the corresponding population \citep{Dillman1991}. To generalize results from a small portion, which is a sample, to a population, following sources of errors needs to be considered \citetext{\citealp[page 9]{Dillman2006}; \citealp{Dillman1991}}:

\paragraph{Sampling Error}
The greater number of people surveyed, the larger degree of precision can be achieved. Therefore, limitations on the number of people surveyed are considered under the sampling error. For example, while public opinion of 100 people results \(\pm\textdollar10\%\) of the true percent, 2,200 people results higher confidence with the percent of \(\pm\textdollar2\%\) \citep[page 9]{Dillman2006}. Surveys relying on a predefined list of recipients considers that the list is randomly generated or with a systematic sampling. Hence, it has got little research to reduce sampling errors compared with face-to-face interviews in which multistage cluster designs\footnote{Cluster sampling selects preexisting groups of population elements instead of a single element of the population \citep[page 106]{Groves2009}. Departments of a university or households in a block represents clusters of people. When the allocation of those sampling resources are stratified and based on multiple stages, frequently three stages, it is called multistage cluster sampling. First step selects the sample of counties, followed by the blocks within those counties, and finally the dwellings from the chosen blocks \citep{Scott1969}.} are used due to cost and time limitations \citetext{\citealp[page 106]{Groves2009}; \citealp{Dillman1991}}.

\paragraph{Coverage Error}
When the list of surveyed people does not include all elements of the population, coverage error happens \citep[page 9]{Dillman2006}. Coverage error is considered as one of the biggest issues of surveys ever since while surveying the general public \citep{Dillman1991}.

\paragraph{Measurement Error}
When a respondent's answer is hard to evaluate or cannot be compared with the other respondent's answers or there are inconsistencies between the observable variables like opinions, behaviors, or attributes and the survey responses, measurement error happens \citetext{\citealp[page 9]{Dillman2006}; \citealp{Dillman1991}}. The possible reason might depend on poor wording, wrong order of the questions or the characteristics of the surveyed person, such as incapability to provide correct answers or motivational factors \citep{Dillman1991}.

\paragraph{Nonresponse Error}
When there is a large amount of people who would not provide a response and their characteristics are different from the ones who responded, then it results to a nonresponse error \citep[page 9]{Dillman2006}. Low responses are considered as a major problem, and many researches have focused on improving the response rates \citep{Dillman1991}.

\section{Response Rate Influences}
\label{sec:2.2:ResRatInf}

As mentioned in the previous section, one of the survey errors is the nonresponse error. Researchers have concerns regarding response rates, since responses coming from survey participants may be substantially different from those of non-respondents, which will result in a biased estimate of representation of the population \citep{Bogen1996}.
\vspace{1cm}

Low response rate was even considered a shortfall of the email methodology despite to its advantages \citep{BachmannD.ElfrinkJ.&Vazzana1999}. In table~\ref{tab:sur_res_met}, there are nine studies, where both postal mail and email are compared side by side. Out of those nine studies, four of them showed a high response rate on the postal mail, three of them got a higher response on email and two studies did not show any significant differences. Parker's (1992) study of AT\&T employees was the only study that was able to get an acceptable high response rate by email. \cite{Schaefer1998} attributed this fact to the novelty of email and that sent emails were carefully examined instead of considered as a company junk email. \cite{Mavis1998} stated that studies cited by the others in support of email surveys, as also shown in table~\ref{tab:sur_res_met}, was not able to compare email data collection with the more traditional methods, and their study design and analyses varied greatly. \cite{Sheehan2006} also focused the attention on many of these studies' small and homogeneous population; therefore, it may not be applicable to represent larger population groups' response tendencies.
\vspace{1cm}

Hence, researchers investigated on how to increase the response rates for email communications. \cite{Schaefer1998} concluded that even though the technology for email is quite different from the well-established postal mail surveying methods, the communication itself is considered to be similar to self-administrated questionnaires delivered by post. Hence, the techniques used in increasing response rates on postal mail can be applied to develop an email methodology. The following techniques indicated below are where the researchers focused on, to evaluate their effects on response rates.

\subsection{Length}
\label{sec:2.2.1:Leng}
For many people, the total amount of time spent on conducting surveys is considered the biggest cost \citep[page 26]{DillmanDonA.SmythJoleneD.Christian2009}. The study from \cite{Heberlein1978} also states that the length of the survey has a negative effect on mail survey response rates, where they stated that each additional question reduces responses by .05\%. On the other hand, \cite{Bradburn1978} suggested that the length of the survey is correlated with its importance; therefore, it will increase the efforts both on the researchers' and respondents' side, resulting to a higher response rate. \cite{Bogen1996}, in his literature review, concluded that the relationship between the interview length and the nonresponse rate is weak and inconsistent.

\subsection{Multiple Contacts}
\label{sec:2.2.2:MultCont}
The researchers found out that the number of attempts in contacting people increases the response rates \citep{Heberlein1978,Schaefer1998}. The scenarios for multiple contacts include pre-notification contact, which is a brief notice for the main request, and follow-up contacts, aimed for the people who did not respond upon the initial contact. Heberlein and Baumgertner (1978) showed that follow-up mailing has a mean return rate of 19.9\% at the initial contact, and continued on with 11.9\% and 10.0\% for the second and third contacts, respectively \citep{Heberlein1978}. Schaefer and Dillman (1998) also stated that the same conclusion applies for the multiple contacts for email in their literature research. According to this, the average response rate for email surveys with a single contact was 28.5\% while 41\% and 57\% for two and more than two contacts, respectively \citep{Schaefer1998}.

\subsection{Personalization}
\label{sec:2.2.3:Pers}
Personalization has been addressed as an important factor in increasing response rates by many researchers \citep{Dillman1991,Schaefer1998}. It builds a connection between the respondent and researcher, by making the respondent feel important and drawing the respondent from out of the group \citep[page 272]{DillmanDonA.SmythJoleneD.Christian2009}. \cite{Dillman1974a} conducted a study to see the effects of personalization, where they reached half of a university's alumni sample via personalized cover letters, while the other half got impersonalized letters. The personalization treatment included personal salutations and real signatures affixed on the letters. They have achieved nearly 9\% greater response rates for the personalized group. It is also stated that this type of personalization techniques can be also applied to emails \citep{Schaefer1998}. In the next section, we will continue with the application of personalization in emails, and give the results of some studies.

\section{Personalization of Emails}
\label{sec:2.3:PersEmai}
Studies on mail surveys showed that personalization helps increases the response rates \citep{Dillman1991,Schaefer1998}. Personalization is also important for email communication since it builds a connection between the respondent and researcher as in the mail surveys studies, and make them feel more important and valued \citep[page 272]{DillmanDonA.SmythJoleneD.Christian2009}. With this argument, \cite{DillmanDonA.SmythJoleneD.Christian2009}, emphasized the social exchange theory\footnote{Social exchange theory was considered as a frame of reference to other theories rather than a theory by itself. It implies a two-sided, mutually contingent and rewarding transactions or exchanges \citep{Emerson1976}.} of the personalization of the email.
\vspace{1cm}

On the other hand, \cite{Barron2002} stressed on the socio-psychological phenomenon, the diffusion of responsibility, which is also an outcome of a volunteer's dilemma. With a volunteer's dilemma, one player is needed to volunteer in order to reach the outcome preferred by everyone else in the game. However, each person might be inclined on hoping that someone else will volunteer, resulting to a scenario of a higher instance of not volunteering, rather than volunteering. According to this, the greater the number of people in the group size, the lesser probability of volunteering will result, which will then produce the diffusion of responsibility effect. In order to experiment on the effect of diffusion of responsibility in the context of email requests, they sent several emails asking for help either to a single address, or to a list of five addresses. In the email body (see Appendix~\ref{app:EmaiVolunDilem}), a fictitious graduate student asked a question to know if the university has a biology faculty, whose answer is actually a given to anyone familiar with the institute. The result of the study showed that the number of replies sent to a single email address per email got a 20\% higher response rate than the number of replies sent to a group of email addresses per email. In addition to this, the study classified the given responses according to its level of helpfulness, and the rate of "very helpful" replies retrieved from the emails sent to a single address per instance was 187\% higher compared to the responses retrieved from emails sent to a group of email addresses per instance.
\vspace{1cm}

Another outcome regarding the use of multiple email addresses in the "To" field resulted concerns from respondents in the study of \cite{Selm2006}. An introductory email including a link to a web-based questionnaire was sent to recipients to explore the opinions of elderly Internet users about an electronic political debate. One of the respondents raised his privacy and confidentiality concerns when the header of the email contained all the email addresses of all of the respondents, explicitly. His reaction was quoted in the study as in listing~\ref{lst:QuotReacConf}.
\vspace{1cm}

\clearpage

\lstset {
 %basicstyle=\footnotesize, basicstyle=\ttfamily
 frame=shadowbox,
 rulesepcolor=\color{black},
 showspaces=false,showtabs=false,tabsize=4,
 numberstyle=\tiny,
 captionpos=b,
 abovecaptionskip=1\baselineskip,
 breaklines=true,
 keepspaces=true,
 breakindent=0pt,
 columns=flexible
}

\begin{lstlisting}[language=, caption={[A Respondent's Reaction Regarding Confidentiality]A Respondent's Reaction Regarding Confidentiality \citep{Selm2006}}, label={lst:QuotReacConf}]
"Well, it could be good (for you) to fill in this form, but I better not. Do you want to know why? 'All responses will be treated confidentially', but what do I see in the address column? I see all the email addresses of those you've sent this message to. Do you folks call that confidentiality!? I've decided not to participate in this 'carefully composed' study, although I do have an opinion on the subject matter."
\end{lstlisting}

Even though the authors believed that the person was just "skeptical" and his reaction displayed a "vivid skepticism". To this date, one of the biggest concerns involving the whole email medium is confidentiality, which might result into very embarrassing situations, including invasion of privacy involving anything, from doing research up to business perspectives. A very recent email message (See listing~\ref{lst:EmaiMessaConfi} for the excerpt) dropped to my email inbox verifies the importance of confidentiality.  
\vspace{1cm}

\begin{lstlisting}[language=, caption={[An Email Message Showing the Importance of Confidentiality]An Email Message Showing the Importance of Confidentiality}, label={lst:EmaiMessaConfi}]
Dear Valued Customer,

Earlier today the email seen bellow was inadvertently sent without utilizing 'Bcc' recipients.

Our sincerest apologies for any inconvenience this may have caused you.

Kind Regards
\end{lstlisting}

In another study by \cite{Heerwegh2005}, personalization is applied to salutations in emails. The randomly drawn 2,540 samples from the student database of Katholieke Universiteit Leuven, Belgium were separated into two equally sized groups. For the non-personalized group, the salutation of "Dear student" was used, while in the personalized group "Dear [First name] [Last name]" was used. The email content was an invitation to a web survey which was about adolescent attitudes towards marriage and divorce. The result of the study showed that the personalization applied group got a 6.9\% higher survey login rate than the non-personalized group. Therefore, they concluded that increased response rates were in line with the social exchange theory and the diffusion of responsibility theory.
\vspace{1cm}

In addition to the personalization of salutations on emails, \cite{Joinson2007} stated the power of its combination with the power or status of the sender. In the study, groups of discussion panels composed of students from the Open University in UK were sent an email invitation to complete a survey. Panel members were assigned on one of the conditions, and salutations were modified to "Dear student", "Dear John Doe", and "Dear John". The sender power were manipulated on the first and last lines of the emails by assigning a neutral power, saying that "From <name> (Strategy, Planning, and Partnerships), The Open University" and a high power "From Professor <name>, Pro-vice chancellor (Strategy, Planning, and Partnerships), The Open University". The results showed that the highest response rate was achieved when a personalized invitation came from a high power source and the lowest when an impersonal one came from a neutral power source (See table~\ref{tab:pow_sal_res}). The possible reason for this was suggested that as personalized salutations increase one's sense of identifiability, its combination with a higher power audience increases, giving them a sense of being socially desirable, a strategic behavior.

\begin{table}[!ht]
\begin{center}
	%\renewcommand{\arraystretch}{2}
	%\tiny
	%\setlength{\tabcolsep}{5pt}
	\caption[Power, salutation and response rates (raw and \%)]{Power, salutation and response rates (raw and \%) \citep{Joinson2007}} \label{tab:pow_sal_res}
    \begin{tabular}{ p{3cm} p{3cm}  p{3cm}  p{3cm} }
	\hline
	& \textbf{Dear Student} & \textbf{Dear John Doe} & \textbf{Dear John} \\ \hline
	\textbf{Neutral power} & 143 (40.1) & 158 (44.4) & 166 (46.6) \\
	\textbf{High power} & 150 (42.1) & 154 (43.3) & 190 (53.4) \\ \hline
    \end{tabular}
\end{center}
\end{table}

The aforementioned studies showed that different forms of personalization help increase the response rates on email communications. However, it has become very easy to add personalized information into emails thanks several third-party software. \citet[page 237-238]{DillmanDonA.SmythJoleneD.Christian2009} stated that over-personalization using software tools can easily result to impersonal messages, similar to the example given below (See listing~\ref{lst:SampOverPersEmai}).
\vspace{1cm}

\clearpage

\lstset {
 %basicstyle=\footnotesize,
 frame=shadowbox,
 rulesepcolor=\color{black},
 showspaces=false,showtabs=false,tabsize=4,
 numberstyle=\tiny,
 captionpos=b,
 abovecaptionskip=1\baselineskip,
 breaklines=true,
 keepspaces=true,
 breakindent=0pt,
 columns=flexible
}

\begin{lstlisting}[language=, caption={[A Sample for an Over-personalized Email]A Sample for an Over-personalized Email \citep[page 237-238]{DillmanDonA.SmythJoleneD.Christian2009}}, label={lst:SampOverPersEmai}]
Dear Don Dillman,

I am writing to inform you and your wife Joye that the XYZ Company has created a new dog food that we are sure your Boston Terrier, Crickett, will find to be very tasty. 

We would like to send a free sample to your home in Pullman, Washington.

Kind regards,
XYZ
\end{lstlisting}

In this message, there is an overwhelmed personalization with the usage of person's wife, their dog's type and name, and their home address. Moreover, experienced email users can easily identify if a message is written by a person or if it is just a computer generated one by looking at the appearance of one's name on certain locations, and similar patterns for other information \citep[page 272]{DillmanDonA.SmythJoleneD.Christian2009}. Therefore, it is difficult to have a correct amount and tone of personalization. The more daily interaction with digital devices there is, the more it will make true and authentic personalization rare, hence achieving such will make it more important and effective \citep[page 238]{DillmanDonA.SmythJoleneD.Christian2009}.

\section{Conculusion}
\label{sec:2.4:Conc}

In conclusion, the researchers conducted more mail survey studies than other survey methods to further investigate data collection \citep{Dillman1991}. Some of those studies tried to answer the question of nonresponse error, which has been considered as a major problem, compared to other survey errors as discussed in section~\ref{sec:2.1.2:SurErr}. According to the mail survey studies, personalization has been addressed as an important factor in increasing response rates by many researchers in addition to other influences affecting response rates as identified in section~\ref{sec:2.2:ResRatInf}. With the advance of world-wide internet usage, many researchers started to consider email as a form of a data collection method, because of its cost and speed benefits compared to other data collection methods as discussed in section~\ref{sec:2.1.1:EmaSur}. However, some studies showed that response rates on email surveys are lower than that of regular mail surveys despite of its advantages; in addition, it may pose as a burden to the researchers during the collation of responses since email communication does not emphasize on any structure, like in web forms or even respondents may come up with additional clarifying questions \citep{Selm2006}. Therefore, even if the technology for email is different from regular mail surveying methods, the researchers considered the response rate influences of regular mail surveys for email since the communication itself is the same in nature. In section~\ref{sec:2.3:PersEmai}, several studies applying different types of personalization were mentioned. Some of those studies modified the header of the emails to study the diffusion of responsibility. Other studies changed the salutations and signatures of the emails, which resulted to an increased response rates to the emails. On the other hand, those studies did not consider the increased awareness of their recipients to the possibility of computerized personalization techniques, which resulted in over-personalized emails. Also, none of the studies gave attention to the personal efforts of a researcher while extracting information from respondents' answers. This study will try to focus on the shortcomings of those studies as well, and provide a web application in attempt to overcome those issues.
\vspace{1cm}

In the next section, existing applications in the market, which leverages the email communication, will be evaluated. While some of them would just focus only on the email communication like email marketing applications, other applications like \ac{CRM} and help desk applications helped this study on identifying useful features that can be deemed helpful in the area of personalized email communications.  


\begin{comment}

--> At the end of the chapter 2, a brief summary/conclusion could be helpful. Thereby, you could also put the related work in perspective to your problem statement, i.e. explaining where the existing work falls short or can be improved (by your solution).


 .... talk about dillman and finish, it is opposite idea, so keep it in this section, sonra overview of tools, de ama once be review et hocaya yolla

--> Respondendts start asking questions to clarify things E21 p446
--> bunu sorunlar kisminda yazabilirisin, niye CRM ihtiyac duydugunu acikliyor: personalization lost its effect on recent years cause of programs p152 dillman ebook 2007 - dillman kitap p237
--> E14 p379 da mail be email aslinda ayni diyor. E18 p1371 altini maviyle cizdin
--> M1, there is a table showing response rate comparison of email vs paper for many studies. USE IT
--> Mention about which of the errors are not applicable to us. Eg.g sampling since we can predefined list of people
--> 10.1.1.189.3715.pdf "Assistance: The Work Practices of Human Administrative Assistants and their Implications for IT and Organizations"speak about the assistance support
--> Bi section adi bu olsun, survey uzerine yazdiklarin bittikten sonra: The issues with email communication
--> Burden on the Researcher E21p442, bunu sonra anlat IYI

--> Talked about the 4 issues of survey in here, and those issues are applicable to emails
--> Compare Web surveys and EMail
--> write about low response rate comparing with mail
--> 7 pages, then comparison of existing products, then crowsourcing (MAYBE you can write about assistant support using Michael's paper), then first prototype, then final, mention about the improvements
--> A chapter about assistance support 10.1.1.189.3715.pdf "Assistance: The Work Practices of Human Administrative Assistants and their Implications for IT and Organizations"
--> Write about people's email life PIP_Work_Email_Report.pdf.pdf


Previously known data collection methods like regular which includes questionnaires where respondents fill and post back, or phone interviews, or face-to-face interviews have got additional options by the development of technology like web and email.
\vspace{1cm}

A first comprehensive high response rate achieved mail survey is done at 1978 \citep[page 3]{Dillman2006}.
The first web survey, which was conducted on 1994 by James E. Pitkow and Margaret M. Recker, had an aim to demonstrate that the web as a survey medium. During that that estimated internet users were only 15 million \citep{Lewis1993}. The survey was posted one of the newsgroup, and got 4,777 responses during its one month online period.

\end{comment}

%%% Local Variables: 
%%% mode: latex
%%% TeX-master: "thesis"
%%% End: 